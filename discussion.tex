% vim: ft=tex
\chapter{Discussion}
% TODO something like a SWOT analysis here (strengths, weaknesses, opportunities, threats)
% TODO general advice: be concise, brief, and specific
% TODO: modifications to the same DIM objects from different actors at the same time => one update will be lost, but consistency is still guaranteed

\section{Value Added}
% TODO what's better than before

\section{Limitations}
% TODO identify potential limitations and weaknesses of the product
Limitations imposed by the design and implementation of the new features are
discussed here.

\subsection{Federation}
\subsubsection{DIM replication}
Eventual consistency is guaranteed even in situations where multiple actors
modify the same DIM objects at roughly the same time. However, some updates
might get lost. In those cases, the last writer wins.

\subsubsection{Message routing}
% TODO: would have been easier if CORE had inter-node sockets itself, but:

\subsection{HA}
A HA cluster has to be complete (both nodes running) during initialization.
Otherwise, only primary node can start to serve requests; the backup node will not
do this.

\subsection{Persistence replication}
Message traffic towards root node sums up because of persistence
replication. This should not be a problem because of \zmq's brilliant
message batching, so the real limit only given by the inter-node network links.

\section{Business Benefits}
Using the new federation capability, Roadster can be deployed in larger
environments, possibly replacing higher-level systems described in
\autoref{sec:scope:sys-integration}.

\section{Ideas for Improvement}
\subsection{High availability}
\subsubsection{Thoughtful heartbeating}\label{sec:discussion:imp:ha:hb}
The BSTAR actor currently happily exchanges heartbeats with the BSTAR actor
running on the other HA peer, even in case all actors but CORE are dead. The
CORE actor's health is periodically checked using the PING/PONG mechanism as
described in \autoref{sec:approach:ha:hb}.

In a subsequent version of Roadster, the CORE actor could relay those PING
requests to the other actors and only respond with a PONG in case the other
actors all have responded with PONG. This mechanism would make the node only
signal life signs in case all actors are healthy.

\subsubsection{HA within a node}
Building upon the previous idea, unhealthy actors could be killed and respawned
by the CORE actor.  Microrebooting unhealthy components without an attempt at
any sophisticated recovery would make Roadster belong to the \emph{crash-only
software} kind of fault-tolerant software.

\subsubsection{Manual failover}\label{sec:discussion:ha:manual-failover}
A manually induced failover could be useful in some cases, e.g. when a
component fails that does not provoke an automatic failover.

Providing a less brutal way of inducing a failover than to turn off the
currently active HA peer could be interesting in a future version.

\subsection{Miscellaneous}
Unordered list of ideas:
\begin{itemize}
	\item switch to Moneta for a unified key-value store interface, then eventually away from \gls{tc} to something more modern and maintained, like LMDB (it is super fast and crash-proof)
	\item TIPC: high performance cluster communication protocol, suitable because Roadster nodes are Linux and there are direct links to peers (required for TIPC)
	\item key management in a DB (instead of files), with GUI to accept new clients
	\item dynamic node topology (maybe via DSL-file in Etcd, or DIM-only, or Zookeeper)
	\item other method for data serialization (like MessagePack), would allow adding other programming languages to the cluster
	\item fast compression for messages, like LZ4 or Snappy
	\item SERVER/CLIENT sockets from ZMQ 4.2 for simplified message routing
	\item Ruby \rb{Integer} class instead of \rb{Fixnum}
	\item (use more symbols instead of strings, e.g. to refer to a certain actor)
	\item \sh{ZMQ_RECONNECT_IVL} and \sh{ZMQ_RECONNECT_IVL_MAX} especially for mobile clients
\end{itemize}

