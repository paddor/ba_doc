% vim: ft=tex
\documentclass[a4paper]{report}
\usepackage[english,ngerman]{babel}
\usepackage[utf8]{inputenc}
\usepackage[a4paper]{geometry}
\geometry{verbose, marginparwidth=15mm, marginparsep=3mm, tmargin=25mm}
\usepackage[svgnames, table]{xcolor}
\usepackage{graphicx}
\usepackage[hyphens]{url}
\usepackage[flushmargin,hang]{footmisc}
\usepackage{hyperref}
\usepackage{xspace} % space, but not before punctuation
\usepackage[hypcap]{caption} % link to top of tables and figures, not to bottom caption
\usepackage{microtype} % letter spacing

% paragraph spacing, no indentation, with normal list spacing
\usepackage[parfill]{parskip} % no idented first line of each paragraph
\usepackage{enumitem}
\newlength\docparskip
\parskip=10pt
\setlength{\docparskip}{\parskip}
\setlist{nosep, itemsep=5pt, parsep=0pt, before={\parskip=5pt}, after={}}

\usepackage{multirow}
\usepackage{longtable}
\usepackage{pgfgantt}

\usepackage{amssymb} % for \checkmark
\usepackage{textgreek} % for \textMu
\usepackage{pifont}% http://ctan.org/pkg/pifont, for \ding{#}
\usepackage{enumitem} % for \begin{itemize}[label={...}]
\usepackage[super]{nth} % for \nth{3} => 3rd with superscript
\usepackage[export]{adjustbox} % left/right aligned images
\usepackage{newfloat}
\usepackage{tikz,pgfplots}
\usetikzlibrary{arrows,decorations.pathmorphing,backgrounds,fit,positioning,shapes.symbols,chains,shapes.geometric,shapes.arrows,calc}
\usepackage[
backend=biber,
hyperref=true,
url=true,
isbn=false,
backref=false,
% style=custom-numeric-comp,
citereset=chapter,
maxcitenames=3,
maxbibnames=100,
block=none]{biblatex}
\bibliography{main}
\hypersetup{
	unicode=true,
	colorlinks=true,linkcolor=blue!75!black,citecolor=blue!75!black,urlcolor=blue!75!black,
	pdftitle={Roadster High Availability},
	pdfsubject={Extension of a SCADA Framework to support High Availability and Authenticated Encryption},
	pdfauthor={Patrik Wenger, Manuel Schuler},
	pdfkeywords={ruby} {zmq} {czmq} {cztop} {high availability} {security} {encryption} {UPC UA} {SCADA} {C++},
}
\usepackage{csquotes}
\usepackage{pdfpages}
\usepackage[toc,xindy]{glossaries}
\makeglossaries

\usepackage{MnSymbol}


%----------------------------------------------------------------------------
\usepackage[newfloat]{minted}
\usemintedstyle{lovelace}
%colorful % quite good
%fruity % bad
%manni % okay, but purple features
%lovelace
\definecolor{bg}{rgb}{0.96,0.96,0.96}


% inline Ruby
\newcommand{\rb}[1]{\mintinline[bgcolor=bg]{Ruby};#1;}

% inline shell
\newcommand{\sh}[1]{\mintinline[bgcolor=bg]{sh};#1;}

% inline C++
\newcommand{\cpp}[1]{\mintinline[bgcolor=bg]{cpp};#1;}
%----------------------------------------------------------------------------

% canonical ZMQ spelling with \zmq
\newcommand\zmq{{\O}MQ\xspace}

% adding source as caption not shown in LOF
\newcommand{\source}[1]{\vspace{-15pt}\caption*{\hfill \scriptsize Source: {#1}} }

% checkmark and crossmark
\newcommand{\cmark}{\ding{51}}
\newcommand{\xmark}{\ding{54}}


% 150% of vertical spacing between table rows
\renewcommand{\arraystretch}{1.5}

\title{Roadster High Availability}
\author{Patrik Wenger, Manuel Schuler}

\begin{document}
\selectlanguage{english}
% vim: ft=tex

% general
\newacronym{HA}{HA}{high availability}
\newacronym{OPC}{OPC}{Open Platform Communications}
\newacronym{UA}{UA}{Unified Architecture}
\newacronym{RUP}{RUP}{Rational Unified Process}
\newacronym{TIPC}{TIPC}{Transparent Inter-Process Communication}
\newacronym{TCP}{TCP}{Transmission Control Protocol}
\newacronym{PGM}{PGM}{Pragmatic General Multicast}
\newacronym{MOM}{MOM}{Message Oriented Middleware}
\newacronym{BSD}{BSD}{Berkeley Software Distribution}
\newacronym{PLC}{PLC}{Programmable Logic Controller}
\newacronym{SCADA}{SCADA}{Supervisory Control and Data Acquisition}
\newacronym{DSL}{DSL}{Domain Specific Language}
\newacronym{CHP}{CHP}{Clustered Hashmap Protocol}
\newacronym{UI}{UI}{user interface}
\newacronym{FFI}{FFI}{Foreign Function Interface}
\newacronym{ZAP}{ZAP}{ZMQ Authentication Protocol}
\newacronym{SOAP}{SOAP}{Service Oriented Application Protocol}
\newacronym{VM}{VM}{virtual machine}
\newacronym{TDD}{TDD}{test-driven development}
\newacronym{CI}{CI}{continuous integration}
\newacronym{PC}{PC}{personal computer}
\newacronym{IoT}{IoT}{Internet of Things}
\newacronym{SSD}{SSD}{Solid State Disk}
\newacronym{RAID}{RAID}{redundant array of independent disks}
\newacronym{FEDRO}{FEDRO}{Federal Roads Office}
\newacronym{IETF}{IETF}{Internet Engineering Task Force}
\newacronym{RFC}{RFC}{Request for Comments}
\newacronym{IEC}{IEC}{International Electrotechnical Commission}
\newacronym{ECC}{ECC}{elliptic curve cryptography}
\newacronym{ASCII}{ASCII}{American Standard Code for Information Interchange}
\newacronym{SSH}{SSH}{Secure Shell}
\newacronym{UUID}{UUID}{Universally unique identifier}
\newacronym{OOP}{OOP}{object-oriented programming}
\newacronym{IP}{IP}{Internet Protocol}
\newacronym{NIC}{NIC}{network interface card}
\newacronym{UPS}{UPS}{uninterruptible power supply}
\newacronym{CPU}{CPU}{central processing unit}
\newacronym{VLAN}{VLAN}{Virtual Local Area Network}

% German
\newacronym{ASTRA}{ASTRA}{Bundesamt f\"ur Strassen}
\newacronym{LTA}{LTA}{Leittechnikanlage}
\newacronym{AR}{AR}{Abschnittsrechner}
\newacronym{AS}{AS}{Anlagesystem}
\newacronym{LR}{LR}{Leitrechner}

% Roadster terminology
\newacronym{DIM}{DIM}{Domain Information Model}
\newacronym{CSP}{CSP}{Clone State Protocol}
\newacronym{RMP}{RMP}{Roadster Messaging Protocols}
\newacronym{ACP}{ACP}{Application Control Protocol}
\newacronym{CSP}{CSP}{Clone State Protocol}
\newacronym{PCP}{PCP}{Peer Control Protocol}
\newacronym{SMP}{SMP}{Supress Management Protocol}
\newacronym{PDP}{PDP}{Persistent Data Protocol}

\newglossaryentry{zmq}{
	name={\zmq},
	description={High-performance \gls{MOM} and concurrency framework,
	implemented as a standalone library},
	sort={0MQ}
}
\newglossaryentry{ruby}{
	name={Ruby},
	description={An interpreted, expressive, general-purpose \gls{OOP}
	language, created by Yukihiro Matsumoto}
}
\newglossaryentry{c}{
	name={C},
	description={A compiled, imperative, very influential low-level
		programming language, invented in the early 1970s as a Unix system
		programming language. Compared to other languages, it very simple,
		knows only a handful of primitives and keywords}
}
\newglossaryentry{actor-model}{
	name={Actor Model},
	description={A mathematical model for concurrent computation where
		there's no shared state and all communication between actors
		happens through messages}
}
\newglossaryentry{zguide}{
	name={Zguide},
	description={An extensive online
		document\footnote{\url{http://zguide.zeromq.org/page:all}} describing
		best-practice patterns for \gls{zmq}}
}
\newglossaryentry{bstar}{
	name={Binary Star Pattern},
	description={A fairly simple hot-standby and failover mechanism to
		achieve high availability between two servers, described as a reliable
		request-reply pattern in the \gls{zguide}}
}
\newglossaryentry{hot-standby}{
	name={hot standby},
	description={A method of high availability by introducing redundancy,
where the secondary system is also up and running, but just doesn't process
requests before the primary system fails. Failover time is typically a few
seconds}
}


\newglossaryentry{clone-pattern}{
	name={Clone Pattern},
	description={A client-server protocol to share state (a list of
		key-value pairs) across multiple clients, described as a reliable
		pub-sub pattern in the \gls{zguide}}
}

\newglossaryentry{unix-domain-socket}{
	name={Unix Domain Sockets},
	description={Named pipes for extremely performant, duplex inter-process
		communication on Unix systems}
}

\newglossaryentry{case}{
	name={case},
	description={An alarm in a Roadster application that needs to be confirmed}
}

\newglossaryentry{LOG}{
	name={LOG protocol},
	description={Used within Roadster for system logging}
}

\newglossaryentry{KISS}{
	name={KISS},
	description={The design principle ``Keep it simple, stupid'', which
		favors simplicity over complexity}
}

\newglossaryentry{wrapper-facade}{
	name={wrapper fa\c{c}ade},
	description={A structural software design pattern which provides an
		object-oriented fa\c{c}ade to a low-level functional subsystem or
		library}
}

\newglossaryentry{czmq}{
	name={CZMQ},
	description={A thin abstraction layer (\gls{wrapper-facade}) for
		\gls{zmq} with some additional functionality, written in clean and
		elegant C}
}
\newglossaryentry{cztop}{
	name={CZTop},
	description={A modern, \gls{FFI} based \gls{ruby} binding for \gls{czmq}, written by Patrik Wenger}
}

\newglossaryentry{nacl}{
	name={NaCl},
	description={Networking and Cryptography Library\footnote{\url{https://nacl.cr.yp.to}}. Modern, state-of-the-art cryptography library, created by the Daniel J. Bernstein}
}

\newglossaryentry{tweetnacl}{
	name={TweetNaCl},
	description={A compact, portable reimplementation\footnote{\url{http://tweetnacl.cr.yp.to}} of the NaCl in the
form of 100 tweets, suited to be included it into one's trusted code base (as
opposed to an external dependency). Implemented Daniel J. Bernstein et al}
}
\newglossaryentry{libsodium}{
	name={libsodium},
	description={A portable and installable variant of \gls{nacl}\footnote{\url{https://libsodium.org}}}
}
\newglossaryentry{websocket}{
	name={WebSocket},
	description={A protocol for full-duplex communication between web
browsers and web servers, standardized by the \gls{IETF} as \gls{RFC} 6455 in
2011}
}

\newglossaryentry{opc-ua}{
	name={OPC UA},
	description={\gls{OPC} gls{UA}: A set of modern standards for industrial control systems, based on cross platform webservices and other modern technology}
}

\newglossaryentry{iec-104}{
	name={IEC 60870-5-104},
	description={A \gls{IEC} transmission protocol used by SCADA applications in power system automation that enables communication via standard networks}
}

\newglossaryentry{modbus-tcp}{
	name={Modbus TCP},
	description={The TCP-based variant of Modbus, a \emph{de facto}
	standard serial communication protocol used to connect electronic devices}
}
\newglossaryentry{isa95}{
	name={ISA-95},
	description={An international standard\footnote{\url{https://en.wikipedia.org/wiki/ANSI/ISA-95}} for developing an automated interface between enterprise and control systems}
}

\newglossaryentry{ddos-attack}{
	name={distributed denial-of-service attack},
	description={is an attempt to interrupt the availability of a service
by flooding it with forged requests using a large number of source systems}
}

\newglossaryentry{Z85}{
	name={Z85 armor},
	description={a space efficient, \gls{ASCII} based, string-safe variant of the Base85 binary-to-text encoding}
}

\newglossaryentry{tc}{
	name={TokyoCabinet},
	description={a library to manage a key-value store in a single file (no server involved)}
}
\newglossaryentry{stdout}{
	name={STDOUT},
	description={the standard output channel of a Unix process (file descriptor 1)}
}
\newglossaryentry{gherkin}{
	name={Gherkin},
	description={A simple language to specify feature specifications in
	steps such as Given, When, Then\footnote{\url{https://cucumber.io/docs/reference}}}
}

\newglossaryentry{CLASS}{
	name={CLASS},
	description={C Language Style for Scalability. A standard to build
	simple, but scalable C libraries.\footnote{\url{https://rfc.zeromq.org/spec:21/CLASS}}}
}

\newglossaryentry{MRI}{
	name={MRI},
	description={Matz' Ruby interpreter. The standard Ruby interpreter,
	developed under the lead of Ruby creator Yukihiro "Matz" Matsumoto}
}

\newglossaryentry{JRuby}{
	name={JRuby},
	description={A Ruby interpreter written in Java}
}

\newglossaryentry{Rubinius}{
	name={Rubinius},
	description={A modern Ruby interpreter written in Ruby and a small C++ kernel}
}

\newglossaryentry{oom-killer}{
	name={OOM killer},
	description={Out-of-memory killer. A facility in Linux that starts
	killing appropriate processes in case the main memory becomes full.}
}

% TODO
% event-driven
% asynchronous

% TODO: streamlining:

% ROUTER & DEALER
% PUSH & PULL
% PUB & SUB
% federation & node, topology/hierarchy??
% subtree (only for DIM)
% cluster (for HA)
% COMM, CORE, STORAGE, LOG, BSTAR, ...
% subnode & supernode
% HA peer
% DIM objects
% DIM replication leads to DIM synchronization (sync across all nodes)


\thispagestyle{empty}

% nice looking cover page with help from https://en.wikibooks.org/wiki/LaTeX/Title_Creation
\begin{titlepage}
\centering
\begin{raggedleft}\includegraphics[trim=10 10 10 10, clip=true, width=0.3\textwidth]{img/hsr_logo.pdf}\end{raggedleft}
\begin{raggedright}\hfill\includegraphics[trim=14.8cm 27cm 1cm 1.4cm, clip=true, width=0.38\textwidth]{img/roadster_factsheet.pdf}\end{raggedright}

\vspace{50mm}
{\scshape\Large Bachelor thesis\\}
\vspace{2cm}
{\huge\bfseries Extension of a SCADA Framework to support High \textls[-150]{A}vailability and Authenticated Encryption\\}
\vspace{1cm}
{\huge\bfseries \#Ruby \#\O{}MQ \#NaCl\\}
\vspace{2cm}
{\Large\itshape Patrik Wenger, Manuel Schuler\\}
\vfill
\begin{center}
\begin{varwidth}{\textwidth}
\begin{description}
	\large
	\item [Client:] mindclue GmbH
	\item [Supervisor:] Prof. Dr. Farhad Mehta
	\item [Expert:] S\"oren Bleikertz
\end{description}
\end{varwidth}
\end{center}
\vfill
% Bottom of the page
{\large September -- December, 2016\\}
\end{titlepage}

%-----------------------------------------------------------------------------
\begin{abstract}
\pagenumbering{roman}

% introduction
Roadster is mindclue GmbH’s in-house framework to build modern supervision and
control applications used in various technical fields including traffic
systems, energy, and water supply. It is written in Ruby and uses \zmq to pass messages
within a system of loosely coupled actor processes following a
shared-nothing architecture.

The missing features include the ability to run a Roadster application in a
hierarchical federation spanning multiple autonomous nodes, high
availability, and secure inter-node communications.

% approach and technologies
The students gathered exact requirements, worked out soltutions fitting Roadster's
layered software architecture, and implemented them following software
engineering methodology. New and a significant amount of existing functionality
has been covered by comprehensive test suites ranging from unit tests to system tests.

% result
As a result, Roadster applications can now be declared and run in a federation.
State is shared through secure inter-node communications.  Highly
available nodes can be declared and run in the form of hot-standby
clusters.

\end{abstract}
%-----------------------------------------------------------------------------

\chapter*{Declaration of originality}
We hereby confirm that we are the sole authors of this document, the
described changes to the Roadster framework, and libraries developed as a
byproduct. Unless stated differently, all illustrations in this document are
our creations.

% TODO any usage agreements or license
%\includepdf[width=\textwidth,pages=1,pagecommand=\section{Permissions},trim=1.9cm 3cm 2cm 2cm,clip]{vereinbarung.pdf}

\chapter*{Acknoledgements}
We would like to thank Prof. Dr. F. Mehta for his competent guidance during
this bachelor thesis.  We highly value his opinions. Due to his polite
parlance, discussing project matters, both of the management and the technical
kind, has always been an enrichment.

We also thank Andy Rohr of the mindclue GmbH for his help and support in
finding our bearings within Roadster's codebase and his precious input in the
design process and decision making.

Special thanks to Pieter Hintjens {\textdagger} (3 December 1962 -- 4 October
2016) for his amazing work and contagious passion within the \zmq and
distributed computing communities. We send our deepest condolences to his
family. Rest in peace.


% TOC, LOF, LOT, LOL
\setcounter{tocdepth}{4}
\tableofcontents
\listoffigures
\listoftables
\listoflistings

\pagebreak
\pagenumbering{arabic}
\setcounter{page}{1}
\setcounter{secnumdepth}{3}

%-----------------------------------------------------------------------------
\part*{Management Summary}\label{part:mgmtsummary}
\setcounter{secnumdepth}{0} % avoid section numbering here

\section*{Initial Situation}
TODO describe initial situation, not too technical\\

Roadster is a next generation monitoring application.

\section*{Software Development Process}
TODO describe decision to use RUP/Scrum
TODO maybe describe what project management tools we'll be using

\section*{Project Phases}
TODO describe this phase in retrospection

\subsection*{Inception}
TODO include Gantt chart for this phase
TODO describe this phase in retrospection

\subsection*{Elaboration}
TODO include Gantt chart for this phase
TODO describe this phase in retrospection

\subsection*{Construction}
TODO include Gantt chart for this phase
TODO describe this phase in retrospection

\subsection*{Transition}
TODO include Gantt chart for this phase
TODO describe this phase in retrospection

\section*{Results}
TODO describe results


%-----------------------------------------------------------------------------
\part{Technical Report}
\chapter{This document}
This chapter briefly describes the conventions used in this document.

\section{Structure}
The content within each chapter begins with a brief overview,
followed by detailed, in-depth descriptions including illustrations.

\paragraph*{Pragraph title.} This is what a titled paragraph title looks. After
chapters, sections, and subsections, it is the smallest structural entity in
this document.
Lorem ipsum dolor sit amet, consectetur adipiscing elit, sed do
eiusmod tempor incididunt ut labore et dolore magna aliqua.

\section{Ruby}
\paragraph*{Language.} Ruby is an expressive, dynamically and strong typed, interpreted
programming language. Classes are defined and can be extended at runtime.
\autoref{lst:ruby:example} is an example of a Ruby code listing.

Commonly used core classes include:
\begin{itemize}
	\item \sh{Integer}:
		Arbitrary sized integers, such as \rb{42}.
	\item \sh{Float}:
		Limited precision floating point number, such as \rb{3.141}.
	\item \sh{String}:
		Mutable character sequences associated with an encoding.
		\rb{'foo'} is a string without interpolation, whereas
		\mintinline[bgcolor=bg]{Ruby};"bar #{name.capitalize}"; is a string with interpolation.
	\item \sh{Symbol}:
		Immutable value objects such as \rb{:foobar}. They are used for identification only,
		especially as keys in a \sh{Hash}. They are basically strings of
		which each only exists once in memory.
	\item \sh{Array}:
		Mutable, ordered sequences of any objects, such as \rb{[ 42, 3.14, 'foo', :bar]}.
	\item \sh{Hash}:
		Hash maps between keys-value pairs, such as
		\mintinline[bgcolor=bg]{Ruby};{ :foo => 'bar', :baz => 'quux' };.
		Keys and values can be arbitrary objects, but Symbols are
		commonly used as keys.
\end{itemize}

\paragraph*{Class names.} Classes and modules (mix-ins) are the only means of
namespacing in Ruby. An example of a fully qualified class name is \sh{Roadster::Actors::Base},
which, by convention, denotes the class name of the abstract\footnote{The concept of abstract classes
does not exist in Ruby as a language construct. Instead, a regular class is defined and merely used
for inheritance.} base class for all actors.

\paragraph*{Methods names.} An
instance method is denoted using \mintinline[bgcolor=bg]{Shell}|Class#method_name|, or simply
\mintinline[bgcolor=bg]{Text};#method_name; if the class
can be deduced from context. On the other hand, a class method is denoted as
\sh{Class.method_name}. Method names ending in \sh{=} are setters. Analogously, \sh{?}
is for Boolean query methods, and \sh{!} is for destructive methods, e.g. making a
modification in-place instead of on a returned copy. See
\autoref{lst:ruby:example} for example usages of these conventions.


\begin{listing}
	\begin{minted}[bgcolor=bg]{Ruby}
# The Greeter class
class Greeter

  # Capitalizes the given name.
  # @param name [String] the person's name to greet
  def initialize(name)
    @name = name.capitalize
  end


  # Greets the person.
  def salute
    puts "Hello #{@name}!"
  end


  # Anonymizes the person to greet.
  def anonymize!
    @name = 'Anon'
  end


  # @return [Boolean] whether the person's name is 'Anon'
  def anonymous?
    @name == 'Anon'
  end
end

greeter = Greeter.new('world')
greeter.salute # prints "Hello World!"

greeter.anonymize!
greeter.anonymous? #=> true
greeter.salute # prints "Hello Anon!"
	\end{minted}
	\caption{Example Ruby listing}
	\label{lst:ruby:example}
\end{listing}


% vim: ft=tex
\chapter{Scope}
This chapter outlines the general scope of this project, beginning with the
students' motivation to work on this bachelor thesis.

The \autoref{sec:scope:init} introduces mindclue GmbH and its Roadster
framework in its state prior to this bachelor thesis to provide the reader with
an understanding detailed enough to comprehend the contributions described in
\autoref{ch:approach}.

Finally, \autoref{sec:scope:goals} elucidates the goals from the original task description
(\autoref{ch:task-desc}), as well as in retrospect.

\section{Motivation}
\subsection{Backgrounds}
To better understand our motivation, it might help to understand our personal
backgrounds first.

\textbf{Patrik Wenger} did his apprenticeship in computer science at Swisscom
Schweiz AG, and stayed work as a full-time employee for five more years
afterwards. In programming he is most fluent in \gls{ruby} and \gls{c}. During
winter 2015/2016, he created \gls{cztop} in leisure time because
there was no suitable Ruby binding for \gls{zmq}/\gls{czmq} available and a side
project of his demanded it. Fascinated with event-driven programming and
software design patterns such as the \gls{actor-model} (e.g. the
Celluloid\footnote{A concurrency framework for Ruby based on the actor model,
\url{https://github.com/celluloid/celluloid}} library on Ruby,
and the Pony programming language\footnote{A young programming language completely based on actors,
\url{http://www.ponylang.org}}), distributed computing and high availability
have long been part of his core interests, especially in conjunction with the
brilliant \zmq library. Having a strong interest in information security and modern
cryptography\footnote{I.e. \gls{nacl} or \gls{libsodium} as used by
\gls{zmq}}, especially in this post-Snowden era, this bachelor thesis could not
be a better match.

\textbf{Manuel Schuler} did his apprenticeship in computer science at
Alcatel-Lucent AG.  Most of his projects involved network monitoring or
configuration automation. In programming he is most fluent in Node.js, Java and .NET. He
made several projects to keep his life simple. After working full-time for
several companies after his apprenticeship, he decided to start his own
business.  Always keen on learning new things and the fact that his work
involved similiar technologies, he did not hestiate to join this bachelor theis
at the first opportunity.

In essence, both students are thrilled to gain more experience in the following
fields and technologies:

\begin{itemize}
	\item Distributed computing
	\item High availability
	\item Information security
	\item \gls{actor-model}
	\item \gls{zmq}
	\item \gls{ruby}
\end{itemize}

\subsection{Opportunities}
Coming from different backgrounds and having different levels of experience in
each of the aforementioned technologies, we cannot wait to learn more about them and put
them to actual use. The fact that the product of this bachelor thesis is most
likely going to be used in the real world only adds to the excitement.

This bachelor thesis involves working with Ruby, the Actor Model, \zmq,
distributed computing with high availability, and state-of-the-art
cryptography. Furthermore, in case of successful completion of this thesis, the results will be used in real-world settings like the Ceneri
Base Tunnel. It is a huge opportunity for a solution completely based on free
and open-source software interacting with other industrial systems over open standards. The students, as well as the client, strongly believe
in customized solutions built on reusable, free open-source software.

In addition to that, we look at this bachelor thesis as an opportunity to
become more fluent in English, both written and spoken, as well as to improve
our skills in crafting scientific documents using {\LaTeX}.

Depending on how we perform together as a team, further collaboration might
result in the future, either between the students themselves, or between the
students and the client. Even if our paths will part, this project will
serve as a valuable reference for future job hunting.

\subsection{Open-source engagement}
Getting the chance to use \gls{cztop} and watch it perform definitely adds to
the motivation as well. Its software design has yet to be proven in more
productive settings.

Another personal goal is to create a reusable open-source library as a
byproduct. The intention is that the library makes certain \zmq-based
communication protocols readily available for other developers facing the same
problems.

\section{Initial situation}\label{sec:scope:init}

\subsection{mindclue GmbH}
The company mindclue GmbH, located in Ziegelbr\"ucke GL, provides its partner
REMTEC AG with complete \gls{SCADA}\footnote{SCADA software resides in level 2 of the enterprise levels (0--4) modeled by the \gls{isa95} standard, \url{https://en.wikipedia.org/wiki/Enterprise_control}} applications. These are then used to
supervise and control operation and safety equipment found in:
% ISA95 “levels”

\begin{itemize}
\item National freeways, e.g. emergency phones
\item Tunnels, e.g. lights, ventilation, and railway electricity
\item Water supply systems
\item Energy facilities
\item Other specialized fields
\end{itemize}

To build these customized applications, their in-house creation
Roadster is used.

\subsection{Roadster framework}
Roadster is a SCADA framework written in Ruby. It has been developed to
produce next-generation SCADA applications to replace legacy
solutions based on its predecessor found in numerous tunnel
facilities in Switzerland.

A Roadster installation combines the following responsibilities:

\begin{itemize}
	\item Interaction with subordinate field devices (monitoring \& controlling)
	\item Sophisticated alarm case management
	\item Persisting data (e.g. time series of sensor data, alarm log)
	\item A machine-to-machine interface to higher level systems (\emph{clients})
	\item A modern web UI customized for the particular installation to interact with operational and executive personnel
\end{itemize}

Among others, the field devices include various kinds of \glspl{PLC}\footnote{E.g.
the SIMATIC S7-1500 by Siemens AG,
\url{http://w3.siemens.com/mcms/programmable-logic-controller/en/advanced-controller/s7-1500/Pages/default.aspx}}
as well as emergency call systems\footnote{E.g. the NIS ComNode by Trans Data
Management AG,
\url{http://trans-data.com/en/k2-categories/item/149-niscomnode}}.
These are interacted with over numerous propietary and standardized protocols,
for which Roadster provides spezialized adapters.

The following two subsections elaborate more on Roadster and how it's used.
However, as far as this bachelor thesis goes, it's not important information
and thus can be skipped.

\subsubsection{System integration}\label{sec:scope:sys-integration}
This section briefly describes the big picture of Roadster's place within
typical production environments and its relationship with other systems.

In some deployments, Roadster has no higher-level client systems, but is
operated as a standalone supervision and control system. In that case, the only
clients would be its human users. This usually applies to solutions delivered
to counties as opposed to solutions operating on a national level.

In other deployments, there are higher level systems which act as clients of a Roadster instance, communicating over
protocols including \gls{SOAP} and \gls{opc-ua}. Their purpose is to
collect and aggregate supervisory data from larger regions. At the top of the
hiearchy are the \gls{FEDRO} (German: \gls{ASTRA}) which combine the information of all
subsystems to provide a nationwide overview.

\begin{figure}[]
	\includegraphics[width=\textwidth]{img/overall_system.pdf}
	\caption{Roadster's place within a typical system}
	\label{fig:roadster:overallsys}
\end{figure}

\autoref{fig:roadster:overallsys} illustrates the typical system. The acronyms
used, which are part of the \gls{FEDRO} terminology, do not have official
English translations; not even the department itself was able to help with
translations on request, so they were left unchanged. A brief description
follows, from the bottom up:

\begin{description}
	\item [ Field devices: ] \hfill\\
	Field devices are various kinds of \glspl{PLC} and other subordinate systems
	used to supervise and control industrial processes. Communication with them
	happens over protocols such as \gls{modbus-tcp}, \gls{iec-104}, and
	\gls{opc-ua}.

	\item [ \gls{AS}: ] \hfill\\
	This is Roadster's domain, or, of course, the domain of another product with similar
	functionality. An AS is responsible for one facility (e.g. emergency call system,
	lighting system, fire alarm system, video monitoring system, ventilation
	system, power supply system, train signaling system).

	\item [ \gls{AR}: ] \hfill\\
	The higher level system of the collection of all AS found in one larger
	facility such as a tunnel. With the results of this bachelor thesis, this
	\emph{could} be Roadster's domain as well.

	\item [ \gls{LR}: ] \hfill\\
	The higher level system of the collection of all AR found in a region such as
	Z\"urich. As with AR, this \emph{could} be Roadster's domain as well.

	\item [ \gls{LTA}: ] \hfill\\
	This collective term comprises both the levels of AR and LR. For
	simplicity and readability's sake, this will be referred to as
	\emph{client} for the remainder of this document.
\end{description}

\subsubsection{Typical hardware}
Roadster typically runs on entry-level rack server hardware powered by an
Intel\textregistered{} Xeon\textregistered{} processor, or industrial box PCs for smaller systems
commonly used for \gls{IoT} which are powered by more energy efficient processors
such as Intel\textregistered{} Core\textregistered{} and Intel\textregistered{}
Atom\texttrademark{}. The machines are usually equipped with 4 -- 6 GiB of main
memory and Gigabit Ethernet. For reliable systems without any moving parts, an
industrial grade \gls{SSD} or two (in a software \gls{RAID} level 1 setup) are used.


\subsection{Software architecture}
As mentioned earlier, Roadster is \gls{event-driven} and built on the Actor model, meaning it exhibits a
\gls{shared-nothing-arch}. Each Roadster node runs a number of Ruby processes
which communicate via \zmq sockets. The key here is communication:

\begin{quote}
``Don't communicate by sharing state; share state by communicating.''
\end{quote}

Running multiple, loosely coupled processes (actors) allows leveraging the full
potential of modern multi-core processors, while avoiding a whole class of
concurrency issues present in traditional concurrency models based on shared
memory and synchronization constructs such as semaphores.

Every Roadster node runs a group of actors:

\begin{description}
	\item [CORE:]\hfill\\
		The CORE actor is responsible to maintain aggregates of supervisory data
		(e.g. sensor data) and create alarm cases to inform
		about critical situations.

		During the application startup it is responsible to start the
		other actors and later on routes messages between them. It also
		plays a key role in keeping state in all actors synchronized. More
		about that in \autoref{sec:scope:csp}.

	\item [COMM:]\hfill\\
		COMM actors communicate with the external services accessible
		by a node. Each of them employs a specific communication
		protocol to act as an adapter to one of various field devices
		(usually called \emph{peers}), or as a server to an attached client
		system.\\

		The exact set of COMM actors running on a particular
		installation depends on a configuration file available to the
		installation.\\

		The webserver for the web UI is one of the COMM actors, denoted
		by COMM.WEBUI.

	\item [STORAGE:]\hfill\\
		This actor is used when information needs to be persisted, such
		as time series or event journals. It is the interface to a
		key-value store.

	\item [LOGGER:]\hfill\\
		This actor collects logging data and sends it to whatever
		target is configured, be it \gls{stdout}, a file, or a syslog server.

	\item [CONSOLE:]\hfill\\
		This actor is not fully implemented yet. It's supposed to
		be a textbased user interface for maintenance and trouble shooting.
\end{description}

\autoref{fig:roadster:arch} illustrates Roadster's architecture. The red boxes
with arrows attached to the actors represend the \zmq sockets which are used
for inter-actor communication. More information about \zmq sockets in
\autoref{sec:scope:zmq}.

\begin{figure}[]
	\includegraphics[trim=4cm 2cm 3.5cm 2.8cm, clip=true, width=\textwidth]{img/roadster_arch.pdf}
	\source{mindclue GmbH}
	\caption{Roadster's software architecture}
	\label{fig:roadster:arch}
\end{figure}

Each actor of Roadster employs three abstraction layers
as illustrated in \autoref{fig:roadster:layers}.  The following
list briefly explains the layers from top (most abstracted) to bottom:

\begin{description}
	\item [Engine layer:]\hfill\\
		Here is the business logic of Roadster, e.g. the \gls{DIM},
		user authentication, adapters for external devices, the web
		\gls{UI}, etc.

	\item [Messaging layer:]\hfill\\
		The \gls{RMP} reside here and implement essential protocols used
		for logging, state synchronization, commands, application controlling,
		and storage. They're explained later in \autoref{sec:rmp}.

	\item [Reactor layer:]\hfill\\
		This layer forms the base, which is where the \zmq sockets and
		\glspl{websocket} are utilized. It is powered by an
		event-loop\footnote{EventMachine is used as a high-performance event-loop to
		manage large numbers of sockets and timers,
		\url{https://github.com/eventmachine/eventmachine}}. Sockets used by COMM actors
		to communicate with various field devices are also integrated into this
		event-loop.

\end{description}

Following the reactor design pattern, the event-loop informs Roadster about
data received on sockets (i.e. \zmq messages or other protocol data), which is
then read and processed using the upper two layers in the next event
tick. This approach eliminates most concurrency issues because within an actor,
nothing actually happens in parallel. An actor merely reacts on messages and
timers sequentially.

\begin{figure}[]
	\includegraphics[trim=1.95cm 2.5cm 1.65cm 2.8cm, clip=true, width=\textwidth]{img/roadster_layering.pdf}
	\source{mindclue GmbH}
	\caption{Roadster's communication layers}
	\label{fig:roadster:layers}
\end{figure}


\subsection{Domain Informaton Model}
The \acrfull{DIM} is a tree data structure that is maintained by every actor
taking part the in business logic of a Roadster node, which are all but the LOG
and STORAGE actors.  The CORE actor and every COMM actor build the DIM from the
installation's configuration files\footnote{The configuration is written in a
\gls{DSL} defined as part of the model classes of Roadster. Ruby makes it easy
to define concise, expressive DSLs.} when starting up. The resulting collection
of domain model instances commonly counts up to a few hundred objects,
depending on the attached field devices and configuration items.


\autoref{lst:roadster:dim-tree} shows the hierarchical structure of a
DIM comprising about 60 objects (shortened to increase readability), including:
\begin{itemize}
	\item Access control lists (users, roles, authentication and authorization rules)
	\item Navigation menus and pages for the web UI
	\item An adapter for a robot field device (utilized by a COMM actor) and its peer, the robot
	\item \emph{Variables} and \emph{aggregates} based on the robot's current position
	\item A guard for the robot
\end{itemize}

The purpose of guards (instances of \sh{Roadster::Domain::Model::Guard}) is to
observe certain variables and aggregates for changes. In case a sensor value
exceeds the range declared as part of the guard, an alarm case is created,
which requires human interaction.


\autoref{fig:roadster:meta-model} illustrates the class diagram of
all the classes instanciated to constitute the DIM. They are all defined within the
Ruby module namespace \sh{Roadster::Domain::Model}.
Most of these these objects are static, but some are dynamic:
\begin{itemize}
	\item The current value of a \emph{variable} or an \emph{aggregate}, stored in an instance of \\
		\sh{Roadster::Domain::Model::DataItem}
	\item Current sessions of web UI users (\sh{Roadster::Domain::Model::Session})
	\item Pending alarm cases (\sh{Roadster::Domain::Model::Case})
\end{itemize}
Modifications to the dynamic objects are replicated across all DIM-aware
actors, which is performed by the \gls{CSP} as explained in~\autoref{sec:scope:csp}.


\begin{listing}
	\begin{minted}[bgcolor=bg]{Text}
root (Root)
root.navigation (Entity)
root.navigation.home (Page)
root.navigation.home.process (Menu)
# [...] 6 more Menu and Page objects clipped
root.var_map (Entity)
root.access_control (Acl)
root.access_control.users (Entity)
root.access_control.users.admin (User)
# [...] user credentials (Parameters) clipped
root.access_control.users.guest (User)
# [...] user credentials (Parameters) clipped
root.access_control.roles (Entity)
root.access_control.roles.superuser (Role)
# [...] 5 Role and Parameter objects clipped
root.access_control.sessions (Entity)
root.version (Variable)
root.version.datum (DataItem) = 0
root.adapters (Entity)
root.adapters.simu (Adapter)
root.adapters.simu.version (Variable)
root.adapters.simu.version.datum (DataItem) = 0
root.adapters.simu.robot_controller (Peer)
root.adapters.simu.robot_controller.conn_state (Variable)
root.adapters.simu.robot_controller.conn_state.datum (DataItem) = "CONNECTED"
root.objects (Entity)
root.objects.robot (Robot)
root.objects.robot.control_mode (Parameter)
root.objects.robot.control_mode.datum (DataItem) = "AUTO"
root.objects.robot.position_x (Variable)
root.objects.robot.position_x.datum (DataItem) = 30.0
root.objects.robot.position_y (Variable)
root.objects.robot.position_y.datum (DataItem) = -9.0
root.objects.robot.polar_position_radius (Aggregate)
root.objects.robot.polar_position_radius.datum (DataItem) = 31.32092
root.objects.robot.polar_position_radius.alarm_guard (Guard)
root.objects.robot.polar_position_angle (Aggregate)
root.objects.robot.polar_position_angle.datum (DataItem) = -16.69924
  \end{minted}
  \caption{An example list of objects in a DIM}
  \label{lst:roadster:dim-tree}
\end{listing}

\begin{figure}[]
	\includegraphics[trim=1.5cm 1cm 1cm 1cm, clip=true, width=0.9\textwidth]{img/meta_model.pdf}
	\source{mindclue GmbH}
	\caption{Class diagram for Roadster's domain model used in the DIM}
	\label{fig:roadster:meta-model}
\end{figure}

\subsection{Roadster Messaging Protocols}\label{sec:rmp}
The \gls{RMP} are a collection of protocols implemented and used by Roadster
internally. They reside in the messaging communication layer, and include:

\begin{description}
	\item [\gls{ACP}:]\hfill\\
		Used to control the application state, e.g. shutdown.
	\item [\gls{CMP}:]\hfill\\
		Used to accept pending \glspl{case}.
	\item [\gls{CSP}:]\hfill\\
		Used to synchronize state between the DIM-aware actors.
	\item [\gls{LOG}:]\hfill\\
		Used for system logging. Log messages from all actors are sent
		to the LOGGER actor via this protocol.
	\item [\gls{PCP}:]\hfill\\
		Used for command execution via COMM peers with feedback.
	\item [\gls{PDP}:]\hfill\\
		Used when data needs to be persisted via the STORAGE actor.
	\item [\gls{SMP}:]\hfill\\
		Used to suppress the generation of certain alarm cases, e.g.
		when a sensor is defect and repeatedly reports critical sensor data.
\end{description}

Each protocol offers \glspl{API} endpoints for the different roles of
participants, a subset of which are then registered in every actor to fulfill
its responsibilities.

The next section provides more detail about the different
delivery guarantees available for message passing.

\subsubsection{Reliability modes}
Following the \gls{actor-model}, messages are passed between actors
asynchronously. This happens in one of two reliability modes:

\begin{description}
\item [Fire \& Forget:]\hfill\\
No guaranteed delivery. This does not mean there are no other mechanisms in
place to ensure reliability, e.g. employed by a protocol itself.\footnote{such
as what \gls{TCP} does by sequencing the segments of a stream}

\item [Dialog:]\hfill\\
An immediate answer is expected, e.g. when creating a user
session. Any protocol can make use of this primitive. Source code that
initiates a dialog looks like a synchronous call, even though
it is handled asynchronously by Roadster's message passing
infrastructure.\footnote{This is done by wrapping the affected
code in a Ruby \sh{Fiber}, which is similar to a thread but
allows for cooperative scheduling as opposed to preemtive.}
\end{description}


\subsection{Directory structure}
\autoref{lst:roadster:directory-structure} gives an annotated overview of Roadster's directory structure.

\begin{listing}
	\begin{minted}[bgcolor=bg]{Text}
roadster
|-- benchmarks                # Benchmark scripts
|-- bin                       # The `roadster` utility
|-- doc                       # Generated API documentation
|-- lib                       # Framework implementation
|   `-- roadster
|       |-- actors            # Reactor layer: Base, Core, Comm, ...
|       |   `-- em-zmq        # Adapter for ZMQ library
|       |-- adapters          # IEC-104, Modbus TCP, OPC, ...
|       |-- engines           # Engine layer: Base, Core, Comm, ...
|       |-- messaging         # Messaging layer: Dialog
|       |   |-- handlers      # Base, Core, Comm, dispatching, ...
|       |   |-- messages
|       |   `-- protocols     # ACP, CMP, CSP, LOG, PCP, PDP, SMP, ...
|       |-- misc              # Conf, Exceptions, Ruby core extensions
|       `-- tools             # CLI
|-- roadster-webui-core
`-- spec                      # Unit and integration tests of the above
    |-- actors
    |   |-- em-zmq
    |-- adapters
    |-- engines
    |-- messaging
    |   |-- handlers
    |   |-- messages
    |   `-- protocols
    `-- support               # Fixtures
        `-- domains
  \end{minted}
  \caption{Directory structure of the Roadster framework}
  \label{lst:roadster:directory-structure}
\end{listing}




\subsection{\zmq}\label{sec:scope:zmq}
As mentioned earlier, Roadster uses \zmq to carry messages between its actors.
To understand the rest of this document, it is
helpful to understand the basics of \zmq. This is a brief introduction to
\zmq for the unfamiliar reader. What follows is a quote from the \gls{zguide}
which does a fairly good job at describing \zmq in a hundred words:

\begin{quote}
``ZeroMQ (also known as \zmq, 0MQ, or zmq) looks like an embeddable networking
library but acts like a concurrency framework. It gives you sockets that carry
atomic messages across various transports like in-process, inter-process, TCP,
and multicast. You can connect sockets N-to-N with patterns like fan-out,
pub-sub, task distribution, and request-reply. It is fast enough to be the
fabric for clustered products. Its asynchronous I/O model gives you scalable
multicore applications, built as asynchronous message-processing tasks. It has
a score of language APIs and runs on most operating systems.  ZeroMQ is from
iMatix and is LGPLv3 open source.''
\end{quote}

\subsubsection{Messages}
A message is just an array of one or more parts (\emph{frames}). From the point
of view of \zmq, each message frame is simply a
chunk of opaque data. Object serialization is outside the scope of \zmq and can be done using
JSON, XML, ASN.1, MessagePack, Google's Protocol Buffers, Apache Thrift, or a
programming language's own implementation of object marshalling, which is what
Roadster does for inter-actor communication. JSON is used between the WEBUI
actor and the actual web application running in the HTTP user agent.

\subsubsection{Socket types}
A number of socket types are provided, of which the most important pairs are:

\begin{description}
	\item [ROUTER and DEALER sockets] \hfil\\
	These sockets implement an asynchronous request-reply communication
	pattern. When sending a message via a DEALER socket, it is sent
	to any of the connected ROUTER sockets (usually just one). When
	reading a message from a ROUTER socket, an additional frame is
	prepended to identify the DEALER socket which sent the message.
	In \zmq terminology, this kind of frame is called an
	\emph{envelope}.
	This identity is a short string (up to 255 bytes) either randomly chosen by \zmq
	or set by the application as a socket option. It does NOT
	resemble an IP address or anything dependent on the chosen
	transport.

	When sending a message via a ROUTER socket, the same
	process happens in reverse: The first frame is taken away and
	used to determine the receiving DEALER socket, which will
	receive all but the first frame.

	Roadster's CORE actor uses a ROUTER socket to send or relay
	messages to specific actors.


	\item [PUSH and PULL sockets] \hfil\\
	These sockets implement an asynchronous pipeline communication pattern
	providing fan-out and fan-in techniques. PUSH sockets can only be
	written to whereas PULL sockets can only be read from.


	Within Roadster, this socket pair is used to perform efficient
	replication of modifications to the DIM. More about CSP
	in~\autoref{sec:scope:csp}.


	\item [PUB and SUB sockets] \hfil\\
	A publish-subscribe communication pattern is provided by these two
	sockets. Message distribution based on subscription to one or
	more topics is possible (simply by matching the beginning of the first
	frame).

	At this point, Roadster actors simply
	subscribe to all messages sent via a the CORE's PUB socket.

\end{description}

Any socket can be either bound and/or connected to any number of endpoints. A few examples:

\begin{itemize}
	\item A simple message passing application following the client-server
		model, can bind a ROUTER socket on the server and connect each
		client's DEALER socket to it. After that, any one of the server
		or the clients can asynchronously send a message.

	\item A SUB socket can connect to multiple PUB sockets to receive
		published messages from all of them

	\item A parallel pipeline can be built using a PUSH socket that fans
		out tasks, which are then read by one of possibly many worker
		instances via its PULL socket.
\end{itemize}

In any of the above cases, the number of connected remote sockets does not
matter. To the application, it's still a single socket, which just happens to
communicate with many other sockets.

\subsubsection{Transport}
The transport technique and transport-specific connection handling is completely
abstracted away by \zmq. There's no possibility that such implementation
details leak into an application which would result in an unwanted increase of
external coupling.

This also means that it does not matter whether an application communicates with
other \zmq sockets within the same process, on the same machine, or on a remote
machine. Common transports are \gls{unix-domain-socket}, \gls{TCP},
\gls{TIPC}, as well as \gls{PGM} based multicast.

\subsubsection{Software packages}
Roadster uses a Ruby gem (software package) called
\emph{ffi-rzmq}\footnote{\url{https://github.com/chuckremes/ffi-rzmq}} to
interface with \zmq. It is unmaintained and does not support newer features of
\zmq (namely secure communication).


For a more details about \zmq, including the \gls{czmq} abstraction layer and
the \gls{cztop} langauge binding for Ruby, see \autoref{ch:zmq}.


\subsection{Existing CSP in a nutshell}\label{sec:scope:csp}
The existing \acrfull{CSP} is closely related to the \gls{clone-pattern} from
the \gls{zguide}. Its goal is to keep the \gls{DIM} (\emph{state}) synchronized
across the set of DIM-aware actors.  It follows a server-client architecture as
opposed to a fully decentralized architecture, which allows for greatly reduced
complexity.

The server part acts as the authorative source for all modifications to the
DIM. This part is performed by the CORE actor and makes use of its ROUTER, PULL, and
PUB socket.

The client part is performed by all COMM actors and utilizes the respective
actor's DEALER, PUSH, and SUB socket.

The protocol consists of three distinct messages flows:

\begin{description}
	\item [Snapshots:]
		Requesting and receiving the complete, current snapshot of the
		state. This is performed via the ROUTER-DEALER pair of sockets
		during the startup of each COMM actor. The CORE actor responds
		with the complete set of dynamic DIM objects.

		In the best case, the snapshot only needs to be requested once
		by each COMM actor. However, as there is no guaranteed message
		delivery, it could happen that an incremental update (issued by
		another COMM actor or the CORE itself) gets lost. With the next
		successfully delivered update, this situation is recognized by
		the receiving COMM actor based on the strictly monotonically
		increasing sequence number (\emph{version}) associated with
		each update. The discrepancy is resolved by simply requesting a
		new snapshot, after which the state has been resynchronized.

	\item [Upstream updates:]
		Updates can originate from a COMM actor (i.e. a new sensor data
		from a field device read by the actor's adapter) and are sent to the
		CORE actor via the PUSH-PULL socket pair.  This works by
		marking the updated objects as ``dirty'' so they can be sweeped
		and marked as ``idle'' when sending the next state update.

		The CORE itself can modify the state as well, i.e. re-evaluate
		aggregates based on changed sensor data, or create/close a
		case. Of course, this change does not need to be sent through a
		PUSH socket first, as it happens directly inside the CORE
		actor. However, the subsequent behavior is the same, namely
		that the new modifications need to be replicated to the other
		COMM actors.

	\item [Downstream updates:]
		After being applied to the CORE's state,
		modifications get a sequence number and are replicated to all
		COMM actors. This happens via the CORE's PUB socket, so all
		COMM actors' SUB sockets receive the update.
\end{description}

By making all updates go through the CORE actor, a single sequence is enforced,
which is crucial to keep the state consistent across all DIM-aware actors.

To avoid risking a gap between requesting the current snapshot and subscribing
to updates, a COMM actor actually subscribes to the updates first, then gets the
snapshot, and then starts reading the updates from the SUB socket (which has been
queueing updates in the meantime, if any). Updates that are older or the same
age as the received snapshot are skipped, and only successive updates are
applied (tested by comparing the sequence numbers).

Because message loss via the third message flow (PUB-SUB) is unlikely but
theoretically possible, the participating actors check for gaps in the sequence number of
each downstream update. If a gap is detected, the current state is discarded and a
complete resynchronization happens. This is brutal, but is very simple and thus
robust; there is no complexity that would leave room for nasty corner cases.

Updates from COMM actors have their own version number, one for each adapter.
This ensures that the CORE is able to do the very same checks to notice
gaps in the sequence of updates from any COMM actor.

% TODO: mention subtrees somewhere else, maybe in the approach
%A feature described in the \gls{zguide}, but not implemented in Roadster as of
%this writing, are subtrees.
%Keys can be treated hierarchically (e.g. \sh{topic.subtopic.key}) and thus, a
%client can optionally subscribe to only a particular subtree. This is useful
%when the number of client grows and not all of the state needs to be on every
%client. In that case, the topic of interest is sent by the client along with
%the ICANHAZ message.

% TODO: add illustration

\subsection{Persistence}\label{sec:scope:persistence}
Certain data on a Roadster node is persisted, which is done by the STORAGE
actor. It uses the
\gls{tc} library as a
serverless key-value store, similar to what the more commonly known
SQLite\footnote{\url{https://www.sqlite.org}} library provides for relational data.

A rough estimation from mindclue GmbH states that, according to data retention
policies, the long-term accumulated data of a Roadster node can be up to $\sim$300 MiB.

There are three kinds of persisted data which are stored in different database files:

\begin{description}
	\item [ Event journal: ] \hfill\\
		The event journal is a history of all \glspl{case}, including the ones
		that have been confirmed and thus removed from the DIM. It
		resides in a \gls{tc} \emph{table} database, where the key is a
		\acrshort{UUID}, one of the columns (attributes) is the timestamp of
		the case, and another one is the actual, serialized \sh{Case}
		object. Objects in this database can be modified, e.g. when a
		pending case is confimed.

	\item [ Time series: ] \hfill\\
		This is a pure key-value store that stores samples of sensor
		data from field devices. Only the most recent value of these
		are actually kept in the DIM. One file per series is used. The
		timestamp is part of the key.

	\item [ Parameters: ] \hfill\\
		Parameters are the third and last kind of persisted data in
		Roadster. Parameters are typically changed by a user. Each
		parameter is backed by a default value from the configuration,
		which is used as long as there is no actual corresponding value
		set or read from the database. Examples include credentials for
		users of the web UI.
\end{description}


\subsection{Tests}
Roadster comes with a few unit tests, and a number of integration tests,
written in RSpec, a well known \gls{BDD} framework for Ruby projects. Test coverage is
about 67\% overall, some files being better covered, some files less. There are
no system tests.

\section{Goals}\label{sec:scope:goals}
To summarize the mandatory goals from the task description in \autoref{ch:task-desc}:

\begin{enumerate}
	\item Getting familiar with Roadster
	\item Extending the communication protocols to support federation of
		multiple nodes
	\item Extending the communication protocols to allow high availability
		clusters of two peer nodes
\end{enumerate}

The optional goals are:

\begin{enumerate}
	\item Encryption of the communication
	\item Providing of the highly available \gls{opc-ua} server interface
\end{enumerate}


\subsection{In retrospect}
During the Elaboration phase when gathering requirements (\autoref{ch:reqs})
and implementing prototypes (\autoref{sec:approach:prototypes} in \autoref{ch:approach}), the
aforementioned goals developed into the following more concrete tasks
(documentation tasks excluded):

\begin{enumerate}
	\item Get familiar with Roadster, especially its multilayer software architecture
	\item Port Roadster onto a new \zmq binding
	\item Provide the ability to declare a federation of Roadster nodes
	\item Inter-node messaging
	\item Heartbeating between a supernode and its subnodes
	\item Message routing to arbitrary nodes in the federation
	\item Creation of alarm cases when a neighboring node is unresponsive
	\item Provide the ability to declare a HA node's primary and backup endpoints
	\item Correctly start engines on two HA peers (going either active or passive)
	\item Selection of the correct supernode (if it's a HA node), which could trigger a failover
	\item Correctly recognize the situation where the passive HA peer needs
		to promote itself to the new active peer
	\item Correctly activate engines on the passive node during failover
	\item Creation of an alarm case when one of the HA peers becomes unresponsive
	\item Reliably replicate persisted data to the supernode
	\item Secure the communication on inter-node sockets using authenticated encryption
	\item Restrict secure communication to selected neighbors based on their public keys
	\item Perform message authentication to ensure end-to-end security (not
		only hop-by-hop which is encompassed by the previous goal)
	\item Provide the ability to declare a node's public key
	\item Provide an utility to generate a new private key
	\item Provide infrastructure so client system interfaces can also trigger a failover
\end{enumerate}


\subsection{Additional goals}\label{sec:scope:add-goals}
The goals described in this section are neither explicitly, nor implicitly part
of the original task description, but developed out of personal interest in the
technical matter.

\subsubsection{Security concerns of SCADA applications}
Secure inter-node communication within a Roadster federation is important to
mitigate common security concerns with SCADA systems which are becoming more
and more open due to standardization. To quote \cite[Security issues]{wp:scada}
wikipedia:

\begin{quote}
``In particular, security researchers are concerned about:
	\begin{itemize}
		\item the lack of concern about security and authentication in
			the design, deployment and operation of some existing
			SCADA networks
		\item the belief that SCADA systems have the benefit of
			security through obscurity through the use of
			specialized protocols and proprietary interfaces
		\item the belief that SCADA networks are secure because they
			are physically secured
		\item the belief that SCADA networks are secure because they
			are disconnected from the Internet.''
	\end{itemize}
\end{quote}

The goal is to provide a framework that makes it trivial to provide reasonably
secure SCADA applications which are assumed to be running on insecure networks.

\subsubsection{Fallacies of distributed computing}
At this place, it is worth noting the common fallacies encountered in
distributed computing, as explained on \cite{dcomp:fallacies}.

\begin{description}
	\item [The network is reliable.] \hfill\\
		Network outages need to be embraced and handled with
		appropriate error-handling to avoid stalls, permanent
		resource consumption, and the need for manual restarts.

	\item [Latency is zero.] \hfill\\
		The ignorance of network latency and packet loss can lead to
		wasted bandwidth when traffic is unbounded. Depending on the
		application, assuming instant packet delivery could lead to
		timing issues like hanging user interfaces, especially if some
		nodes are not in the same LAN.

	\item [Bandwidth is infinite.] \hfill\\
		Trying to send too much information can lead to bottlenecks,
		slow applications, and simply wasted bandwidth.

	\item [The network is secure.] \hfill\\
		This touches the subject mentioned before. SCADA applications
		often run in isolated networks, at least not directly reachable
		from the internet, but that does not make them secure. Threats
		from within are real and thus one cannot assume a network is
		secure.

	\item [Topology does not change.] \hfill\\
		Topology changes constantly. Services are added and removed. An
		application must not depend on specific endpoints or routes.

	\item [There is one administrator.] \hfill\\
		The overhead introduced by having coordinate multiple
		administrators is not negligible and can constrain one's options
		when e.g. having to upgrade software.

	\item [Transport cost is zero.] \hfill\\
		Getting application entities onto the wire costs, i.e.
		computational resources and in some cases money. Data needs to
		be serialized, which takes CPU cycles and adds to the latency.
		Network equipment to e.g. mitigate the other fallacies -- e.g.
		to make it (more) reliable and more secure -- can cost more
		money than is planned for in the project budget.

	\item [The network is homogeneous.] \hfill\\
		Virtually all networks include devices running different
		operating systems, in different versions, on different
		hardware. On an application level, different subsystems are
		implemented in different programming languages or speak
		different protocols.

\end{description}

The goal is to conduct this thesis with the aforementioned fallacies in mind and embrace
them in the resulting contributions.

% vim: ft=tex
\chapter{Requirements}
The requirements gathered during the first meeting with the client are
explained in this chapter. These are more concrete than the ones listed in the
Task Description in \autoref{ch:task-desc}.

First of all, these are the priorities from the client's point of view in
descending order:

\begin{enumerate}
\item Clustering
\item Single-level \gls{HA}
\item Multi-level \gls{HA}
\item Persistence synchronization
\item Security (optional)
\item OPC UA \gls{HA} (optional)
\end{enumerate}

The following sections explain the requirements in greater detail.

\section{Functional}
This section elucidates the functional requirements, as opposed to the
non-functional requirements.

\subsection{Cluster}
This requirement is to allow running Roadster on multiple nodes in a
hierarchical topology.

Extending the \gls{CSP} to keep the \gls{DIM} in sync across all nodes is a central
part of the clustering functionality. This means replicating the
items within the DIM that are marked "dirty" (updated, but not synchronized yet) onto all other actors on all other nodes.

According to
the Data Model class diagram, these should only be instances of one of three
classes (marked yellow in the diagram):
\begin{itemize}
	\item \rb{DataItem}
	\item \rb{Session}
	\item \rb{Case}
\end{itemize}

In addition to that, there needs to be some kind of message routing so a user
of one node's web UI can send a command to another node where it will be
executed. An example for this is a forced value in the DIM to ignore the
actually measured value reported by a device in case the device is known to be
wrong.

It's important that every node subtree can live on autonomously even if the
link to its supernode or the supernode itself fails.

The above requirements imply that changes to the DIM can only be done by the
respective node. In other words, a node can only change its own values. It
cannot change values of supernodes, nor is it allowed to change values of
subnodes directly. This is to ensure that each node is its own source of truth
for all other nodes in the setup.


\subsubsection{Typical Usecases}
The typical use cases include:

\begin{description}
	\item [ Single level, single node ]
		This is the legacy setup and is what Roadster is already able to do.

	\item [ Single level \gls{HA} ]
		This is when there are exactly two nodes, both of them
		connected to the same PLC. There are two nodes for redundancy.

	\item [ Multi level, \gls{HA} at root only ]
		There can be multiple levels in the hierarchy, such as two or
		three (anything else is considered exotic), and there's a HA
		cluster at the root.
\end{description}

% TODO: illustrate using diagrams

The exotic cases which can be ignored, if need be, include:

\begin{itemize}
	\item Multi level, HA at bottom
	\item Multi level, HA in the middle
\end{itemize}


\subsection{Single level HA}
This is where each of the two nodes forming a HA unit is directly connected to a number of
subsystems such as \glspl{PLC}, forming two redundant network paths to each
subsystem. Both nodes are able to interact with the subsystems to perform
operation tasks (e.g. reading sensor data, writing down configurations), but
only one of them (the active one) must do so.

The two nodes must automatically find consensus on which one is currently active. The
passive one must automatically take over in case the active one is confirmed to
be dead. Certain memory ranges can be used by the nodes to help find consensus
without interfering with normal operation of the subsystem. The kinds of
failures that need to be handled include:
\begin{itemize}
	\item Hardware/software failure on the primary node
	\item Failure of one of the redundant networking paths connecting the subsystem to the two nodes
\end{itemize}


\subsection{Multi level HA}
This is where a node pair is the parent of one or more subnodes.

The types of failures that need to be handled include:
\begin{itemize}
	\item Software failure on the primary node, like an application or OS crash
	\item Hardware failure on the primary node, like a defect power supply
	\item Failure of the network link connecting a node to the cluster
\end{itemize}

All three failure types listed above can collectively be called \emph{crash},
as their effects are the same from the point of view of the cluster.

\subsection{Persistence synchronization}
This is about the synchronization of persisted data, which is currently stored
in TokyoCabinet databases on a Roadster node. With the clustering, this is
still true: Every node will have its own key-value store. Updates for persisted
data must only flow from south to north (towards the root node), so the root
node can collect and maintain a replication of the persisted data of all
subnodes, recursively.

It's important that every node and its subnodes form an autonomous subtree. So
in case the link to its supernode fails, it has to continue working. As soon as
the link is repaired, synchronization of the delta (the newly added data) can
be initiatd. Sending just the delta should be possible since keys in the
database contain timestamps.

This is not the same as \gls{CHP} (\gls{DIM} synchronization), as the DIM is shared across
all nodes and is a relatively small data structure. The TokyoCabinet databases
can possibly contain large amounts of data (in the hundreds of megabytes) and
are shared only towards the root node (thus "bubbling up").

100\% consistency is not an absolute requirement for persistence synchronization.
However, it is mandatory that updates make it to the root node within 30 seconds.

\subsection{OPC UA HA}
\emph{This is optional.}

A given \gls{HA} pair needs to provide an OPC UA Server Redundancy interface,
as described in \cite[6.4.2.4 Non-transparent Redundancy,
p.~96]{opc-ua:behavior:server-redundancy}.

% ---------------------------------------------------------------------------
\section{Use Cases}
The use cases are briefly described here. These should simply describe common
scenarios based on the requirements above. They can later be turned into
concrete testing scenarios.

\subsection{UC01: Cluster}
A root node R has two subnodes A and B. A user connected to the root node's web
UI wants to suppress a case repeatedly generated on subnode A. The DIM shall be
kept in sync across all nodes.

\subsection{UC02: Hardware failure at top}
A subnode A is connected to a root-level HA pair (nodes R1 and R2). The active
HA peer (R1) crashes. R2 is supposed to take over as soon as subnode A fails
over to R2.

\subsection{UC03: Hardware failure at bottom}
A HA pair is connected to a field device. The active nodes crashes. The passive
one has to notice and take over to ensure continued monitoring of and access to
the field device.

\subsection{UC04: Persistence synchronization}
A root node R has two subnodes A and B. Persisted data on A has to bubble up to
root, even after temporary failure of the link between them. Root eventually
contains the union of all subnodes' persisted data.

\subsection{UC05: OPC-UA HA}
A HA pair provides an OPC UA interface. On failover, the superordinate system
shall continue to interact with the leftover HA peer.

% ---------------------------------------------------------------------------
\section{Non-Functional Requirements}
The following subsections illustrate the non-functional requirements.

\subsection{Simplicity}
The two reoccuring patterns that surfaced during the requirements gathering
meeting were:

\begin{enumerate}
\item \gls{KISS} principle. Simplicity is favored, as experience shows that
	simpler systems are more stable, so complexity should be avoided if not
	absolutely necessary.

\item No premature optimization since it's the root of all evil.\footnote{Quote
	by Donald Knuth: ``Premature optimization is the root of all evil.''}
\end{enumerate}


\subsection{Testing}
Regarding testing, the following requirements exist:
\begin{itemize}
	\item the student's contributions are verified with at least unit tests
	\item use cases shall be integration tested in a close-to-reality setup
\end{itemize}

\subsection{Security}
\emph{This is optional. Also, this requirement has the lowest priority not
because it's insignificant, but because it's easy to enable transport level
security on ZMQ sockets.}

The communication between a given set of Roadster nodes must be secured using
encryption.
% TODO: wait for Andy's answer on #9
% TODO common concern about SCADA systems lacking security\\


\subsection{Coding Guidelines}
The coding guidelines desired by the client are basically the ones written down
in the popular Ruby style guide \cite{rb:style-guide}, with the following
differences or special remarks:

\begin{itemize}
	\item method calls: only use parenthesis when needed, even with arguments (as opposed to \footnote{\url{https://github.com/bbatsov/ruby-style-guide\#method-invocation-parens}})
	\item 2 blank lines before method definition (slightly extending \footnote{\url{https://github.com/bbatsov/ruby-style-guide\#empty-lines-between-methods}})
	\item YARD API doc, 1 blank comment line before param documentation, one blank comment line before code (ignoring \footnote{\url{https://github.com/bbatsov/ruby-style-guide\#rdoc-conventions}})
	\item Ruby 1.9 symbol keys are wanted (e.g. \rb{foo: "bar", baz: 42} instead of \rb{:foo => "bar", :baz => 42}, just like \footnote{\url{https://github.com/bbatsov/ruby-style-guide\#hash-literals}})
	\item align multiple assignments so there's a column of equal signs
\end{itemize}

% vim: ft=tex
\chapter{Approach}\label{ch:approach}
This chapter describes the approach taken by the students to fulfill the
requirements. An introductory overview of all changes in retrospect is given in
\autoref{sec:approach:overview}.

Prototypes developed as proof-of-concepts are explained in
\autoref{sec:approach:prototypes}.

The \autoref{sec:approach:testing} shows the comprehensive testing
infrastructure set up to test all contributions.

The exact approach taken to develop every feature can be found in
\autoref{sec:approach:federation} and the following sections.


\section{Overview}\label{sec:approach:overview}
Numerous changes throughout Roadster's three architectural layers have been
performed. This section summarizes them quickly, before the rest of the chapter
goes into detailed descriptions regarding testing techniques, porting Roadster
to a new \zmq binding, developing the prototypes, and how each required feature
was designed and implemented exactly.

With the exception of the tests, almost all of these changes took place within
Roadster's \sh{lib} directory.

\subsection{Reactor layer}
The following actor classes have been added to allow a hierarchical inter-node
communication with a supernode, subnodes, and a HA peer node:
\begin{description}
	\item [\sh{Roadster::Actors::Upstream}] Registers sockets to communicate with a supernode.
	\item [\sh{Roadster::Actors::Downstream}] Registers sockets to communicate with subnodes.
	\item [\sh{Roadster::Actors::BStar}] Registers sockets to communicate with the \gls{HA} peer.
\end{description}
The above mentioned inter-node sockets are registered in addition to the sockets of a normal COMM actor.
The CORE actor \sh{Roadster::Actors::Core} has been adapted to start the
appropriate set of actors, depending on the actual federation topology
configured.

The module \sh{Roadster::Actors::Codecs}, which performs the serialization and
deserialization of messages when they're sent/received on a socket, has been
moved down from the messaging layer and extended with message authentication.
The new class \sh{Roadster::Actors::MessageAuthenticator} performs the actual
process of generating and verifiying cryptographic signatures.

\subsection{Messaging layer}
The following new protocols have been added:
\begin{description}
	\item [\sh{Roadster::Messaging::FCP}] The Federation Communication
		Protocol. The initial purpose was prototyping inter-node
		message passing. Eventually it is used for ping/pong
		heartbeating, as well as routing messages through nodes to
		arbitrary actors.

	\item [\sh{Roadster::Messaging::BSP}] The Binary Star Protocol. High
		availability mechanisms use it to perform heartbeating and
		communicate the current state within two HA peer nodes.

\end{description}

Analogous to the reactor layer, the following handler classes have been added:
\begin{itemize}
	\item \sh{Roadster::Messaging::Handlers::Upstream}
	\item \sh{Roadster::Messaging::Handlers::Downstream}
	\item \sh{Roadster::Messaging::Handlers::BStar}
\end{itemize}
The exact set of protocols used and supported by each actor are enabled in the
definitions of these handler classes.

Minor changes to two further classes of the messaging layer were necessary to
allow specifying a destination node (in addition to the actor name) as the
recipient of a message, and enable message routing using the new Federation
Communication Protocol.


\subsection{Engine layer}
The following engine classes have been added:
\begin{description}
	\item [\sh{Roadster::Engines::Upstream}]
		Implements mechanisms such as answering pings from the
		supernode with pongs, sending DIM updates to the supernode, and
		providing the lower layers with cryptographic keys to enable
		secure communications.

		If the supernode is configured to be a \gls{HA} cluster,
		this engine helps initiating a failover when necessary.

	\item [\sh{Roadster::Engines::Downstream}]
		Periodically pings subnodes, publishes DIM updates to all
		direct subnodes, and provides cryptographic keys to secure
		communication with subnodes.


	\item [\sh{Roadster::Engines::BStar}]
		The exact high availability mechanisms are implemented here.
		Just like the Downstream and Upstream engines of a supernode
		and its subnodes, the two instances of this engine continuously
		stay in contact to determine which HA peer is active, which one
		stays passive, and recognize the situation where one of them
		becomes unresponsive.
\end{description}

Common CSP functionality has been extracted into the mix-in \sh{Roadster::Engines::CSPMethods}.


\section{Prototypes}\label{sec:approach:prototypes}
As a way of getting familiar with the Roadster code base, as well as to prove
the concepts worked out during the first elaboration iteration, two prototypes have
been developed during the coming elaboration iterations. Andy Rohr, as the
main author of Roadster, was immensely helpful by giving an introduction to
the code base early on.  Although quite overwhelming, the first impression was
that the code is clean, makes good use of abstractions and has loosely coupled
classes.

The two prototypes, namely the inter-node communication and the high availability
mechanism, have been developed with the idea of "cheap, quick, and dirty" in mind. This
allows experimentation and minimizes the cost of failure.
The approach to these two prototypes are described briefly in the next two sections.


\subsection{Inter-node communication}
The idea of this prototype is to achieve rudimentary communication between two
Roadster nodes.

To do so, two new actors need to be added for the communication with neighboring nodes of the
federation.  Since COMM actors are used to communicate with systems outside of
a node, the new actors are:
\begin{description}
\item [COMM.UPSTREAM]\hfill\\
This actor is responsible for communication with the direct supernode.
\item [COMM.DOWNSTREAM]\hfill\\
This actor is responsible for communication with direct subnodes.
\end{description}

The necessary classes in each architectural layer have been implemented as
subclasses of
\begin{itemize}
\item \sh{Roadster::Actors::Base},
\item \sh{Roadster::Messaging::Handlers::Base}, and
\item \sh{Roadster::Engines::Base}.
\end{itemize}

To send messages via new actors, the new protocol \gls{FCP} has been defined.
Encompassing a single message type \sh{Roadster::Messaging::FCP::Messages::Hello}, it provides a
simple means to prototype rudimentary inter-node communication. The supernode's
API of this protocol (\sh{Roadster::Messaging::FCP::API::Supernode}) provides
the ability to send messages of the aforementioned message type to all direct
subnodes via the COMM.DOWNSTREAM actor's PUB \zmq socket.

Running multiple instances of Roadster was fairly simple.
To be able to store the supernode instance's \zmq endpoint somewhere and have
it available to the subnode instance, a new domain model class
\sh{Roadster::Domain::Model::Node} has been defined. The endpoint information
is its only property.

This worked as expected. The reception of the periodic test messages sent by
the supernode's COMM.DOWNSTREAM actor are immediately logged on the subnode's
console.

\paragraph{Evolution} The final version of the FCP protocol includes means to route arbitrary
messages through a federation of nodes to its destination actor, as well as
message types for ping/pong heartbeating. \autoref{sec:approach:msg-routing}
and \autoref{sec:approach:hb} go into more detail.

As described in \autoref{sec:scope:csp}, there are three distinct message flows
in the \gls{CSP}. The same applies to inter-node DIM synchronization. A
supernode with multiple subnodes will act as the server in CSP terminology.

\begin{description}
	\item [COMM.UPSTREAM:]
		This actor's responsibility is to provide a communication
		interface to the supernode. To allow fulfilling its set of
		tasks, it registers three different sockets of type
		DEALER, PUSH, and SUB.

	\item [COMM.DOWNSTREAM:]
		To fulfill its part of the CSP, this actor registers
		three sockets of type ROUTER, PULL, and PUB.
\end{description}


\subsection{High availability}
As suggested by the task description in \autoref{app:task-description}, the
\gls{bstar} has been implemented as two standalone Ruby scripts: One for the
server process, and one for the client process. \gls{cztop} is used as the \zmq
Ruby language binding.

All the client does is periodically send a request to its currently selected
server process along with an increasing number, which the active server simply
echoes. In case no response from the server is received within a certain amount
of time, the client destroys its socket and registers a new one to connect it
to the other server.

Manual tests have shown that it works as expected and reliably. The two server
processes (primary and backup) can be started in any order, immediately start
communicating with each other and determine the one to become active. Killing
the currently active server makes the client to switch and re-send its pending
request to the passive server, which then immediately takes over.


\section{Testing}\label{sec:approach:testing}
This section describes the test methods used to check units, the integration of
multiple components, as well as the behavior of new contributions across the
entire framework.  All test results are described in \autoref{ch:res}.

The following subsections describe in detail how testing is performed.

\subsection{Unit tests}
To ensure the correctness of the implementations, unit tests are written using the test framework
RSpec. 100\% coverage of the students' contributions can be achieved by
adhering to \gls{TDD}, or more specifically, \gls{BDD}. RSpec offers itself as
a valuable testing tool providing expressive assertion and mocking
facilities, as well as more advanced features such as testing spies\footnote{\url{https://github.com/rspec/rspec-mocks\#test-spies}} which
are a perfect use case for Ruby's dynamic binding, a.k.a \emph{duck typing}.\footnote{\url{https://en.wikipedia.org/wiki/Duck_typing}}

Naturally, this also simplifies refactoring the code without the risk of
breaking functionality going unnoticed. Unit tests reside under the \sh{spec}
directory of Roadster's code base.

An example of a typical unit test is shown in \autoref{lst:testing:unit}.

% TODO RSpec extension

\begin{listing}
	\begin{minted}[bgcolor=bg]{Ruby}
module Roadster
  describe Actors::BStar do
    subject { described_class.new messaging_class }
    let(:messaging_class) { double 'messaging handler class', new: messaging }
    let(:messaging) do
      instance_spy Messaging::Handlers::BStar, name: 'messaging_handler_name'
    end

    describe '#publish_message' do
      let(:msg) { Messaging::Messages::Base.new 'sender', 'receiver' }
      it 'publishes message to remote HA peer' do
        expect(subject).to receive(:emit_message).with(:bstar_pub, msg)
        subject.publish_message msg
      end
    end

    describe '#register_ha_sockets' do
      # [...]
    end
  end
end
	\end{minted}
	\caption{Example of two unit tests in RSpec.}
	\label{lst:testing:unit}
\end{listing}


% TODO speccing only public methods and side effects of private methods

% TODO * testing env, specs (~150 -> ~900)



\subsection{Integration tests}
Integration tests verify the interaction between the individual components.
To test new core functionality like federation, high availability, and persistence
synchronization, integration tests have been written.

In order to test the failover or synchronization functionality, individual processes
can simply be killed and restarted later if the scenario defines this.


\subsection{System tests}
System tests are designed to test the application under close-to-reality
conditions, treating the application as a black box: On a certain input, it is
supposed to react with a certain output. The internal mechanisms to produce said
output and reach a certain state are insignificant.

% TODO how to test persistence synchronization

\paragraph{Manual} At times when developing a feature, manual testing can give
useful insights, especially when trouble shooting. Running Roadster manually in
foreground within a terminal session helps with that.  When, for example,
developing a new inter-node protocol, running multiple nodes simultaneously
becomes a necessity. Although on the same development machine, this can be
simulated by starting them as different process groups. Through the
configuration, each Roadster instance is given a different set of endpoints,
e.g. different TCP ports or Unix domain sockets. This is possible and feasible
since \zmq completely abstracts the transport away and thus really doesn't
matter if other Roadster instances are running locally or remotely.

\paragraph{Automated} To create an more realistic environment including real
network conditions, a more lower level approach is taken. Stronger isolation
between Roadster instances can be achieved using software containers. Network
virtualization can be used to induce latency and packet loss to simulate the
properties of a congested network.
Docker and Mininet together provide all of the required functionality.

When a feature depends on external sensor data from a field device (e.g. a \gls{PLC}),
a fake adapter is used. It continually fakes the actions of a fictional robot.

\begin{description}
	\item [Docker:]\hfill\\
		Docker is a container software that makes use of the
		\emph{cgroups}\footnote{\url{https://en.wikipedia.org/wiki/Cgroups}}
		feature of the Linux kernel, which allows the isolation of
		collections of processes. This includes isolation of CPU,
		memory, and I/O usage, as well as process ID and network
		namespaces.

		Docker also provides a programmable way to start with a clean
		state at the beginning of every run of the test suites. All
		state changes and file system modifications are discarded after
		the container is stopped.

	\item [Mininet:]\hfill\\
		Mininet allows creating virtual networks instantaneously.
		Like Docker, it also relies on \emph{cgroups}
		and network namespaces to efficiently run multiple \glspl{VM}
		sharing the same kernel, and provide isolation. These very same
		primitives are used by Docker itself.

		Network infrastructure such as packet switching hardware and
		network links can be modeled in Mininet, including the
		simulation of link quality issues such as induced latency and
		packet loss.

	\item [Fake adapter:]\hfill\\
		A fake implementation of a fictional \gls{PLC}, namely a robot
		that that changes its position randomly in a two-dimensional coordinate
		system serves the purpose of getting external sensor data.
		Using Roadster's guards it will also generate cases as soon as
		it moves out of a predefined, virtual bounding box. This robot
		is implemented using about 15 lines of Ruby code and is part of
		the example application provided by mindclue GmbH.
\end{description}

The system test results from each construction iteration can be found in \autoref{ch:res}.


\subsection{Continuous integration}
\Gls{CI} helps prevent problems when integrating big chunks of code changes,
also known as \emph{integration hell}. A new CI build is triggered upon each
push of changes to the repository using a predefined build script.

\paragraph{GitLab CI} Due to the fact that Roadster's GitHub repository is private, online \gls{CI}
services such as \emph{Travis~CI}\footnote{\url{https://travis-ci.com/}, requires payment after the
first 100 builds} cannot be used without payment. Fortunately,
GitLab\footnote{\url{https://about.gitlab.com}} is a free, open-source GitHub-like development collaboration software and includes CI functionality. It can be installed on private infrastructure, such as the
\gls{VM} provided by HSR, where it has been installed and configured.

\paragraph{Auto triggering} Every push of changes to Roadster or the example application triggers the CI
solution set up on the HSR VM. It then installs Roadster's dependencies, the Roadster
framework itself, the example application, and then runs all of Roadster's test
suites. This is useful to get informed proactively when something breaks.

\subsubsection{Documentation build}
It is deemed bad practice to keep generated files in a Git
repository. Having an additional file to commit is tedious and if done
frequently, it can bloat the repository's commit history since Git treats PDF
files as binary files, and thus cannot use efficient algorithms to work out the
differences only.

This is why compiling the TeX sources for this document is also done reactively
via a pipeline configured on GitLab CI. The resulting artifact, a PDF file, is
then automatically uploaded to Git as a file associated with a newly created
public release. This is done via GitHub's webservice API for managing
releases.\footnote{\url{https://developer.github.com/v3/repos/releases/}}


\subsubsection{Docker container}
To provide a clean test environment for every CI build, a Docker container is used.
It prepares the container with all necessary components before every build
which then runs all test suites. The components include:

\begin{itemize}
	\item Mininet
	\item Ruby
	\item Python
	\item The most recent development version of Roadster (of the appropriate branch)
\end{itemize}

The system tests are run on the GitLab server after each construction
iteration. To do so, the Python scripts set up a virtual network and then start
the appropriate Roadster instances. Some features require a link to fail or a
node to fail, which can be achieved at arbitrary times through Mininet commands or
simply kiling Roadster instances wherever needed. The generated log data is
evaluated later to verify that the behavior for each defined feature is
correct.


\subsubsection{Test scenarios}
Test scenarios are deduced from the features specified in \autoref{ch:reqs}.
They contain configurations for Mininet and the particular Roadster nodes.
The procedure of each scenario is modeled in feature step files. This subsection
describes the implementation and general structure of the test scenarios in detail.


Every run of the system tests is based on the current versions of

\begin{itemize}
	\item The Roadster framework itself
	\item The example application
\end{itemize}

In case the current development is taking place in a feature branch, the same
branch is tried to be checked out in both test working copies. For example to
test a new feature like encryption, it is developed in a feature branch called
\emph{encryption} in Roadster's repository. Because the example application
will need modifications as well (such as the knowledge of public keys for each
node in a federation), it those changes are implemented in a feature branch of
the same name.

A shell script creates the appropriate configuration files (fixtures) before
each test scenario is run and ensures that a directory for each Roadster
instance exists where log data is stored during the run.

Mininet is a Python library. To interactively use Mininet, it makes sense to
implement all system test code in Python. This allows taking influence on the
virtual network and on particular hosts within that network at any time. To
parse and evaluate the features written in the Cucumber format, the \gls{BDD}
framework Behave\footnote{\url{http://pythonhosted.org/behave/}} is used.

\subsubsection{Directory structure}

The directory structure used within the Docker container is illustrated
in~\autoref{lst:testing:docker:directory-structure}.

\begin{listing}
	\begin{minted}[bgcolor=bg]{Shell}
/tmp/
|-- roadster
|   |-- ci
|   |   |-- run_system_tests.sh
|   |   `-- run_unit_tests.sh
|   `-- features
|       |-- autonomy.feature
|       |-- environment.py
|       |-- ...
|       |-- steps
|       |   |-- autonomy.py
|       |   |-- behave_util.py
|       |   `-- ...
|       |-- support
|       |   |-- conf.root.primary.rb
|       |   |-- federation.rb
|       |   `-- ...
`-- projects
	`-- ba-roadster-app
		|-- conf.root.primary.rb
		|-- data.root.primary
		|   |-- eventjournal.tct
		|   `-- parameters.tch
		|-- lib
			|-- domain
			|   |-- federation.rb
			|   `-- ...
			`-- ...
	\end{minted}
	\caption{Directory structure of system tests within Docker container.}
	\label{lst:testing:docker:directory-structure}
\end{listing}

\subsubsection{Roadster federation topology}
The federation used for system tests is build as follows:
\begin{figure}[]
	\center
	\includegraphics[width=0.5\textwidth]{img/logical_federation_setup.pdf}
	\caption{Logical federation topology used in system tests.}
	\label{lst:testing:topo:logic}
\end{figure}

The root node is a HA cluster, which acts as such when appropriate according to
the feature being tested. At other times, only its primary node is active and
thus acts as a non-HA supernode of sN1 and sN2. The naming scheme is enforced
by Mininet and thus somehow clunky and cryptic.

The configuration also dictates that there are two simulated robots running on
subnode sN1, which continually update sensor data in the DIM and create new
cases due to their nature of virtually moving around randomly and thus trying
to escape the bounding box.

Subnode sN2 merely acts as a listener and is not directly supervising any field
devices.

\subsubsection{Virtual network plan}
\begin{figure}[]
	\center
	\includegraphics[width=\textwidth]{img/physical_network_mininet.pdf}
	\caption{Physical federation topology used in system tests.}
	\label{lst:testing:topo:logic}
\end{figure}

All nodes are attached to an \gls{ovs} instance which allows them communicate with each
other via the TCP/IP transport. In addition to this star topology, there is an
additional, dedicated link between the two HA peers which form the root HA
cluster.

%----------------------------------------------------------------------------

\section{Port to new \zmq binding}\label{sec:approach:port}
One of the first changes to Roadster's codebase is porting to a new \zmq
binding. This makes sense for the following reasons:

\begin{itemize}
\item To exclude possible failures from faults in the unmaintained ffi-rzmq\footnote{\url{https://github.com/chuckremes/ffi-rzmq}} library.
\item Encryption is needed later anyway, which is not supported by the currently used library.
\item All other tasks involve \zmq communication anyway.
\end{itemize}

There is currently only a single Ruby package that is maintained, supports
encryption, and is freely available, which is \gls{cztop}. Technically it is a
binding for the \gls{czmq} abstraction library, which is the modern and recommended way of
using \zmq. \autoref{sec:zmq:czmq} explains CZMQ in further detail.

As stated in the task description already, Roadster's event loop makes use of
the \zmq options \sh{ZMQ_FD} and \sh{ZMQ_EVENTS}. Getters for these had to be
added to in CZTop, which was a matter of minutes.

Due to Roadster's software architecture where different concerns
are cleanly separated, all code that actually depended
on ffi-rzmq was located in a single file. The following
things needed to be done:

\begin{itemize}
\item Load CZTop instead of ffi-rzmq.
\item Remove code to manually send and receive the parts of multi-part messages. This has been simplified
in CZMQ and thus is now a single method call.
\item Remove error checking code. CZTop always checks error codes, and raises
an appropriate exception if needed. These exceptions are now translated to Roadster's internal exceptions.
\item Simplify code that deals with \zmq options such as \sh{ZMQ_FD} and \sh{ZMQ_EVENTS}.
\item Replace library calls to use CZTop instead of ffi-rzmq.
\end{itemize}

This was about an hour's work.


%----------------------------------------------------------------------------
\pagebreak
\section{Federation}\label{sec:approach:federation}
Running multiple Roadster nodes can be useful to aggregate information from
several field devices, especially if they cannot be connected to the same node,
possibly because of physical distance. Such a federation of autonomously
running nodes can then aggregate information in the root node, which can then
provide it to higher level client systems. Also, the root node's web UI shall
serve as a centralized point of access to the federation and offer an overview
of it to operational and technical personnel.

A Roadster federation consisting of a root node and two subnodes is illustrated
in \autoref{fig:federation}.
\begin{figure}[]
	\includegraphics[width=\textwidth]{img/federation_protocol.pdf}
	\caption{Federation between a supernode and two subnodes.}
	\label{fig:federation}
\end{figure}

Adding federation functionality to Roadster involves the following aspects:
\begin{itemize}
	\item Means to declare a federation
	\item Message routing
	\item Heartbeating
	\item Inter-node CSP
\end{itemize}

\subsection{Federation definition}
The federation topology has to be defined somewhere. This can be done using a
\gls{DSL} and then put into a static file shared
on all nodes of a Roadster federation. Each actor could then read the file at
startup, just like it is done for other configuration pieces of a Roadster node.

\autoref{lst:dsl:topo:no-ha} shows how such a configuration might look.

To allow the actors of a node to deduce which node they belong to, it is
necessary to provide an additional configuration, e.g. \sh{conf.local_node}.
This option then can be set in the node-specific configuration file as shown in
\autoref{lst:conf:root}.

\begin{listing}
	\begin{minted}[bgcolor=bg]{Ruby}
# Roadster application configuration
module Roadster

  conf.local_node = 'root'

  conf.log_level = ::Logger::Severity::INFO
  conf.pid_file  = File.join conf.root, 'log', 'root.pid'
  conf.log_file  = File.join conf.root, 'log', 'root.log'
  conf.data_path = File.join conf.root, 'data.root'

  conf.webserver_port = 3010
  conf.environment = :development
end
	\end{minted}
	\caption{Node-specific configuration of the node 'root'.}
	\label{lst:conf:root}
\end{listing}

Not every node in a federation is the same: Each one has its own set of field devices (if any).
The configured name of the local node can be used in conjunction with the
federation topology definition to deduce important information such as neighboring
nodes and the exact set of adapters to load on the local node.

\begin{listing}
	\begin{minted}[bgcolor=bg]{Ruby}
module Conf::Federation
  def self.conf
    proc do
      primary_endpoints \
        router: 'tcp://0.0.0.0:20000',
        pull:   'tcp://0.0.0.0:20001',
        pub:    'tcp://0.0.0.0:20002'

      adapters do
        adapter :foobar do
          # config for 'foobar' adapter and its peers
        end
      end

      node :s1 do
        adapters do |s1_node|

          adapter :simu do
            label           'SIMU'
            desc            'Process simulation adapter'
            adapter_class   Roadster::Adapters::Simulation

            peer :robot_controller_01 do
              label 'robot_01'
              uri   ''
            end

            peer :robot_controller_02 do
              label 'robot_02'
              uri   ''
            end
          end
        end

        objects do |s1_node|
          robot :robot_01 do
            label_i18n  de: 'Mein schicker roboter nummer 1',
                        en: 'My fancy robot number 1'
          end
          s1_node.objects.robot_01.reference_peer s1_node.adapters.simu.robot_controller_01

          robot :robot_02 do
            label_i18n  de: 'Mein schicker roboter nummer 2',
                        en: 'My fancy robot number 2'
          end
          s1_node.objects.robot_02.reference_peer s1_node.adapters.simu.robot_controller_02
        end
      end
    end
  end
end
	\end{minted}
	\caption{Federation DSL example without HA.}
	\label{lst:dsl:topo:no-ha}
\end{listing}


\subsubsection{Inter-node sockets}
To provide all functionality used by any of Roadster's messaging protocols, the
new actors from the prototype are completed according to the illustration in
\autoref{fig:federation}. As a result, the COMM.DOWNSTREAM actor registers a
ROUTER, a PULL, and a PUB socket, whereas the COMM.UPSTREAM actor registers a
DEALER, a PUSH, and a SUB socket.

\subsection{Message routing}\label{sec:approach:msg-routing}
Messages need to be sent from an actor on one node to an actor on another node.
The best place to put this logic is the CORE actor which already does this for
messages exchanged within a node. The existing mechanism needs to be extended to take other nodes
into consideration as well. Then messages can be passed around hop-by-hop.

Message routing shall be stateless, following the well-known and battle proved
example of routing in \gls{IP}. This means that even a message sent in
\emph{Dialog} mode shall not leave pending resources acquired on nodes other
than the source and the destination node.


The \sh{Roadster::Messaging::Messages::Base} class has been extended to accept
not only a destination actor as recipient, but also a node. The simplest way
possible, which also happens to be very convenient when writing protocol code,
is to follow the email example, e.g. \sh{actor@node}. The \sh{node} is actually
the full path of a node, starting from the root node. As an example, a message
sent to the CORE actor of the node \sh{root.s1}, the recipient shall be
\sh{core@root.s1}.

If the recipient node is not specified, the local node shall be inserted
automatically upon sending. The same applies to the source node. This is to
ensure that legacy code unaware of the federation feature still works as expected.

Any message that is to be routed shall be sent to the CORE first, which was
already the case before this bachelor thesis. What the CORE now does, is
inspect the message's recipient: If it's destined to the local node, it is
handled as always. Only if the destination node is a remote node, the new
routing logic in \sh{Roadster::Messaging::FCP::Core} comes into action.

Routing within a hierarchical topology is very easy, especially when the
recipient's full path within the hierarchy is part of the message. If the
recipient node's path is not a prefix of the local node, the next hop has to be
the supernode. Otherwise, it has to be one of the subnodes, which is determined easily,
being simply the path one component longer than the one of the local node.

Due to the hierarchical property of Roadster federations, network loops are
definitely not present. This simplifies the routing mechanism further, because
no means of limiting a message's lifetime is necessary, as it is in \gls{IP} in
the form of a time-to-live field. IP needs this mechanisms because network
topologies can change at any time, even creating loops, which could then result
in a packet circulating indefinitely.

Any wrong recipient set in the message results in the message being discarded
immediately by the ROUTER socket if it is not aware of a corresponding remote socket.

 The complete code for this routing algorithm is illustrated in
 \autoref{lst:approach:msg-routing}.


\begin{listing}[H]
	\begin{minted}[bgcolor=bg]{Ruby}
class Core < BaseAPI
  def initialize(messaging)
    super
    @subnode_regex = /^#{Regexp.escape local_node}\.\w+/
    @supernode = local_node.split('.')[ 0 .. -2 ].join('.')
  end

  # Forward message using messaging layer. Either to local actor or to
  # remote node.
  #
  def forward_message(msg)
    if local_node == msg.receiver_node
      # local delivery
      msg.add_envelope msg.receiver
    else
      set_next_hop msg
    end

    messaging.forward_message msg
  end

  private

  # Lookup and set next hop(s) of message using "string based"                                                                                                             # routing. It's either forwarded to the supernode or one of the
  # subnodes.                                                                                                                                                              #
  # @param msg [Messaging::Messages::Base] the message to prepare for                                                                                                      #   the next two hops to another node
  #                                                                                                                                                                        def set_next_hop(msg)                                                                                                                                                      if subnode = msg.receiver_node[ @subnode_regex ]
      # forward to/via subnode
      msg.add_envelope "comm.upstream@#{subnode}"      # second hop
      msg.add_envelope "comm.downstream@#{local_node}" # first hop
    else
      # forward via supernode
      msg.add_envelope "comm.downstream@#{@supernode}" # second hop
      msg.add_envelope "comm.upstream@#{local_node}"   # first hop
    end
  end
end
	\end{minted}
	\caption{Message routing algorithm.}
	\label{lst:approach:msg-routing}
\end{listing}


\subsubsection{Example}
When a user of the root node's web UI wants to change a value on a field device
connected to the root node's subnode, a command is sent from the browser to the
root node's COMM.WEBUI actor. From there it is sent to the CORE actor, which
routes it via the DOWNSTREAM actor to the subnode. There it is relayed by the
COMM.UPSTREAM actor to the the CORE actor, which then routes it locally to the
correct COMM actor, which in turn executes the command on on the field device.

\subsubsection{Heartbeating}\label{sec:approach:hb}
To avoid filling a PUSH socket in case the supernode is unavailable, and also
just to keep an up-to-date view of neighboring node's liveliness, heartbeating
is performed between a supernode and its subnodes.


Two new message types have been defined:
\begin{itemize}
	\item \sh{Roadster::Messaging::FCP::Messages::Ping}
	\item \sh{Roadster::Messaging::FCP::Messages::Pong}
\end{itemize}

Two-way heartbeating (ping/pong) is used, where the supernode periodically
sends a Ping message to all its direct
subnodes. This is done via the COMM.DOWNSTREAM's PUB socket (one-to-many).
Every subnode then replies with a Pong
message, done via the COMM.UPSTREAM's DEALER socket (one-to-one).

Each received Pong message proves to the supernode that the respective subnode
is responsive. Of course, this also applies to the Ping message: Receiving a
Ping message tells the subnode that its supernode is responsive.

The implementation actually sends a sequence number with every Ping message,
which is then echoed as part of the Pong message. This makes it possible to
recognize the situation of a much delayed Pong message, including the
calculation of the exact delay (spanning multiple Ping periods).

Why are Pings not sent in the \emph{Dialog} mode? Dialog messages are not
suitable for one-to-many communications.

The \gls{ZMTP} version 3.1, as supported by recent versions of \zmq, actually
contains a heartbeating feature. But ZMTP 3.1 is not stable yet as of this
writing, and in case a remote socket becomes unresponsive, the connection is
silently disconnected, which is not very useful to keep track of neighboring
nodes' liveliness. Its intended use is to detect stale TCP connections and thus
release possibly blocked processes.

For the aforementioned reasons, application heartbeating is performed.

Due to the message routing being used for the Pong message, the a full cycle of
Ping/Pong involves the following actors, starting with the Ping message:
\begin{enumerate}
	\item COMM.DOWNSTREAM of the supernode
	\item COMM.UPSTREAM  of the subnode
	\item COMM.CORE of the subnode
	\item COMM.UPSTREAM of the subnode
	\item COMM.DOWNSTREAM of the supernode
	\item COMM.CORE of the supernode
	\item COMM.DOWNSTREAM of the supernode
\end{enumerate}

This is good because it means both CORE actors of any supernode-subnode pair
are involved. In case, for some reason, the CORE actor is unresponsive, Ping
messages are not simply echoed by the COMM.UPSTREAM actor, which is the desired
behavior.


\subsection{Inter-node CSP}
The DIM has to be replicated across a hierarchical federation of nodes, as the
DIM is the essential data structure representing the overall state of a
Roadster federation.

One of the new actors' responsibility is the replication of updates to the DIM,
similar to how the \gls{CSP} works within a single node.  To do so, both the
COMM.UPSTREAM and COMM.DOWNSTREAM actor simply subscribe to DIM updates within
a node just like any other COMM actor does.  Then they replicate any subsequent
updates over the network to the neighboring nodes. The existing three socket
pairs are used, since they are protocol agnostic thanks to Roadster's layered
architecture.

\paragraph{Replication is bidirectional}
It is possible that two sibling nodes will have performed modifications to the
DIM while they were disconnected from their common supernode. Thus, the initial
exchange of snapshots, as well as the subsequent propagation of modifications
to the DIM has to be bidirectional.

\paragraph{CAP theorem}
The CAP theorem \cite{wp:cap} states that it is impossible for a distributed
computer system to simultaneously provide consistency, availability, and
network partition tolerance. In the face of a network partition, one has to
chose between availability and consistency. Because subsystems of a Roadster
federation must be autonomous, the obvious choice is availability. Eventual
consistency is achieved when recovering from a network partition by simply
reinitiating the DIM replication process.

\subsubsection{Object ownership}
The concept of ownership has to be introduced. Every object within the DIM
shall belong to a particular node. Modifications to the objects shall be done via
that node, and only that node, so the node is the authorative instance for all modifications of its own objects.
For that, every object in the DIM has to have a reference
to the object representing its owner node, which is part of
the configuration that is loaded from the static configuration loaded by every
actor at initialization.

Some objects like parameters (described in \autoref{sec:scope:persistence})
will apply to a whole federation and will thus be owned by the root node.
Examples of these are UI user credentials.

To avoid having to store an additional reference to a node object in every DIM
object, the process of finding its owning node can be simplified by rearranging
the objects. Making all objects direct or indirect children of the owning node
means that determining their owning node is merely a traversing of the tree up
to the first node object.

As a result, the existing domain model class \sh{Roadster::Domain::Model::Root}
has been eliminated. Instead, the root node's
\sh{Roadster::Domain::Model::Node} object takes its place in the DIM. As a
consequence, the whole configuration of adapters, field devices, aggregates,
and such now belong to the root node. Of course, the same goes for any subnodes
and their configuration.

\subsubsection{Degree of replication}
There are several variants when it comes to what exactly of the DIM should be replicated:

\paragraph{Variant 1: Self-subtree only}
Synchronize on subtree only, which means a node only knows the DIM part of
itself. The big disadvantage is that it will not have a copy of the rest of the
DIM, which can be useful to inspect variables on neighboring nodes, especially
when they're unreachable.

\paragraph{Variant 2: Sync complete tree}\label{par:approach:dim:var2}
Always sync on complete tree, which means getting the snapshot from the
supernode and merge the own subtree into it, replacing whatever subtree is
already there. This works because of the autonomy requirement for nodes and
their subnodes. This variant is very easy to implement at first.

\paragraph{Variant 3: Either sync on subtree or complete tree}
Make it configurable: Either sync on subtree or on complete tree. The
topology DSL would allow to specify this property for each node. This is
the best of both worlds, but more effort.

Variant 2 will be the first step, as following the \gls{KISS} principle is one
of the non-functional requirements. Variant 3 will be the second step, if at
all. This depends on whether it will be needed performance-wise.

\subsubsection{Avoiding code duplication}
The \gls{CSP} has to work as an intra-node protocol (within a node, which is
the legacy behavior), as well as an inter-node protocol (within a federation).
That means \gls{CSP}-related server functionality has to be present in the CORE
engine, as well as in the DOWNSTREAM engine. CSP-related client functionality has
to be present in all existing COMM engines, the UPSTREAM engine, \emph{as well
as the DOWNSTREAM engine}, since it also acts as a CSP-client to the CORE engine on
the same node.

To avoid a considerable amount of code duplication, CSP-related functionality
is extracted into its own reusable module, which is then mixed into all
relevant actors. The \emph{Template Method} design pattern is used to handle
the common case of handling a received
\rb{Roadster::Messaging::CSP::Messages::ModelUpdate} message. The hook methods,
called by the template method, are then implemented 
in each particular class where necessary. Where no special behvior is needed,
the mixed-in methods are left unchanged.

\subsubsection{Ignoring update reflections}
Because of the way the \gls{CSP} works, modifications to the \gls{DIM} pushed
to the supernode have to be published back to all subnodes, just like the CORE
actor publishes modifications to back to all local actors. That means the update is
also reflected back to the subnode on which the update originated. A subnode's
UPSTREAM actor has to recognize this situation and discard the update in question
immediately.

This is easily achieved by comparing the version number, because a reflected
update always has a version number $\leqslant$ the local version number.

\subsubsection{Version monotonizing}
In case a node has to be rebooted due to whatever reason, its version numbers
start over at zero.
This leads to the situation where its domain model updates sent to neighboring
nodes are discarded as being outdated, until they reach their previous value.
This is why snapshots from neighboring nodes are initially requested to
instantly increase the version numbers to their previous value.

The implementation is straight-forward. When applaying a snapshot --- as well as regular
updates --- a mere monotonical increase of the version number is enforced by
taking the maximum of the two, no matter if the increase was caused by local or
an foreign snapshot.

\paragraph{Bootstrapping}
When starting an entire federation of nodes, and each node initializes foreign
node versions to zero, the very first update or snapshot from that node would
be recognized as outdated and thus discarded. This is why all
\sh{Roadster::Domain::Model::Node} instances representing foreign nodes are
initialized to version -1, so the very first update ($\geqslant 0$) is always
applied.
% TODO: supernode implementation braucht dieses Funktion auch - weil

\begin{figure}[]
	\includegraphics[width=\textwidth]{img/activity_diagram_upstream.pdf}
	\caption{CSP upstream activity diagram}
	\label{fig:csp:upstream:activty:diagram}
\end{figure}

\begin{figure}[]
	\includegraphics[width=\textwidth]{img/activity_diagram_downstream.pdf}
	\caption{CSP downstream activity diagram}
	\label{fig:csp:downstream:activty:diagram}
\end{figure}

\begin{figure}[]
	\includegraphics[width=\textwidth]{img/sequence_diagram_model_update_pub.pdf}
	\caption{CSP version sequence diagram}
	\label{fig:csp:version:sequence:diagram}
\end{figure}

Beim Start eines Nodes werden die foreign update Versionsnummern auf -1 gesetzt, so
dass der erste Snapshot nicht discarded ist.

\subsection{Access control}
% TODO: DIM access control

\subsection{Remote case management}
% TODO gem 'simple_states' failed and thus was dropped, reduced complexity
% TODO how to write your own FSM


%----------------------------------------------------------------------------
\section{High availability}\label{sec:approach:ha}
If Roadster is going to be run in a federation, measures need to be taken to
mitigate the risk of failure, since many nodes are more likely to fail than a
single node. Availability shall be ensured by
adding redundancy on certain levels of the node hierarchy (e.g. at the bottom
of the topology, or at root level), in the form of a
fully functional backup node in addition to the primary one.

Run together in a hot-standby cluster, the passive node's responsibility is to
take over in case the active one fails to deliver its service.



\subsection{Defining reliability}
When speaking about reliability, it is worth listing the failures the solution must be
able to handle. According to the requirements, and thinking a bit further,
handleable failures of the following three categories have been identified:

\begin{description}
	\item [Hardware failure on the active node:] \hfill\\
		No hardware runs forever. Any of the following things could
		happen spontaneously at any time:

		\begin{itemize}
			\item A non-redundant disk can fail fatally
			\item Memory can encounter an irrecoverable error
			\item The \gls{CPU} catches fire
			\item The node's power supply starts smoking
			\item The node's switch port can break
			\item The node's \gls{NIC} can break
			\item A power or network cable mysteriously fail due to wear and tear
			\item A random power outage that only affects one node
			\item Someone accidentally pulling the node's power
				plug or network cable (human-induced)
		\end{itemize}

	\item [Software failure on the active node:] \hfill\\
		Bug-free software is rare, if it exists at all, and the
		following scenarios are possible:

		\begin{itemize}
			\item Roadster crashes, e.g. the CORE actor becomes unresponsive
			\item the disk becomes full
			\item the memory becomes full (and triggers the \gls{oom-killer})
			\item the whole OS freezes or crashes (less likely, but possible)
			\item Accidental shutdown of the currently active HA
				peer, either just the Roadster application, or
				the whole node (human-induced)
		\end{itemize}

		More about handling crashes of certain actors can be found in
		\autoref{sec:approach:ha:hb}.

	\item [Network failure:] \hfill\\
		This only includes:
		\begin{itemize}
			\item Failure of the link connecting a HA node to the
				rest of the federation.
			\item Configuration mistakes in
				switching/routing/firewall equipment which cut
				off the currently active HA peer (human-induced)
		\end{itemize}
\end{description}

Handling a failure means an actual service interruption can be avoided.

\subsubsection{Failures not handled}
Failures that do not have to be or cannot feasibly be handled include:

\begin{description}
	\item [Link failure between a subnodes and one of the HA peers:]\hfill\\

		This cannot be handled since the two HA peers would have to
		continually share the number of subnodes connected to them, and
		based on that, make a decision on which one should be active or
		passive. Since the link between them could fail as well, this
		decision cannot be done reliably, which could lead to the
		dreaded split brain syndrome.

		Actually the \gls{bstar} could be extended to support this kind
		of mechanism. The details are described in
		\autoref{sec:approach:ha:bstar-ext}.


	\item [Link failure between a HA peer and a field device:]\hfill\\
		The \gls{bstar} algorithm will not initiate a failover since the
		active peer is still responsive and is able to send heartbeats to the passive node.

		The interrupted communication with the affected field device
		could and definitely should cause an alarm, but no failover,
		since they're only half of the conditions that have to be met
		for a failover.

		\autoref{sec:discussion:ha:manual-failover} describes a
		proposal for manually induced failovers.

	\item [ Complete network outage:]\hfill\\
		A complete network outage, such as caused by a routing or
		firewall configuration mistake, cannot be handled gracefully by
		Roadster.

	\item [ Non-redundant hardware failure:]\hfill\\
		In case non-redundant hardware fails, such as a whole network
		switch, cannot be handled gracefully by Roadster.

	\item [ Complete power outage:]\hfill\\
		A complete power outage will not be handled gracefully by
		Roadster. Providing a \gls{UPS} to keep Roadster nodes and
		network equipment up and running in such a case is matter of
		the respective facility in question.

	\item [ Natural disasters:]\hfill\\
		These cannot be handled gracefully by Roadster. Damage from an
		earth quake, a fire, meteorological disasters, solar flares, or
		similar, are outside of the scope of Roadster's high
		availability feature.
\end{description}

The \gls{zguide} describes a very simple mechanism to achieve this kind of high
availability with exactly two redundant nodes: The \gls{bstar}. It
provides a set of clients a highly available service by running two server
nodes in a hot-standby setup. It is simple and thus very robust, takes measures to avoid the
split-brain syndrome, and is fairly easy to implement, even as reusable
code.

The implementation could be contained within a new kind of COMM actor
called BSTAR. This makes sense since, in a way, it talks to the outside world.
That part of the outside world just happens to speak the same language. This
means that the new kind of actor most likely will not need an adapter like COMM
actors usually do in Roadster.

\subsection{Binary Star in a nutshell}
Two HA peers are started either as primary or as backup. After an initial
handshake, the primary one becomes active, the backup node becomes passive. The
two continually exchange heartbeats (and their current state with regards to
the \gls{bstar}). Clients always connect to the primary node's endpoint first. This
is illustrated in \autoref{fig:ml:ha}.

\begin{figure}[]
	\includegraphics[width=\textwidth]{img/ML-HA_bstar.pdf}
	\caption{Schema of HA cluster and a higher level client.}
	\label{fig:ml:ha}
\end{figure}

\begin{figure}[]
	\includegraphics[width=\textwidth]{img/state_machine_diagram_bstar.pdf}
	\caption{BStar machine state diagram}
	\label{fig:bstar:state:diagram}
\end{figure}

\subsubsection{Socket types}
The sockets used by both HA peers to send and receive heartbeats and state
updates is a PUB-SUB pair each. Using PUB-SUB sockets makes sense to avoid queues filling up
in case of an actual failure, which PUSH-PULL sockets would do. PUB sockets simply drop messages when there are
no SUB sockets connected, which is the desired effect.


\subsubsection{Configuration}
The DSL has been extended to support an additional property called
\sh{backup_endpoints}. Furthermore, the hashmaps stored in the properties for
the primary and backup endpoints simply contain another endpoint each: The ones
used by the COMM.BSTAR actors. This is illustrated in \autoref{lst:dsl:topo:with-ha}.
Using this information, direct subnodes of a HA-node can determine the correct
supernode, as well as know the other node's endpoints in case the currently
active HA peer fails.



\begin{listing}
	\begin{minted}[bgcolor=bg]{Ruby}
module Conf::Federation
  def self.conf
    proc do
      primary_endpoints \
        router: 'tcp://0.0.0.0:20000',
        pull:   'tcp://0.0.0.0:20001',
        pub:    'tcp://0.0.0.0:20002',
        ha_pub: 'tcp://0.0.0.0:20003'

      backup_endpoints \
        router: 'tcp://0.0.0.0:20010',
        pull:   'tcp://0.0.0.0:20011',
        pub:    'tcp://0.0.0.0:20012',
        ha_pub: 'tcp://0.0.0.0:20013'

      adapters do
        # telnet server for client systems (fake OPC-UA HA interface)
        adapter :telnet do
          label           'Telnet'
          desc            'Telnet Server'
          adapter_class   Roadster::Adapters::TelnetServer

          peer :server do
            label   'LTA'
            desc    'LTA'
            uri     "tcp://0.0.0.0:#{Roadster.conf.telnet_server_port}"
          end
        end
      end

      node :s1 do
        # [...]
      end
    end
  end
end
	\end{minted}
	\caption{Federation DSL example with HA.}
	\label{lst:dsl:topo:with-ha}
\end{listing}



\subsubsection{OPC UA}
One of the non-functional goals demanded the HA solution be developed with
\gls{opc-ua} in mind. OPC-UA describes several ways of non-transparent server
redundancy, which seems to map closely to how the \gls{bstar} works.

\subsection{Failover}
The passive node takes over when the following two conditions are met:

\begin{enumerate}
\item No life signs from the active node
\item Connection requests from clients
\end{enumerate}

The second condition is to prevent the split-brain syndrome and thus can be
thought of as an external vote for the node to actually initiate the failover.
This works because clients will try to (re-)connect to the HA peers in
round-robin fashion.  This algorithm is explained in \cite[Chapter 4 - Reliable
Request-Reply Patterns, Client-Side Reliability (Lazy Pirate
Pattern)]{zmq:zguide}.
% TODO: add Ruby example

In case the currently active peer crashes, the two conditions will be met.
This means that the passive node starts accepting snapshot requests (ICANHAZ
messages) and updates the DIM, so every other node will know about the new,
active node. This is needed for the message routing to work.

To establish or destroy connections (either to other nodes or to field
devices), callbacks from the \gls{bstar} library can be used to react on
becoming active and becoming passive. To do something in another actor,
Roadster's messaging infrastructure can be used.

% TODO Going active:
% * Core: guards (cases), ...
% * IoComm: peer connections
% * Downstream: subnode pinging, CSP publishing
% * Upstream: CSP pushing to supernode
% * Storage: ?
% * Logger: -


% TODO: Going passive
% TODO passive node: why not Ping subnodes too? Upstream sockets would have to be
%  connected to two nodes. => implement ROUTER-ROUTER


\subsubsection{No client, no failover}
A failover can only happen if there is at least one client. With no clients, a
vote cannot happen. Thinking about it, it is not a bad thing if the failover
cannot happen in that case, since there is no client depending on the service
brought by a HA cluster.

\subsubsection{Alarm generation}
% TODO using Guard
When a failover happens, it makes sense to create an alarm case in the
DIM, so the outage is visible to operational personnel in one of the web UIs.
The same applies to the case where the passive node goes down, although it
does not have an immediate effect on availability.  This is so the operational
personnel can act upon the alarm and e.g. initiate field forces to inspect the
failed node and repair it.

Once repaired, it is restarted with the exact same configuration --- either primary
or backup. Since there is already an active node (either the primary one,
or the backup one), the newly repaired node will become the new passive node.

\subsubsection{Failover from backup to primary node}
Once failed over, the newly active backup node stays active. It does so until it
fails itself or is manually stopped. It never automatically switches back to
make the primary node the new active one without a failure. This is key. If a
node becomes unreachable, the failover happens automatically, but anything else will
require human interaction.

Subsequently, when the previously broken, primary node has been repaired, it
rejoins the \gls{bstar} cluster as the passive node. At that point, the backup
node can be killed if need be, and the primary node will take over again.  This
works because the \gls{bstar} operates symmetrically after a successful
handshake during initialization.

\subsection{Replication}
% diagonal HA
The passive HA peer needs to stay up-to-date, e.g. take part in DIM
synchronization. Instead of clients replicating modifications to both HA peers, as
described in \cite[Chapter 5 - Advanced Pub-Sub Patterns, The Clustered Hashmap
Protocol]{zmq:zguide}, the passive HA peer could just act as a supernode to the
active HA peer. This eliminates the need for an additional protocol. In case of
a failover, the roles of course need to be exchanged.

% TODO: what if HA cluster has supernode
{\color{red}A problem would be the case where the HA cluster has a supernode (although
considered an unusual case). That would mean the active peer has two supernodes
connected at the same time, which is not supported by the planned communication
infrastructure (especially with encryption, see
\autoref{sec:approach:encryption:ha}).  A solution could be to perform the
replication to the HA peer only via sockets of the two BSTAR actors.}

\subsection{Rolling upgrades}
% OPTIMIZE move to Discussion
An obvious side benefit of this comes into play with upgrades. Using the HA
cluster capability it is possible to perform rolling upgrades, meaning one HA
peer can be upgraded while the other one stays in service. After a successful
upgrade, Roadster on the active peer can be stopped, upgraded, and restarted as
the passive peer.

\subsection{A note on dedicated links}
The \gls{zguide} mentions that a dedicated network link between the two HA peers
(traditionally done with a crossover cable) is the best solution to prevent the
split-brain syndrome. This is true. But in some cases, it could also prevent a
failover from happening.

Imagine part of the currently active peer's network equipment fails, i.e. one
of its \glspl{NIC} connected to the switch fails, or one of its switch ports
fail. In that case it will be unreachable to the rest of the
federation or to the client. A failover would be desired. But it
cannot happen, since heartbeats will still be exchanged with the passive node over
the dedicated link. So it is a trade-off between risking the split-brain
syndrome and not being able to perform a failover.

\subsubsection{Extending the \gls{bstar}}\label{sec:approach:ha:bstar-ext}
% vision
Of course the \gls{bstar} mechanism could be extended to handle cases where the
active peer loses connectivity to its clients, but not to the other HA peer. In
a case like that, the passive node could actually take over to minimize
service downtime. The active HA peer could communicate the number of fully connected clients,
which would be zero in case it is cut off from its clients. That way, the
passive node could recognize the situation correctly, since it will get
requests from clients, which are trying to fallback to the passive peer since
the active peer is unresponsive. Knowing that the active peer has zero
connected clients, it could promote itself to the new active peer.

% what's needed for higher-level clients
For this to work, a way of counting fully connected (not just requesting)
clients has to be introduced. The COMM actor providing the \gls{opc-ua}
interface would know best how many clients are connected. This would solve the problem for single-level HA setups.

% what's needed for subnodes as clients
In case of a multi-level federation with a HA cluster at its root, the
cluster's subnodes form another set of clients. Because nodes within a
federation communicate with each other via \zmq sockets, telling how many
clients are connected is a bit less straight-forward, because \zmq abstracts
connection handling completely away. \zmq declares such needs as application matter.
The subnodes would have to implement application-level heartbeating.
That is, the subnodes that are fully connected at a given point in time (i.e. enrolled
in the DIM replication) would periodically send heartbeats which could then be
used to maintain an up-to-date list of active clients (where stale client
entries automatically expire).  The number of active clients (the size of the
list) could be communicated privately just between the two HA peers, along with
the heartbeats. Publishing that information via the DIM might be problematic,
since DIM updates go over the main network link.

% step down from active to passive
Something else needed is a HA peer's ability to step down from being the
active node as soon as its peer promotes itself to the newly active peer. In
the standard \gls{bstar} mechanism, this would be recognized as the split-brain
syndrome and handled fatally.

% discard because KISS
Doing this without an actual need would be a violation of the KISS principle,
so the first approach will not include any extensions to the \gls{bstar}.

\subsubsection{Corner case}
A highly unlikely, but dangerous corner case \emph{could} lead to the
split-brain syndrome. The following steps would be required to happen:

\begin{enumerate}
	\item The backup node is currently the active one

	\item Then the dedicated link fails, which means no heartbeats are
		exchanged anymore

	\item A new client tries to connect or an existing one is temporarily
		gone and tries to reconnect
\end{enumerate}

Since clients try to contact the primary node first, this could
lead to dreaded the split-brain syndrome. The primary node does not hear any
life signs from the backup node (the rightfully active peer), and client
requests count as votes. This means it will promote itself to the active peer,
even though the backup node is still active.

\subsection{A note on heart beating}\label{sec:approach:ha:hb}
Special attention needs to be paid when it comes to sending heartbeats within a
HA cluster. A na\"ive developer might implement the BSTAR actor so it
autonomously sends heartbeats. This works as long as the failures only affect
the hardware. But what if a software error happens in the CORE actor?  A
Roadster node certainly cannot function without the CORE actor. In that case the
BSTAR actor should not send out heartbeats anymore.

A better implementation would have the BSTAR actor periodically inquire (PING
request) the CORE actor whether it is still healthy, and only send out a
heartbeat in case it gets a response (PONG) in time. This can be done with a
simple mechanism, where the CORE actor is programmed to respond to every PING
request with a PONG response. Similar infrastructure of Roadster could be used
instead, if it already provides it.

Further ideas based on this fundamental mechanism are discussed in
\autoref{sec:discussion:imp:ha:hb}.

%----------------------------------------------------------------------------
\section{Persistence synchronization}\label{sec:approach:psync}
% TODO: big picture
The persisted data and updates to it, handled by the STORAGE actor, need to
bubble up and collected in the root node.

\subsection{Aspects}
There are multiple aspects involved in persistence synchronization:

\begin{description}
	\item [Initial synchronization:]\hfill\\
		How does one get the initial delta of updates since the last
		synchronization?


		\paragraph{Event journal} Event journal data is not
		append-only. Existing cases can still be modified, e.g. to
		confirm them. So the solution to get the full delta of
		modifications, but not more than needed, is to request the
		oldest non-finished case for each owning subnode (direct and
		indirect ones).

		\paragraph{Time series} The good thing about time series is
		that the records, once written, will not be modified. That allows
		for a very simple method of getting new time series records by
		just requesting all records added after a particular point in
		time, per series. However, since old time series data is
		deleted after a certain amount of time (e.g. 6 months), the
		timestamp of oldest record remaining also needs to be requested
		by the supernode, per series, so it can delete them from its
		local files as well.

		\paragraph{Parameters} Every param is owned by a node, and is
		synchronized through the DIM.  What is persisted is merely the
		last updated value so the node is able to boot with not only
		sensible defaults, but actual values that are most likely still
		correct. This means that there is no delta of modifications to
		communicate. Modifications to parameter objects in the DIM just
		need to be persisted.

	\item [Continuous synchronization:]\hfill\\
		Send further updates, one-by-one, in an event-driven manner.
		This is only needed in case the solution aims for event-driven
		(i.e. close to instantaneous) synchronization.

		\paragraph{Event journal} Non-finished cases live in the DIM.
		That means any node could theoretically persist a history of
		all cases owned by its subnodes (direct and indirect ones) by
		itself. Another way would be for any node to only create a
		persisted history its own cases and then communicate the
		modifications to the persisted data to its supernode. Every
		node would in turn include modifications from its subnodes
		(direct and indirect ones) in the stream of modifications to
		its supernode. {\color{red}It is not clear yet, what is the best
		solution here.}\\

		\paragraph{Time series} The most recent values of sensor data
		is held in the DIM. That means any node in the federation could
		do the sampling and persisting of time series data by itself.
		{\color{red}It is not clear yet what is the best solution here.}\\
		The deletion of old records does not have to be bubbled up,
		since that is done by a Roadster node autonomously whenever it
		persists a new one.

	\item [HA peer sync:]\hfill\\
		How is the passive HA peer updated?
		This not only matters when the supernode is a HA cluster (multi
		level), but also when it is at the bottom of the node hieararchy
		(single level).

		% diagnoal HA
		Same as with the DIM synchronization between HA peers, this
		could be done by simply treating the passive HA peer as a
		supernode.
\end{description}


\subsection{Variants}
There are multiple variants to achieve the needed functionality.

\subsubsection{Polling only}
The supernode just periodically request persistence
deltas. This would be handled over a DEALER-ROUTER pair of sockets. The nice
thing about this variant is that the subnode only has to do one thing, which is
responding to requests from the supernode(s); it does not have to proactively
send any updates after sending the an initial delta.

A big drawback is that the synchronization only happens periodically. This
does not seem to fit well into the overall Roadster architecture, which is
completely event-driven (no polls or "sleeps").

Another drawback is efficiency. This variant will periodically cause the
subnode's database to be searched for possibly large amounts of keys. Depending on the size of the
database and the efficiency of searching through keys, this could be a lot of
wasted resources or even cause bottle necks when interacting with the STOR
actor.

Overall, this variant is very simple, but does not offer some features we'd
normally expect from a framework like Roadster. The fruits are hanging low;
achieving event-driven synchronization and better efficiency is easy.

\subsubsection{Event-driven}
This variant avoids the delays introduced by the polling mechanism of the first
variant. Just like with the \gls{CSP}, the initial delta is requested and
communicated via the ROUTER-DEALER pair. As soon as the transmission of the
delta begins, the subnode also starts propagating any new modifications via the
PUSH socket. The supernode collects modifications from all subnodes through its
PULL socket.

To avoid filling the PUSH socket unnecessarily in case the supernode is
temporarily unavailable, the heartbeating done via the UPSTREAM/DOWNSTREAM actors will
be used to recognize such a situation. The continuous propagation of
modifications can be stopped, the sockets reinitialized, making the node ready
for whenever the supernode is back available.

\subsection{Chosen Variant}
No definite choice has been made yet. The client seems to favor the polling variant.
% TODO: choose a persistence synchronization variant

\section{Encryption}\label{sec:approach:encryption}
% TODO: entropy measurement
% TODO: big picture
It is fairly easy to enable transport security to secure the communication
between \zmq sockets over an unsecure network.

\subsection{Assigning socket roles}
Regarding the CURVE security mechanism in \zmq, one of the two involved sockets
needs to be designated as the CURVE server, and the other one as a CURVE
client.\footnote{This is a simple method call on each socket where the involved
keys are passed as well.} This will be done intuitively: A supernode's
DOWNSTREAM actor's sockets will act as CURVE servers, and a subnode's
UPSTREAM actor's sockets will act as CURVE clients. This is not just an
arbitrary choice: A CURVE client socket is given exactly one CURVE server
certificate. A supernode can have multiple subnodes, but the opposite is
impossible.\footnote{HA is a special case, see
\autoref{sec:approach:encryption:ha}} Thus, the socket inside the supernode's
DOWNSTREAM actor will be designated as the CURVE server, and the sockets
inside the subnodes' DOWNSTREAM actors will be designated as the CURVE
clients. An example of the client side is shown in \autoref{lst:auth:fednf}, as
performed by the UPSTREAM actor.

\begin{listing}
	\inputminted[bgcolor=bg]{Ruby}{listings/auth/fednf.rb}
	\caption{Enabling CURVE mechanism on the client.}
	\label{lst:auth:fednf}
\end{listing}

\subsection{Mutual authentication}
% keys on both sides, and which nodes need whose public key
In addition to the server authentication, which is always performed, client
side authentification is desired. This means that each federation node must be
in possession of its respective clients' public keys.

% distribution of public keys
Public keys and the private keys are generated in pairs.  Due to
the nature of public keys, they can be distributed conveniently and safely
through the shared configuration file that defines the federation
topology.\footnote{Curve25519 public keys are only 40 ASCII characters long in
\gls{Z85} notation.}

% distribution of secret keys via SSH
However, distributing the private keys is less convenient, since they must be kept
private. This can be done via \gls{SSH}, logging into each node and generating
the key pair right on the node itself. This way, the private keys would never
even touch the network.

\subsubsection{ZAP}
A socket designated as CURVE server needs a way to authenticate arriving client
connections. This is done via an in-process \gls{ZAP} authentication handler,
as described in \autoref{sec:zmq:security}. That authentication handler
needs to be able to access all clients' public keys to be able to perform its
job.

The ZAP protocol (used between a CURVE server socket and the authentication
handler) is very simple and thus makes it possible to relay the authentication
requests from there to a central ZAP authentication handler, so all clients'
public keys could be stored and managed centrally. But that would violate the
autonomy requirement, so this idea has been discarded immediately.

CZMQ provides an authentication handler\footnote{It is simply a function
(\cpp{zauth()}) that is executed by a CZMQ actor. The whole construct is
abstracted in CZTop through \rb{CZTop::Authenticator}.} that can lookup public
keys awaiting authentication in a designated directory on the file system,
which it does by instantiating a certificate store.\footnote{That certificate
store is a class in CZMQ called \cpp{zcertstore}, represented through
\rb{CZTop::CertStore} in CZTop.} The certificate store in turn looks up the
certificates on disk by the public key in question.

There is also the possibility of providing an already instanciated certificate
store to the authentication handler. This is useful because a certificate store
allows the insertion of certificates (public keys) in-memory, avoiding the need
to load certificates from disk. This greatly simplifies the key distribution.
Public key files will not have to be copied from one node to its subnodes
anymore. Instead, they can be taken directly from the shared configuration file
that defines the federation topology. As an example, \autoref{lst:auth:fedsf}
shows how to start an authentication handler which does client authentication
as it is done inside the DOWNSTREAM actor.

\begin{listing}
	\inputminted[bgcolor=bg]{Ruby}{listings/auth/fedsf.rb}
	\caption{Enabling CURVE mechanism on the server to perform client authentication.}
	\label{lst:auth:fedsf}
\end{listing}

\paragraph{Failover}\label{sec:approach:encryption:ha}
It is possible that a node's supernode is actually a HA cluster consisting of
two nodes. Those two nodes should have their own key pairs for additional
security. This imposes a problem for the subnodes: It is only possible to assign
a single CURVE server certificate (public key). In case the active HA peer becomes
unresponsive and the subnode(s) falls back to the passive HA peer, a new CURVE
server certificate needs to be assigned to the client sockets. This should be
possible. But there's a better solution:
Since the socket could be \cite[Binary Star Implementation, Binary Star client
in C]{zmq:zguide} ``confused''\footnote{This probably applies only to REQ
sockets, since they implement \cite[Client-Side Reliability (Lazy Pirate
Pattern)]{zmq:zguide} ``a small finite-state machine to enforce the
send/receive ping-pong''}, it is good practice to simply destroy the existing
socket and create a fresh one, assigning the correct supernode's CURVE server
certificate to it.

\paragraph{IP based access control}
In addition to the CURVE mechanism, blacklisting or whitelisting clients based
on their \gls{IP} addresses is possible and supported by
\rb{CZTop::Authenticator} through the PLAIN mechanism. It would even be
feasible to whitelist the exact client IP addresses, since they can be deduced
from the shared configuration. But it is not part of requirements and thus will not
be done within the scope of this bachelor thesis.

\subsection{Key generation and distribution procedure}
The following procedure can be followed to generate the required keys in
advance and make them available through the shared configuration file:

\begin{enumerate}
	\item for all nodes
	\begin{enumerate}
		\item generate key pair on the node itself and save it on the
			node
		\item remember the public key
	\end{enumerate}

	\item put all public keys (in Z85 notation) to their respective section
		in the shared config file defining the federation topology
\end{enumerate}

This procedure could be implemented as a script that performs the procedure
over \gls{SSH} while the nodes are still in the
lab (assuming that is secure). Of course that script could also be used when
the nodes are already installed in a production environment. However, for that to be
secure, the nodes' authenticities must have been established in advance, or
man-in-the-middle attacks would be possible while logging into the nodes,
compromising everything.

% TODO: `roadster keygen CONFIG` util

% TODO 256 bit encryption
% TODO 16 byte authenticator
% TODO 24 byte nonce

% TODO Entropy without encryption:
% ```
% Entropy = 5.390141 bits per byte.
% 
% Optimum compression would reduce the size
% of this 4518 byte file by 32 percent.
% 
% Chi square distribution for 4518 samples is 36298.08, and randomly
% would exceed this value less than 0.01 percent of the times.
% 
% Arithmetic mean value of data bytes is 83.2517 (127.5 = random).
% Monte Carlo value for Pi is 4.000000000 (error 27.32 percent).
% Serial correlation coefficient is 0.310227 (totally uncorrelated = 0.0).
% ```


\subsection{End-to-end security}
Encrypting and authenticating network traffic is good, but it only secures the
communication between two neighboring nodes. Messages could travel via multiple
hops to reach their destination node. To provide end-to-end security for such
messages, they need to be authenticated as well, using cryptographic
signatures.

The challenge here is to decide on what to sign. Ruby objects are serialized
into \glspl{BLOB}, which is perfectly suitable as input for a \gls{MAC} algorithm.
This is possible because the message's on-wire
representation doesn't change from origin to destination.

The codecs implemented in Roadster, which serialize a message into the an
on-wire representation, are part of the messaging layer. This fact makes it
hard to adapt the existing codecs to add authentication. The codecs should be
the final authority before the message is sent via a socket, and should be the
first to inspect an arriving message. The advantage is that a forged,
unauthentic message can (and must) be discarded immediately, without even
decoding it.  Because Ruby marshalling allows the serialization and
deserialization of almost any object, merely decoding a message could be
dangerous, as described by the Ruby documentation itself on \cite[Security
considerations]{rb:doc:marshal}.  For this reason, the codecs need to be moved
from the messaging layer down to the reactor layer.


\subsubsection{Implementation}
The codecs have been adapted to sign \emph{tag} and authenticate messages using
an authenticator object, an instance of the
new class\\\sh{Roadster::Actors::MessageAuthenticator}. It offers the two
instance methods \mintinline[bgcolor=bg]{Text};#tag; and \mintinline[bgcolor=bg]{Text};#authenticate!;.
Codecs --- which are stateless, only offering \sh{.encode} and \sh{.decode}
class methods --- are passed the authenticator object.

The resulting MAC \emph{security tag} is stored in an additional \zmq frame
before the message BLOB itself. Thus, the codec version (sent in the first
frame of every message) has been updated to represent the incompatible change.
It is important that the new frame is only used as a MAC and not any other
information, as the authenticity of any other information cannot be verified.

Intra-node messages do not have to be authenticated, whereas inter-node
messages have to be. This requires some logic to be present in the codecs
to only use the authenticator on inter-node sockets. To reduce
code complexity, and also because codecs have no knowledge about sockets, the
\emph{Null Object} design pattern has been used. When a socket is passed a
message to be sent, or when receiving a message, the appropriate authenticator
object is passed, which is either an instance of
\sh{Roadster::Actors::MessageAuthenticator} for inter-node sockets, or
\sh{Roadster::Actors::MessageAuthenticator::Null} for intra-node sockets.


\section{OPC UA Interface: High availability}\label{sec:approach:opc-ua}
Some client systems interact with Roadster over the standardized \gls{opc-ua}
interface. According to the standard in \cite[6.4.2 Server Redundancy,
p.~94]{opc-ua:behavior:server-redundancy}, there are multiple ways to ensure
interoperability when running the root node as a \gls{HA} cluster. The variant
which comes closest to the \gls{HA} mechanism favored in the original task description is
\emph{Non-transparent server Redundancy}, where the client system is aware of
the hot-standby setup. Specifically, it also has to be aware of the failover
mode. Described failover modes in the standard are \emph{Cold}, \emph{Warm},
\emph{Hot} and \emph{HotPlusMirrored}. The mode \emph{Warm} describes the
favored failover mode best, quoting \cite[6.4.2.4.4 Server Failover Modes,
p.~98]{opc-ua:behavior:server-redundancy}:


\begin{quote}
``Warm Failover mode is where the backup Server(s) can be active, but cannot
connect to actual data points (typically, a system where the underlying devices
are limited to a single connection). Underlying devices, such as PLCs, may have
limited resources that permit a single Server connection. Therefore, only a
single Server will be able to consume data. The ServiceLevel Variable defined
in Part 5 indicates the ability of the Server to provide its data to the
Client.''
\end{quote}

The concept to induce a failover in a \gls{HA} cluster (limited to a
Roadster federation only, excluding client systems) as part of the
COMM.UPSTREAM actor's responsibilities can actually be implemented in any other
actor, e.g. an adapter facing client systems. All the adapter on the passive
node has to do is send a vote to the COMM.BSTAR actor in case the client system
tries to connect and request a service. The following code would be sufficient:

\begin{listing}[H]
  \begin{minted}[bgcolor=bg]{Ruby}
msg = Messaging::BSP::Messages::Vote.new 'comm.opcua', 'comm.bstar'
route_message msg
  \end{minted}
  \caption{OPC-UA adapter: How to send a vote to the BSTAR actor.}
  \label{lst:approach:opc-ua}
\end{listing}

mindclue GmbH actually offers an adapter implementing the \gls{opc-ua}
interface, wrapped into a Ruby software package. Unfortunately, an
incompatibility in the package's dependencies, makes its use impossible at the
moment. Instead, an implementation of a simple Telnet server is planned. The
server shall recognize the command \sh{VOTE!}, and perform the required steps to
relay the vote to the COMM.BSTAR actor.

% vim: ft=tex
\chapter{Results}\label{ch:res}
This chapter documents the final results.

\section{System tests}
All functional, required features have been implemented, as shown in \autoref{tab:systemtestresults}.
\begin{table}[H]
  \centering
  \begin{tabular}{|m{5mm}|m{5mm}|m{5mm}|m{50mm}|}
    \hline
    \bf C1 & \bf C2 & \bf C3 & \bf Feature \\
    \hline
	  \bf \color{green!65!black}\cmark & \color{green!65!black}\cmark & \color{green!65!black}\cmark & Federation \\
    %\hline
    \bf \color{green!65!black}\cmark & \color{green!65!black}\cmark & \color{green!65!black}\cmark & DIM extension \\
    %\hline
    \bf \color{green!65!black}\cmark & \color{green!65!black}\cmark & \color{green!65!black}\cmark & Autonomy \\
    %\hline
    \bf \color{green!65!black}\cmark & \color{green!65!black}\cmark & \color{green!65!black}\cmark & Message routing \\
    %\hline
    \bf \color{red!50}\xmark & \color{green!65!black}\cmark & \color{green!65!black}\cmark & High Availability \\
    %\hline
    \bf \color{red!50}\xmark & \color{green!65!black}\cmark & \color{green!65!black}\cmark & Persistence synchronization \\
    %\hline
    \bf \color{red!50}\xmark & \color{red!50}\xmark & \color{green!65!black}\cmark & Encryption (optional) \\
    %\hline
    \bf \color{red!50}\xmark & \color{red!50}\xmark & \color{green!65!black}\cmark & OPC-UA (optional) \\
    \hline
  \end{tabular} \\
  \caption{Systen tests results}
  \label{tab:systemtestresults}
\end{table}

\section{Unit tests}
All new contributions are 100\% covered by unit tests.

\section{Integration tests}
Existing integration tests have been refactored.

\section{Non-functional requirements}
All non-functional requirements, including the Ruby style desired by mindclue
GmbH, have been achieved.


\section{Programming statistics}

\autoref{tab:programmingstatistics} shows the programming statistics.
\begin{table}[H]
  \centering
  \begin{tabular}{|m{50mm}|m{30mm}|}
   \hline
	\bf Ruby LoC before & 7161 \\
	\hline
	\bf Ruby LoC after & 9123 \\
	\hline
	\bf Ruby LoC & 1962 \\
	\hline
	\bf Ruby Classes before BA & 187 \\
	\hline
	\bf Ruby Classes after BA & 223 \\
	\hline
	\bf New Ruby Comments & 1468 \\
	\hline
	\bf New Ruby files & 23 \\
	\hline
	\bf Python LoC & 684 \\
	\hline
	\bf New Python files & 14 \\
	\hline
	\bf Number of commits in 
		\newline roadster repository & 579 \\
	\hline
	\bf Number of commits in 
		\newline ba-roadster-app repository & 26 \\
	\hline
	\bf Number of line changes 
		\newline roadster repository & 18749{\color{green!70} ++} / 8063{\color{red!70} -{}-} \\
	\hline
	\bf Number of line changes 
		\newline ba-roadster-app repository & 61{\color{green!70} ++} / 364{\color{red!70} -{}-} \\
	\hline
	\bf Number of new branches & 14 \\
	\hline
	\bf Number of new unit-tests & 803 \\
	\hline
	\bf Number of features & 7 \\
	\hline
	\bf Number of scenarios & 15 \\
	\hline
	\bf Number of system-tests & 104 \\
    \hline
  \end{tabular} \\
  \caption{Programming statistics}
  \label{tab:programmingstatistics}
\end{table}
%TODO cztop, federation, federation_csp, message_routing, ping_pong, webui_csp_fix, case_handling_fix, model_ownership, profiling, optimizations, msgpack_codec, node_as_root, high_availability, encryption
\subsection{Metrics}
% TODO  * number of new classes
% TODO  * flog/reek/rubocop/flay/metric_fu... metrics
% TODO short description ...
% ABC metric Assignments, Branches, Conditions. An otherway for countig something like LoC. Lower values means the code is short and easily to read.
The follow listing shows the calculated ABC metric values. Higher means more complex code. It is an modern approach to LoC or Function Point.
\begin{listing}[H]
	\begin{minted}[bgcolor=bg]{text}
mschuler@mschuler-vb:/tmp/roadster/lib$ flay  -s .
Total score (lower is better) = 2162

  225.00: ./roadster/messaging/handlers/webui.rb
  172.00: ./roadster/messaging/handlers/core.rb
   95.00: ./roadster/messaging/handlers/bstar.rb
   90.00: ./roadster/engines/modules/report/eventjournal.html.erb
   88.00: ./roadster/messaging/handlers/storage.rb
   85.00: ./roadster/messaging/handlers/downstream.rb
   77.00: ./roadster/messaging/handlers/upstream.rb
   77.00: ./roadster/messaging/handlers/comm.rb
   72.00: ./roadster/adapters/protocols/iec104/info_objects/point_information.rb
   72.00: ./roadster/engines/core.rb
   70.00: ./roadster/messaging/protocols/csp/api.rb
   68.00: ./roadster/messaging/messages/base.rb
   64.00: ./roadster/engines/modules/database/tc_eventjournal_db.rb
   57.00: ./roadster/adapters/protocols/modbus/pdu/write_multiple_registers.rb
   57.00: ./roadster/adapters/protocols/modbus/pdu/read_holding_registers.rb
   57.00: ./roadster/actors/downstream.rb
   56.00: ./roadster/engines/modules/domain/host/host.rb
   54.00: ./roadster/messaging/protocols/pcp/api.rb
   54.00: ./roadster/engines/modules/domain/model/case.rb
   52.00: ./roadster/engines/comm.rb
   51.00: ./roadster/messaging/handlers/logging.rb
   38.00: ./roadster/messaging/protocols/usp/api.rb
   38.00: ./roadster/messaging/protocols/smp/api.rb
   38.00: ./roadster/actors/upstream.rb
   36.00: ./roadster/engines/modules/database/tc_base_db.rb
   35.00: ./roadster/actors/core.rb
   35.00: ./roadster/actors/storage.rb
   32.00: ./roadster/messaging/protocols/pdp/messages.rb
   32.00: ./roadster/messaging/protocols/smp/messages.rb
   32.00: ./roadster/messaging/protocols/pdp/api_timeseries.rb
   22.00: ./roadster/adapters/protocols/iec104/info_objects/info_object_70.rb
   22.00: ./roadster/adapters/protocols/iec104/info_objects/info_object_100.rb
   20.00: ./roadster/engines/webui.rb
   19.00: ./roadster/messaging/protocols/pdp/api_parameters.rb
   19.00: ./roadster/messaging/protocols/usp/messages.rb
   19.00: ./roadster/messaging/protocols/pcp/messages.rb
   16.00: ./roadster/messaging/protocols/cmp/messages.rb
   16.00: ./roadster/actors/comm.rb
	\end{minted}
	\caption{flay result}
	\label{lst:metrics:flay:result}
\end{listing}

\subsection{Collaboration statistics}
\autoref{tab:collaborationstatistics} shows the collaboration statistics.
\begin{table}[H]
  \centering
  \begin{tabular}{|m{60mm}|m{10mm}|}
	\hline
	\bf Slack messages with the projectpartner & 3700 \\
	\hline
	\bf Slack messages with the client & 1300 \\
	\hline
	\bf Median time per issue  & 8.68 hours \\
	\hline
	\bf Median time per task & 3.55 hours \\
    \hline
  \end{tabular} \\
  \caption{Collaboration statistics}
  \label{tab:collaborationstatistics}
\end{table}

% TODO  * test examples
% TODO test results (red/green)

% vim: ft=tex
\chapter{Discussion}
This chapter analyzes the outcome of this bachelor thesis in review.
Value added is documented in \autoref{sec:disc:value-added}. Limitations are
discussed in \autoref{sec:disc:lim}.

\section{General}
General aspects are discussed in this section.

\subsection{Value Added}\label{sec:disc:value-added}
Actual value added for mindclue GmbH is a much more capable Roadster framework.
The capability to run a Roadster application on multiple, inter-connected nodes
can make considerable distinction on the market, since Roadster can now itself
be run as a higher-level system (see \autoref{sec:scope:sys-integration})
aggregating information from lower-level
systems.

The autonomy of a Roadster nodes is a valuable trait of this solution. State
exchange is graceful, meaning it tolerates temporary network partitions.

Network communication between Roadster nodes is secure. Traffic is not only
encrypted, but also authenticated using modern, patent-free, high-security,
high-speed cryptography \cite{zmq:curvezmq}.

The existing test suites of Roadster have been greatly extended. The isolation
of existing unit tests has been increased by leveraging more of RSpec's
features.
The new container-based system test infrastructure can be leveraged for
close-to-reality testing of new features before they are deployed to production
systems.

Maintainability has been kept the same or even improved. Certain duplicated
code (like the handling of model updates in the CORE and COMM engines) has been extracted to a
common mix-in. Certain closely-coupled methods (like the ones in
\sh{Roadster::Domain::Host} concerning model updates) have been extracted to a
new highly cohesive class. The suboptimally placed codecs have
been moved closer to where message serialization and deserialization is used,
which allowed for end-to-end message authentication.

Performance has been maintained. In certain cases where both regular
expressions and basic string functions could be used, benchmarks have been
performed and the faster solution has been chosen for the implementation. An
example is shown in \autoref{lst:perf:string}.
\begin{listing}[]
  \begin{minted}[bgcolor=bg]{Ruby}
# @return [String] the receiver node
def receiver_node
  # NOTE: This is the same as the line below, but faster.
  i = @receiver.rindex('@') and @receiver[ i+1, @receiver.length-i-1 ]
  # @receiver[ /@(.*)/, 1 ]
end
  \end{minted}
  \caption{Example of an optimized String handling method.}
  \label{lst:perf:string}
\end{listing}


%   * test suite with profiling activated


\subsection{Limitations}\label{sec:disc:lim}
\paragraph{Static topology}
A Roadster federation topology is static. It cannot be dynamically updated at
this point. Adding new nodes to an existing federation means having to restart
the other nodes after updating their configuration.

\paragraph{High availability}
A HA node currently cannot have any kind of supernode, which is considered an
uncommon topology, but still is a limitation. More about this in \autoref{sec:disc:ha}.


\paragraph{Persistence replication}
Message traffic towards root node sums up because of persistence
replication. This should not be a problem because of \zmq's brilliant
message batching, so the real limit only given by the inter-node network links.


\subsection{Ideas}
\subsubsection{Message serialization performance}
MessagePack\footnote{\url{http://msgpack.org}} could be used instead of Ruby
Marshalling to serialize messages. It is known to be very fast, very small, and
would increase interoperability because MessagePack is available for in all
common programming languages. Furthermore, security would be improved too,
because MessagePack does not allow the serialization of arbitrary Ruby objects.
Benchmarking showed that using MessagePack would allow a considerable
performance increase, as shown in \autoref{lst:bm:msgpack}.
\begin{listing}[]
  \begin{minted}[bgcolor=bg]{Text}
#####  ENCODING  #####
Warming up --------------------------------------
                RUBY     8.109k i/100ms
             MSGPACK    12.742k i/100ms
Calculating -------------------------------------
                RUBY     84.962k (± 3.5%) i/s -    429.777k in   5.064866s
             MSGPACK    137.498k (± 3.7%) i/s -    688.068k in   5.011320s

Comparison:
             MSGPACK:   137497.8 i/s
                RUBY:    84962.0 i/s - 1.62x  slower



#####  DECODING  #####
Warming up --------------------------------------
                RUBY     9.612k i/100ms
             MSGPACK    12.665k i/100ms
Calculating -------------------------------------
                RUBY    102.331k (± 4.5%) i/s -    519.048k in   5.082607s
             MSGPACK    130.003k (± 5.1%) i/s -    658.580k in   5.079560s

Comparison:
             MSGPACK:   130002.5 i/s
                RUBY:   102330.9 i/s - 1.27x  slower
  \end{minted}
  \caption{Benchmark comparison between Ruby Marshalling and MsgPack serialization.}
  \label{lst:bm:msgpack}
\end{listing}

\subsubsection{Miscellaneous}
Here is an Unordered list of ideas:
\begin{itemize}
	\item Switch to Moneta\footnote{\url{https://github.com/minad/moneta}} for a unified key-value store interface in Ruby, then eventually away from \gls{tc} to something more modern and maintained, like LMDB\footnote{\url{http://lmdb.tech/doc/}}, which is super fast and crash-proof.
	\item TIPC: High performance cluster communication protocol, suitable because Roadster nodes are Linux and there are direct links to peers (required for TIPC)
	\item Key management in a DB (instead of files), with GUI to accept new clients
	\item Dynamic node topology (maybe via DSL-file in Etcd, or DIM-only, or Zookeeper)
	\item Fast compression for messages, like LZ4\footnote{\url{https://github.com/lz4/lz4}} or Snappy\footnote{\url{https://github.com/google/snappy}}, which could be interesting for Roadster nodes connected via a cellular data network
\end{itemize}



% -------------------------------------------------------------------------
\section{By feature}

\subsection{Messaging}
\subsubsection{Broadcast messages}
Broadcast messages could be useful, especially when it comes to inter-node CSP.
Instead of having the protocol (or even the engine, as of now) decide where to
relay a domain model update to ensure a consistent state across all actors and
nodes, this kind of broadcast messaging could be offered by Roadster's
messaging routing itself. However, care would have to be taken to avoid
accidental message storms.

Message routing would have been a lot easier to implement if CORE had inter-node sockets itself, but:
\begin{itemize}
	\item Separation of concerns would suffer
	\item The CORE actor could become a big monolithic class, if done wrong
	\item Maintainability would decrease
	\item Unix philosophy would be violated: ``Do one thing, and do it well.''
\end{itemize}

\subsubsection{Inter-node CSP}
Eventual consistency is guaranteed even in situations where multiple actors
modify the same DIM objects at roughly the same time. However, some updates
might get lost. In those cases, the last writer wins.

Subtree synchronization has not been implemented, as deemed as not necessary at
this moment. Replication performance is by far sufficient, with messages being
sent and received within 1 -- 3 milliseconds, and the space requirements of a
complete DIM was observed to be very small so far, being just a small number of
plain Ruby objects.

\subsection{DIM}
\paragraph{Visibility}
A feature to define the visibility on a domain model object basis could be
useful when information needs to be shared across actors, but not across nodes.
A new visibility property could accept the values \rb{:federation}, \rb{:node}, and \rb{:actor} to define the desired visibility.
One case where this feature would have been useful is the current choice of the
supernode HA peer currently communicated with. In the current version, this is
simply an instance variable in the COMM.UPSTREAM engine.

\paragraph{Object level signatures}
In certain situations, such as when requesting a snapshot of the domain model,
foreign objects could be sent by a neighboring node that is not actually the owner
of said objects. This means that it is impossible to prove that those objects
actually originated from their owning node. A mechanism implemented so as to
simply skip the end-to-end authenticity checks when applying snapshots would be
exploitable. A possible solution would be object level signatures, stored as a
property within the property itself. Any node who receives a domain model
update, whether direct (updates) or "replayed" (snapshots) could verify the
signature of each object update. This would require a canonicalized
representation of the object's (signed) contents to use as input for the
signing algorithm, similar to the W3C recommendation for XML
Canonicalization.\footnote{\url{https://www.w3.org/TR/xml-exc-c14n/}}

\paragraph{CSP refactoring}
After having worked a considerable amount of time on \gls{CSP} related
features, the impression is that it could be refactored. Initially described in
an issue\footnote{\url{https://github.com/thewesch/ba-roadster-doc/issues/80}}
on GitHub, it seems that too much CSP-related logic is implemented in the
engine layer. The students' opinion is that the new module
\sh{Roadster::Engines::CSPMethods} would not be needed if more CSP-related
functionality would be implemented directly in the protocol APIs within the
messaging layer.

\subsection{High availability}\label{sec:disc:ha}
A HA cluster has to be complete (both nodes running) during initialization.
Otherwise, only primary node can start to serve requests; the backup node will not
do this.


A HA cluster currently cannot have any kind of supernode right now, because
COMM.UPSTREAM's DEALER socket cannot decide which node receives a message sent.
This should not be a significant limitation as of right now, but in case a
future version should ever have this capability, there are two solutions:

\begin{enumerate}
	\item ROUTER-ROUTER communication between supernode and subnodes.
	\item Refactor
\end{enumerate}

Proposal for the refactoring solution:
\begin{itemize}
\item Rename COMM.BSTAR into COMM.SIDESTREAM
\item Publish domain model updates over the "horizontal" PUB socket as well (just like from COMM.DOWNSTREAM does)
\item Add DEALER socket (DEALER-DEALER communication works)
\item PUSH/PULL socket is not needed because passive does not perform any modifications to DIM
\item The primary always \sh{bind()}s, the backup node always \sh{connect()}s
\end{itemize}

Benefits would include:
\begin{itemize}
\item The actors COMM.UPSTREAM and COMM.DOWNSTREAM are back to a single responsibility only.
\item The COMM.UPSTREAM actor does not even have to be started on HA nodes.
\item The COMM.DOWNSTREAM actor only needs to be started with subnodes actually present.
\item The COMM.DOWNSTREAM actor would never need to discard and re-register its inter-node sockets because the set of allowed subnodes (public keys) would be constant.
\end{itemize}

\subsubsection{Binary Star Extension}
A possible optimization to the \gls{bstar} is to go active with enough votes,
even if the currently active HA peer is still responsive. This would help in
the situation where only the active HA peer is disconnected from subnodes due
to a network partition.

\subsubsection{Rolling upgrades}
An obvious side benefit of the high availability feature comes into play with
upgrades. Using the HA cluster feature it is possible to perform upgrades
without downtime, sometimes referred to as \emph{rolling upgrades}.  One HA
peer can be upgraded while the other one stays in service. After a successful
upgrade, Roadster on the active HA peer can be stopped, upgraded, and restarted
as the passive HA peer.


\subsubsection{Thoughtful heartbeating}\label{sec:discussion:imp:ha:hb}
The COMM.BSTAR actor currently happily exchanges heartbeats with the COMM.BSTAR actor
running on the other HA peer, even in case all actors but CORE are dead. The
CORE actor's health is periodically checked using the ping/pong mechanism as
described in \autoref{sec:approach:ha:hb}.

In a subsequent version of Roadster, the CORE actor could relay those Ping
requests to the other actors and only respond with a Pong in case the other
actors all have responded with Pong. This mechanism would make the node only
signal life signs in case all actors are healthy.

\subsubsection{HA within a node}
Building upon the previous idea, unhealthy actors could be killed and respawned
by the CORE actor.  Microrebooting unhealthy components without an attempt at
any sophisticated recovery would make Roadster belong to the \emph{crash-only}
kind of fault-tolerant software.

\subsubsection{Manual failover}\label{sec:discussion:ha:manual-failover}
A manually induced failover could be useful in some cases, e.g. when a
component fails that does not provoke an automatic failover.

Providing a less brutal way of inducing a failover than to simply shut down the
currently active HA peer could be interesting in a future version.



\subsection{Security}

\paragraph{Traffic analysis}
CURVE \zmq encrypts and authenticates network traffic, but traffic analysis
attacks are still possible. To mitigate this, messages would have to be padded
with random amounts of data, and spurious dummy messages would have to be
passed around.

\paragraph{Connection lifetime}
In addition to that, no more than $2^{64} - 1$ ZMTP commands (roughly equaling to
messages) must be sent per connection. After that, a wraparound of the nonce
could happen. Granted, this is a rather astronomic limitation.

\paragraph{Signed configurations}
During the interim presentation, the suggestion of signing the Roadster configuration files has come up.
Unfortunately this would not increase security, as an attacker who has already
gained access to the system could simply change Roadster's (interpreted) source
code to disable any authenticity checks of said configuration files.


\section{Miscellaneous}
In case a Roadster node ever has to calculate a huge amount of data, this
process could be outsourced to other nodes having a small load. As described in
\cite{ieee:data-mining}, this could be useful for data mining applications,
leveraging the execution power of many actors.


% --------------------------------------------------------------------
\section{Review of additional goals}
The additional goals mentioned in \autoref{sec:scope:add-goals} have been kept
in mind and achieved. Namely:
\begin{itemize}
	\item Security concerns of SCADA applications
	\item Fallacies of distributed computing
\end{itemize}

\subsection{Security concerns of SCADA applications}
As documented and discussed before, the security of Roadster as a SCADA
application is guaranteed even across insecure networks. Security through
obscurity has not been practiced. No matter if the network of a particular
Roadster installation is or is not connected to the internet, security of
inter-node sockets is ensured using modern cryptography.

Thus, this additional goal has been achieved.

\subsection{Fallacies of distributed computing}
Roadster's federation feature has been developed with the well-known fallacies
of distributed computing in mind. The following list briefly describes how each
fallacy is dealt with:

\begin{description}
	\item [The network is reliable.] \hfill\\
		Handled by \zmq. All transport details are completely
		abstracted away. Intricacies of network protocols are handled
		gracefully by \zmq.

	\item [Latency is zero.] \hfill\\
		Latency does literally not matter in the actor model. Messages
		sent to an actor are queued by \zmq if necessary, and
		then processed by the actor one by one. No assumptions on the
		order or actual timings of message delivery have been made.

	\item [Bandwidth is infinite.] \hfill\\
		Handled by \zmq's brilliant batching mechanism as described in \autoref{ch:zmq}.
		Messages of the \gls{RMP} are generally pretty small at this
		point. Using MessagePack and/or a compression library would
		help if message sizes ever cause bottlenecks.

	\item [The network is secure.] \hfill\\
		Handled by \zmq's CURVE security mechanism which uses strong
		cryptography, as described earlier in this chapter, as well as
		in \autoref{sec:approach:encryption}.

	\item [Topology does not change.] \hfill\\
		Roadster's topology is static through its configuration. In
		case the network topology changes and endpoints need to be
		adapted, this must be done in the configuration, followed by a
		restart of the Roadster instance.

		Temporary network partitions caused by changes in the network
		topology are handled gracefully by \zmq and Roadster's autonomy
		feature.

	\item [There is one administrator.] \hfill\\
		This is outside the scope of this bachelor thesis, as
		particular deployments were not the concern. When deploying
		Roadster in a new network, it is advised to determine a single
		administrator responsible for the network to avoid unnecessary
		overhead.

	\item [Transport cost is zero.] \hfill\\
		Again, latency does not matter. More computation and space
		efficient serialization techniques are available if necessary.
		Expensive network equipment to increase reliability or security
		of Roadster applications is not needed.

	\item [The network is homogeneous.] \hfill\\
		This is handled by \zmq to some degree, as \zmq is available on
		many platforms and all common programming languages. To further
		improve interoperability when needed, good replacements for
		Ruby Marshalling are available. As an extreme example,
		different Roadster actors could then be written in different
		programming languages in the future. Protocol incompatibilities
		are handled brutally in Roadster: If a received message
		contains a wrong protocol version, it is immediately discarded.
\end{description}

\chapter{Conclusion}
TODO write conclusion, we're the best and everything is awesome
TODO they should teach the actor model in APF, because ...

\printbibliography
\printglossaries

%-----------------------------------------------------------------------------
\appendix
\part{Appendix}
\chapter{Self Reflection}
% TODO how did we perform, completion of goals, accuracy of estimated efforts, efficiency, resourcefulness
% TODO: learned a lot about the usefulness abstraction layers


\chapter{Development infrastructure}
\section{Personal development machines}
P. Wenger's personal development machine for Roadster was an Ubuntu 16.04 VM
running in VirtualBox on macOS. Most of the work took place within a fullscreen terminal
emulator (iTerm2\footnote{\url{https://iterm2.com}}) connected to the VM via SSH. Tmux was used within that to
persist the open workspace (multiple running shells and panes) across
suspensions of the VM.

Tmux windows used:
\begin{enumerate}
	\item NeoVim\footnote{\url{https://neovim.io}} with tabs and windows for development of features and specs.
	\item Automatic execution of specs whenever a file (implementation or
		specs) changes. This is done using
		Guard.\footnote{\url{https://github.com/guard/guard}}
	\item Manually started Roadster instances (usually two in side-by-side Tmux panes).
\end{enumerate}

The VM was not used to write this document. Instead, an instance of NeoVim and
a PDF viewer was used directly on the host OS. The repository's Makefile helped
automating the build process.

M. Schuler used RubyMine\footnote{\url{https://www.jetbrains.com/ruby}} on Windows for development and an Ubuntu VM to run
Roadster and the new system tests. To do so, the source code directory being
modified by the IDE was mounted inside the VM.

\section{Tools}
The source code of this document and all of our code contributions are hosted
on GitHub. The students will organize and perform their work directly on the
site as far as possible. This means creating a Project board for each of the
development phases, creating, assigning, and closing issues, as well as using
the Wiki feature to plan and document meetings with the professor and the
client. Time tracking is done externally on Everhour. See
\autoref{tab:dev-tools}.


\begin{table}[H]
  \centering
  \begin{tabular}{|p{50mm}|p{35mm}|p{35mm}|}
    \hline
    \bf Use & \bf Name & \bf Version \\ \hline
    \bf Editor & NeoVim & v0.2.0-150-g33319b1b \\ \hline
    \bf IDE & RubyMine & 2016.3 \\ \hline
    \bf Version control system & Git & 2.* \\ \hline
    \bf Project management  & GitHub, Everhour &  \\ \hline
  \end{tabular} \\
  \caption{Development tools}
  \label{tab:dev-tools}
\end{table}


\subsection{Infrastructure}
\autoref{tab:infrastructure} shows the infrastructure provided by HSR. It was mainly used for \gls{CI}.

\begin{table}[H]
  \centering
  \begin{tabular}{|p{25mm}|p{50mm}|}
    \hline
    \bf Server name & sinv-56092.edu.hsr.ch \\ \hline
    \bf IP address & 152.96.56.92 \\ \hline
    \bf OS & Ubuntu 14.04 LTS \\ \hline
    \bf End of life & March \nth{3} 2017 \\ \hline
    \bf CPU & 1 @ 2.2 GHz \\ \hline
    \bf RAM & 1 GiB \\ \hline
    \bf Disk space & 15 GiB \\ \hline
  \end{tabular} \\
  \caption{Technical specifications of VM provided by HSR}
  \label{tab:infrastructure}
\end{table}


\chapter{Task description}\label{ch:task-desc}
The following five pages are the original task description, signed by Prof. Dr.
F. Mehta. Due to administrative changes in the submission process, no optical
media is submitted. Instead, the digital artifacts are submitted online via
\url{https://moodle.hsr.ch}.

\includepdf[pages={-}]{signed_task_description.pdf}


\chapter{License}
As stated in the task description, all of our code contributions underlie the
ISC license, which is functionally equivalent to the MIT license and the
Simplified BSD license, but uses simpler language. In addition to that, we
hereby explicitly grant mindclue GmbH unrestricted usage of all our code
contributions.


% vim: ft=tex
\chapter{Project monitoring}
\section{Performance analysis}
\subsection{Weekly overview}

\autoref{fig:projmon:weekly} illustrates the weekly time spent on this bachelor
thesis. The thick horizontal line represents the planned effort of 28.5 hours a
week, which has been calculated using the risk analysis explained
in~\autoref{sec:projplan:est-time}.  Milestones are denoted by the green
labels. The legend along the x-axis represents the semester weeks, including
the particular RUP phases.


\begin{figure}[]
	\includegraphics[trim=2cm 18.3cm 4.6cm 2.8cm, clip=true, width=\textwidth]{img/project_monitoring_weekly_diagram.pdf}
	\caption{weekly hours}
	\label{fig:weekly:hours}
\end{figure}

% TODO Translate
Wie man aus dieser Grafik herauslesen kann, haben wir zu Beginn der Arbeit 
am wenigsten Stunden pro Woche aufgewendet. Das liegt daran, dass wir zuerst die 
genauen Anforderungen abklären mussten und teilweise erst nach einer Besprechung 
oder einer Rückmeldung an bestimmten Punkten weiterarbeiten konnten. 
Vor dem Meilenstein „Release“ hatten wir alle Informationen zusammen und 
konnten mit vollem Elan die Anforderungen umsetzen.


\begin{table}[H]
  \centering
  \begin{tabular}{|p{100mm}|p{35mm}|}
    \hline 	\bf Planned HSR hours & 720h \\ \hline
	\bf Estimated hours & 816h \\ \hline
	\bf P.Wenger worked hours & 413h \\ \hline
	\bf M.Schuler worked hours & 403h \\ \hline
	\bf Everhour tags & 14 \\ \hline
	\bf Github issues & 94 \\ \hline
	\bf Github tasks & 230 \\ \hline
	\bf Biggest issue \#33 Refine Document & 80h \\ \hline
  \end{tabular} \\
  \caption{project stats}
  \label{tab:projectstats}
\end{table}


% TODO translate
Die von der Modulbeschreibung geforderten, Soll-Stunden wurden um 10% überschritten haben. 
Der Grund der Überschreitung ist unteranderem
die Implementation der optionalen Features und die grossflächig angelegten Tests.
Die einzelnen Personen haben ein sehr ausgeglichenes Stundentotal. 
Das liegt daran, dass wir in regelmässigen Abständen eine Auswertung 
in Everhour gemacht haben und dadurch die Tasks gerechter verteilen konnten. 

\begin{figure}[]
	\includegraphics[trim=4cm 18.8cm 3.5cm 2.8cm, clip=true, width=\textwidth]{img/project_monitoring_diagrams.pdf}
	\caption{hours per tag}
	\label{fig:hours:per:tag}
\end{figure}


\begin{figure}[]
	\includegraphics[trim=2cm 12cm 3.5cm 11.5cm, clip=true, width=\textwidth]{img/project_monitoring_diagrams.pdf}
	\caption{hours per feature}
	\label{fig:hours:per:feature}
\end{figure}

% TODO translate
Die ersten Feature brauchten knapp mehr Zeit als geplant. Es wurde schon früh während der Prototyp Phase
gemerkt das es mehr Zeit braucht als geplant in die Einarbeitung in Roadster. Das fehlende Wissen und
die "Perfektionswut" waren unteranderem Grund dafür.
Im weiteren Projektverlauf konnte die Performance der einzelnen Projektmitarbeiter gesteigert werden
was auf das erlernte Wissen über Roadster zurückzuführen war, sowie durch die gut
Strukturiere Architektur von Roadster was uns das Erweitern der Software sichtlich erleichtert hat.


% TODO translate
Die Archivdaten unseres Everhour Projekts befinden sich auf dem Moodle. 
In diesem kann detailliert nachverfolgt werden, 
für welche Tasks wir wie viel Zeit aufgewendet haben.


\subsection{Aufgewendete Stunden nach Feature}
% TODO Translate
Da wir in dieser Arbeit ein bestehendes Produkt erweitert haben war es zu beginn an
schwierig die Aufwände abzuschätzen. Die Soll/Ist Abschätzungen wurden von Iteration zu Iteration
immer genauer. Nachfolgend die Soll/Ist Aufwandschätzung der einzelnen Feature.

% TODO checklist item: each up to 4h

% TODO Fillup the right data
\begin{tikzpicture}
\begin{axis}[
	title={hours per feature},
	x tick label style={
		/pgf/number format/1000 sep=},
	ylabel=Hours,
	enlargelimits=0.05,
	legend style={at={(0.5,-0.1)},
	anchor=north,legend columns=-1},
	ybar interval=0.7,
]
\addplot 
	coordinates {(1, 50)};
\addplot 
	coordinates {(2, 40)};
\legend{is, should}
\end{axis}
\end{tikzpicture}


% NOTE: meeting minutes are kept online
% NOTE: time reports are kept online

% vim: ft=tex
\chapter{Project plan}
\section{Software development process}
The \gls{RUP} is used to plan and manage this term project. It’s an iterative,
structured, yet flexible development process which suits this kind of project.
At HSR, it’s taught as part of the Software Engineering courses and is thus a
primary candidate.

Another candidate was Scrum, which we decided against as it’s only feasible
with teams of three to nine developers.

\section{Timetable}
\subsection{Estimated time}\label{sec:projplan:est-time}
The project spans from September \nth{19} until December \nth{23}. We expect a
necessary effort of about 28.5 hours a week which results in a total of
400 hours per group member. See \autoref{tab:timetable}.

\begin{table}[H]
  \centering
  \begin{tabular}{|p{100mm}|p{35mm}|}
    \hline 	\bf Project duration & 14 weeks \\ \hline
	\bf Manpower & 2 \\ \hline
	\bf Effort per person & 28.5 hours per week \\ \hline
	\bf Total estimated time without weighted damage & 720 hours \\ \hline
	\bf Total estimated time including weighted damage & 856 hours \\ \hline
	\bf Project start & September \nth{19} 2016 \\ \hline
	\bf Project end & December \nth{23} 2016 \\ \hline
  \end{tabular} \\
  \caption{Timetable}
  \label{tab:timetable}
\end{table}

The average between best case and worst case is $\frac{720 + 856\,hours}{2} =
788$ hours. Counting in a reserve of 10\% more of additional effort that might be needed to
mitigate materialized risks, this results in $788\,hours + 10\% *
136\,hours\approx 800$ hours of total effort.


\subsection{Time tracking}
All time spent on the bachelor thesis are done on Everhour. Every time record
is associated with a GitHub issue where applicable.
The estimated time is set directly on each GitHub issue using the Everhour plugin.

Everhour allows the export of all time spent on a project, which will be
included as part of the final submission of this bachelor thesis.

\subsection{Quality assurance}


\subsubsection{Software}
To ensure a consistent code quality in the main development branch
(\emph{master}), unit tests and integration test are run before every push of
new changes to the repository on GitHub. Any pending issues found are to be
fixed before the push.


The development of larger -- requiring more than one commit -- features is done
on a specific feature branch.  Upon completion, instead of merging the branch
directly into the main development branch, a pull request is opened on GitHub.
This allows the other student and/or the client to review, discuss, and improve
the changes. The final merge is to be done by a person other than the one who
opened the pull request.

\subsubsection{Documentation}
Important sections are proof read by the co-student and discussed during the
weekly stand-up meetings.

At the beginning of this bachelor thesis, the professor has been consolidated
for inputs regarding the structure of this document. His advice has been
respected as far as feasible.

Andy Rohr from mindclue GmbH has assisted as another proof reader, especially
for technical correctness regarding the Roadster framework.


\subsection{Meetings}
Meetings are called for by the students on an as-needed basis in consultation
with the professor or the client. Prior to a meeting, the students prepare a meeting
agenda on the wiki of the documentation repository on GitHub.

The meeting agenda for meetings with the professor include the new progress
(incremental), problems, agenda (such as administrative questions), and
short-term goals.

After each meeting, the students follow up by writing down meeting minutes
which include the list of actual participants, decisions made, and pending
tasks.

In addition to the meetings with the client and the professor, the students
hold a weekly standup meeting placed in early/mid-week, which is to be a short
exchange of the current project status, review of the last week, and a
discussion of the next short-term steps.

\subsubsection{Review}
By the end of the bachelor thesis, a total of six meetings (including the
kickoff meeting) have been conducted together with the professor. The meeting
minutes are on \cite[Meetings]{gh:wiki}.

The client mindclue GmbH has been met four times. The first meeting was to
gather the requirements during the first week. The second meeting was necessary
to clarify the requirements and decide upon a more structured format of the
requirements (Cucumber features). The third meeting took place during the
second Elaboration iteration to help with Roadster's codebase in preparation of
the upcoming construction phase. The last meeting was a code review meeting
after the first Construction iteration. Further code reviews were performed
directly on GitHub.

Acute discussions and more lightweight decisions with the client were held on
the particular GitHub issues \cite{gh:issues} directly, as well as the Slack
communication platform.


\section{Risks}
There were nine risks that have been identified by the end of the Inception phase.
The risks have a total damage of 369 hours. The total damage hours multiplied
by the probability of admission result in a total of 137 hours. The weighted damage
hours are included in the project planning.

\subsection{Handling risks}
Due to the nature of the risks, it is only natural they change over the course of a project.
To mitigate this, the risks are checked regularly (in weekly meetings) using \autoref{tab:init-risks}.

Changes to existing risks are possible. This usually means that either the likelihood or
the unweighted damage must be adapted immediately. Moreover, it is possible for a risk to
be completely ruled out, or that a new risk arises. All these points need to be discussed in
the team and tracked accordingly.

\section{Listed risks}
\begin{tabular}[t]{@{}>{\raggedright}p{0.45\textwidth}}
  \textbf{\textit{P = Probability}}
  \begin{enumerate}[topsep=0pt,itemsep=-2pt,leftmargin=13pt]
  \item Rare (10\%)
  \item Unlikely (30\%)
  \item Possible (50\%)
  \item Likely (70\%)
  \item Certain (90\%)
  \end{enumerate}
\end{tabular}
\begin{tabular}[t]{@{}>{\raggedright}p{0.52\textwidth}@{}}
  \textbf{\textit{D = Damage potential  / R = Risk}}
  \begin{enumerate}[topsep=0pt,itemsep=-2pt,leftmargin=13pt]
  \item Insignificant
  \item Negligible
  \item Moderate
  \item Serious
  \item Significant
  \end{enumerate}
\end{tabular}

\autoref{tab:init-risks} lists the initially identified risks. The delay is
specified in days. One day equals 16 man-hours.

\begin{center}
  \begin{longtable}{|p{6mm}|p{30mm}|p{6mm}|p{8mm}|p{30mm}|p{64mm}|}
    \hline \multicolumn{1}{|l|}{\textbf{ID}} &
    \multicolumn{1}{l|}{\textbf{Description}} &
    \multicolumn{1}{l|}{\textbf{P}} &
    \multicolumn{1}{l|}{\textbf{DP}} &
    \multicolumn{1}{l|}{\textbf{Prevention}} &
    \multicolumn{1}{l|}{\textbf{Measures to be taken upon event}} \\ \hline
    \endfirsthead

    \multicolumn{6}{c}%
    {{\bfseries \tablename\ \thetable{} -- continues}} \\
    \hline \multicolumn{1}{|l|}{\textbf{ID}} &
    \multicolumn{1}{l|}{\textbf{Description}} &
    \multicolumn{1}{l|}{\textbf{P}} &
    \multicolumn{1}{l|}{\textbf{DP}} &
    \multicolumn{1}{l|}{\textbf{Prevention}} &
	\multicolumn{1}{l|}{\textbf{Measures to be taken upon event }} \\ \hline
    \endhead

    \hline \multicolumn{6}{c}{{Continues on the next page}} \\ \hline
    \endfoot

    \hline
    \endlastfoot
    R01
		& Roadster requires different ZMQ contexts to function (not possible with CZTop because CZMQ hides contexts)
		& \cellcolor{green!50}1
		& \cellcolor{green!50}3
		& check with client	(done)
		& extract and run affected ZMQ \newline sockets in their own process \newline delay: 1-2 days \\ \hline
	R02
		& Wrong protocols chosen / protocol design flaw
		& \cellcolor{orange!50}3
		& \cellcolor{orange!50}5
		& Architecture \newline reviews prototypes
		& Fix (reevaluate reengineer, redesign) \newline delay: 8-12 days	\\ \hline
	R03
		& ZMQ communication patterns (such as Binary Star) are difficult to implement as clean, reusable code
		& \cellcolor{yellow!50}2
		& \cellcolor{yellow!50}4
		& Use software engineering knowhow to aim for clean, reusable prototypes
		& Nice solution: \newline build more iteratively, step by step \newline delay: 2-4 days \newline \newline dirty solution:
		\newline customized solution built right into Roadster, not as a public gem \newline delay: 1-2 days \\ \hline
	R04
		& CZTop design flaws/limitations
		& \cellcolor{green!50}2
		& \cellcolor{green!50}2
		& Check functionality in the elaboration phase
		& Adapt CZTop \newline delay: 1-2 days \\ \hline
	R05
		& CZMQ changes API
		& \cellcolor{green!50}1
		& \cellcolor{green!50}3
		& (hope)
		& Adapt CZTop, change CZTop adapter in Roadster, or just avoid upgrading CZMQ (use a commit before the breaking change) \newline delay: 1-2 days \\ \hline
	R06
		& Wrong time estimations
		& \cellcolor{yellow!50}4
		& \cellcolor{yellow!50}3
		& Use time well during planning, and define clear milestones
		& If possible, change the duration of the individual project phases. Otherwise, drop planned features
		(starting with the optional goal) \newline delay: 4-5 days \\ \hline
	R07
		& Managing multiple Projects (at least one per repo) on GitHub too painful
		& \cellcolor{green!50}4
		& \cellcolor{green!50}1
		& Setup project structure in the elaboration phase
		& Partial solution: \newline CodeTree (cannot seem to be used for private repos like Roadster itself (maybe yes! see  mindclue/roadster\#5) \newline
		\newline complete solution: \newline Use a single Project which just has cards that link to issues from other repos.
		Linking to "foreign" issues is additional effort but should be straight forward using GitHub syntax (https://github.com/org/repo/issues/42)
		\newline delay: 1 day \\ \hline
	R08
		& Prolonged loss of a team member
		& \cellcolor{green!50}2
		& \cellcolor{green!50}3
		& Track absences in meeting minutes.
		& In a prolonged absence, move milestones and, if necessary, change the project scope. \newline delay: 3-10 days \\ \hline

	R09
		& Failure to achieve the defined task in time
		& \cellcolor{green!50}1
		& \cellcolor{green!50}4
		& Continuous monitoring whether we are on schedule and whether all requirements are met.
		& Meeting convened as we still can transpose a large part of the required task within the prescribed period. \newline delay: 1-5 days\\ \hline
   \caption{Initial risks} \label{tab:init-risks} \\
   \end{longtable}
\end{center}

\begin{table}[H]
  \centering
  \scriptsize
  \begin{tabular}{|m{27mm}|m{24mm}|m{20mm}|m{20mm}|m{20mm}|m{20mm}|@{}m{0pt}@{}}
    \hline
    \bf Probability / Damage
  & \bf 1-Insignificant.
  & \bf 2-Negligible
  & \bf 3-Moderate
  & \bf 4-Serious
  & \bf 5-Significant
  & \\ [10pt]
    \hline
    \bf 5-Certain 
  & \cellcolor{yellow!50}
  & \cellcolor{yellow!50}
  & \cellcolor{orange!50}
  & \cellcolor{red!50}
  & \cellcolor{red!50}
  & \\ [10pt]
    \bf 4-Likely
  & \cellcolor{green!50} R07
  & \cellcolor{yellow!50}
  & \cellcolor{yellow!50} R06
  & \cellcolor{orange!50}
  & \cellcolor{red!50}
  & \\ [10pt]
    \bf 3-Possible
  & \cellcolor{green!50}
  & \cellcolor{green!50}
  & \cellcolor{yellow!50}
  & \cellcolor{yellow!50}
  & \cellcolor{orange!50} R02
  & \\ [10pt]
    \bf 2-Unlikely
  & \cellcolor{green!50}
  & \cellcolor{green!50} R04
  & \cellcolor{green!50} R08
  & \cellcolor{yellow!50} R03
  & \cellcolor{yellow!50}
  & \\ [10pt]
    \bf 1-Rare
  & \cellcolor{green!50}
  & \cellcolor{green!50}
  & \cellcolor{green!50} R01, R05
  & \cellcolor{green!50} R09
  & \cellcolor{green!50}
  & \\ [10pt]
    \hline
  \end{tabular} \\
  \caption{Initial risk matrix}
\end{table}

\begin{ganttchart}[
  hgrid,
  vgrid,
  x unit=9mm
]{1}{14}
\ganttset{bar incomplete/.append style={fill=gray!40},
  group/.append style={draw=black, fill=gray},}
\gantttitle{Calendar weeks}{14} \\
\gantttitlelist{38,...,51}{1} \\
\gantttitle{Semester week}{14} \\
\gantttitlelist{1,...,14}{1} \\
\ganttgroup{Inception}{1}{1} \\
\ganttgroup{Elaboration 1}{2}{4} \\
\ganttgroup{Elaboration 2}{5}{6} \\
\ganttgroup{Elaboration 3}{7}{7} \\
\ganttgroup{Construction 1}{8}{9} \\
\ganttgroup{Construction 2}{10}{11} \\
\ganttgroup{Construction 3}{12}{13} \\
\ganttgroup{Transition}{14}{14} \\
\ganttbar[bar/.append style={fill=green}]{R01}{1}{6} \\
\ganttbar[bar/.append style={fill=orange}]{R02}{1}{5}\ganttbar[bar/.append style={fill=yellow}]{}{5}{5}\ganttbar[bar/.append style={fill=green}]{}{5}{6} \\
\ganttbar[bar/.append style={fill=yellow}]{R03}{1}{5}\ganttbar[bar/.append style={fill=green}]{}{5}{6} \\
\ganttbar[bar/.append style={fill=green}]{R04}{1}{5} \\
\ganttbar[bar/.append style={fill=green}]{R05}{1}{4} \\
\ganttbar[bar/.append style={fill=yellow}]{R06}{1}{5}\ganttbar[bar/.append style={fill=green}]{}{5}{7} \\
\ganttbar[bar/.append style={fill=green}]{R07}{1}{2} \\
\ganttbar[bar/.append style={fill=green}]{R08}{1}{13} \\
\ganttbar[bar/.append style={fill=green}]{R09}{1}{13} \\
%\ganttmilestone{Milestone 1}{11}
\end{ganttchart}

% TODO proof reading
\begin{table}[H]
  \centering
  \begin{tabular}{|p{20mm}|p{20mm}|p{102mm}|}
    \hline \bf Week & \bf Risk & \bf Description \\% [10pt]
    \hline 4 & R02 & The roughly planning of the protocols eliminated many uncertainties. \newline 
	The protocols were tested and no errors were found. \\% [10pt]
	\hline 6 & R02 & The prototype implementation showed that the rhougly designed protocol works as planned. \\% [10pt]
    \hline 4 & R03 & The evaluation of the different variants has reduced the risk. \\% [10pt]
	\hline 6 & R03 & The prototype implementation eliminated the risk. \\% [10pt]
	\hline 4 & R06 & Through the designing of all planned protocols and the first steps with roadster has reduced the risk. \\% [10pt]
	\hline 7 & R06 & The more effort involved in implementing the federation prototypes shows
		 that at the beginning of the construction phase it must be planned with more reserve time.
		The risk can be eliminated by means of the insights gained. \\% [10pt]
    \hline
  \end{tabular} \\
  \caption{Risk timeline change protocol}
\end{table}


\section{Phases \& iterations}
\autoref{tab:phases} illustrates the project's phases and iterations.

\begin{center}
  \begin{longtable}{ | p{25mm} | p{25mm} p{35mm}| p{5mm} | }
    \hline \multicolumn{1}{|c|}{\textbf{Iteration}} &
    \multicolumn{2}{p{70mm}|}{\textbf{Description}} &
    \multicolumn{1}{c|}{\textbf{Due}} \\ \hline
    \endfirsthead

    \multicolumn{4}{c}%
    {{\bfseries \tablename\ \thetable{} -- continues}} \\
    \hline \multicolumn{1}{c}{\textbf{Iteration}} &
    \multicolumn{2}{p{70mm}}{\textbf{Description}} &
    \multicolumn{1}{c}{\textbf{Due}} \\ \hline
    \endhead

    \hline \multicolumn{4}{c}{{Continues on the next page}} \\ \hline
    \endfoot

    \hline
    \endlastfoot
	Inception
	& \multicolumn{2}{p{70mm}|}{Setup proj mgmt, init documentation, define scope, understand requirements,
	set priorities, assess \& analyze risks, estimate schedule, get familiar with Roadster}
	& SW01 \\ \hline
	  \textbf{MS Inception}
	& \textbf{Date}
	& \multicolumn{2}{l|}{25th Sept 2016} \\
	& \textbf{Description}
	& \multicolumn{2}{l|}{Inception phase complete} \\
	& \textbf{Workproducts}
	& \multicolumn{2}{l|}{Project plan} \\
	& & \multicolumn{2}{l|}{Risk matrix} \\
	& & \multicolumn{2}{l|}{Project mgmt infrastructure} \\
	\hline
	\hline
	Elaboration 1
	& \multicolumn{2}{p{70mm}|}{Gather requirements, fundamental thoughts on testing, roughly protocol designs,
	federation, single \& multi level HA, persistence, key distribution, OPC-UA HA interface}
	& SW04 \\ \hline
	  \textbf{MS E1:}
	& \textbf{Date}
	& \multicolumn{2}{l|}{16th Oct 2016} \\
	Protocol & \textbf{Description}
	Designs & \multicolumn{2}{l|}{Protocol designs are defined} \\
	& \textbf{Workproducts}
	& \multicolumn{2}{l|}{Requirements \& use cases} \\
	&
	& \multicolumn{2}{l|}{Protocol designs} \\ \hline
	Elaboration 2
	& \multicolumn{2}{p{70mm}|}{Implement prototypes (federation, single \& multi level HA, persistence, secure socket,
	communication, OPC-UA HA interface}
	& SW06 \\ \hline
	  \textbf{MS E2:}
	& \textbf{Date}
	& \multicolumn{2}{l|}{30th Oct 2016} \\
	Prototypes & \textbf{Description}
	& \multicolumn{2}{l|}{Prototypes are implemented and tested.} \\
	& \textbf{Workproducts}
	& \multicolumn{2}{l|}{Runnable prototypes} \\
	  \hline
	  \hline
	Elaboration 3
	& \multicolumn{2}{p{70mm}|}{Revise risks, finish bulk of documentation, (reserve)}
	& SW07 \\ \hline
	Construction 1
	& \multicolumn{2}{p{70mm}|}{Port CZTop, federation (refactor \& integrate prototype)}
	& SW09 \\ \hline
	\textbf{MS C1:}
	& \textbf{Date}
	& \multicolumn{2}{l|}{20th Nov 2016} \\
	Federation & \textbf{Description}
	& \multicolumn{2}{l|}{Runnable federation functionality on top of CZTop.} \\
	& \textbf{Workproducts}
	& \multicolumn{2}{l|}{Federation functionality, CZTop integration} \\
	\hline
	\hline

	Construction 2
	& \multicolumn{2}{l|}{Refactor, integrate and verify HA prototypes, persistence replication}
	& SW11 \\ \hline
	\textbf{MS C2:}
	& \textbf{Date}
	& \multicolumn{2}{l|}{4th Dec 2016} \\
	HA & \textbf{Description}
	& \multicolumn{2}{l|}{Working HA functionality and persistence replication} \\
	& \textbf{Workproducts}
	& \multicolumn{2}{l|}{HA functionality} \\
	& & \multicolumn{2}{l|}{Persistence replication} \\ \hline
	Construction 3
	& \multicolumn{2}{p{70mm}|}{Security (implement prototype, test), OPC UA HA (implement prototype, verify)}
	& SW13 \\ \hline
	\textbf{MS C3:}
	& \textbf{Date}
	& \multicolumn{2}{l|}{18th Dec 2016} \\
	Security & \textbf{Description}
	& \multicolumn{2}{l|}{Secure communication between nodes} \\
	& \textbf{Workproducts}
	& \multicolumn{2}{l|}{Security} \\
	\hline
	\hline

	Transition 1
	& \multicolumn{2}{p{70mm}|}{Polish documentation, write abstract, create poster, print documentation \& upload artifacts}
	& SW14 \\ \hline
	\textbf{MS T1:}
	& \textbf{Date}
	& \multicolumn{2}{l|}{23rd Dec 2016 - 5:00 pm} \\
	Delivery & \textbf{Description}
	& \multicolumn{2}{l|}{Complete handover of thesis} \\
	& \textbf{Workproducts}
	& \multicolumn{2}{l|}{Thesis in paperform} \\
	\hline
	\hline

   \caption{Phase and iterations} \label{tab:phases}
   \end{longtable}
\end{center}

\chapter{ZMQ}
TODO explain ZMQ in greater detail

TODO strong abstraction (one socket for many connections, connection handling transparent, transport and encryption transparent, no concept of peer addresses)\\
TODO brokerless/with broker, up to you\\
TODO basic patterns\\
TODO extended patterns\\
TODO not only a "MOM", but a multi threading library (Actor pattern)\\

\end{document}
