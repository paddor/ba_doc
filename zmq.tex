% vim: ft=tex
\chapter{\zmq}\label{ch:zmq}
\zmq is a \gls{MOM} implemented as an open source library, that is, it does not
require a dedicated broker. Instead, it offers sockets with an abstract
interface similar to \acrshort{BSD} sockets. Different types of sockets are used for
different messaging patterns such as request-reply, publish-subscribe, and
push-pull.
% TODO: decenralized, scalable

A single socket can bind/connect to multiple endpoints, which allows \zmq to
use round-robin on the sender side, and fair-queueing on the receiver side,
where applicable. It does not matter whether the communication happens
in-process (between threads), inter-process (e.g. over \glspl{unix-domain-socket}), or
inter-node (e.g. over \acrshort{TCP}/\acrshort{PGM}/\acrshort{TIPC}), since the transport is completely
abstracted away. The same goes for connection handling; an arbitrary amount of
connections is handled over a single socket and reconnecting after short
network failures is done transparently.

\paragraph{Order of bind() and connect()}
Usually in network programming, the server is the one performing the
\cpp{bind()} operation and the client is the one performing the \cpp{connect()}
operation. This restriction does not hold true in \zmq. Since \zmq completely
abstracts the connection handling away, the order in which these two operations
happen is insignificant. In case the other side has not bound its socket
yet, \zmq will just retry establishing a connection if the last attempt failed,
by default every 100 milliseconds. This means that it is a free choice as to which socket
performs what action. Each socket type merely has default action.


\paragraph{Performance} \zmq is lightweight and allows for extremely low latencies, which means it can
also be used as the fabric of concurrent applications, e.g. for the \gls{actor-model}.
In case of the TCP transport, it incorporates advanced techniques such
as smart message batching to achieve significantly higher throughputs than with
raw TCP or other \gls{MOM} solutions as observed and described by the CERN in
\cite[Figure 2, Middleware evaluation and prototyping, p.~4]{cern:new-cmw}.
This is done by avoiding redundant networking stack traversals as explained on
\cite[Performance]{zmq:faq}.

\paragraph{Communication patterns} To build a solution with \zmq, its sockets are used as building blocks to
design custom message flows. Certain patterns are used to achieve reliability
with respect to the failure types that need to be addressed in particular.  The
\gls{zguide} explains best practices, including commonly needed, resilient
messaging patterns.

The aforementioned characteristics make \zmq a valuable asset when it comes to building
robust, distributed high-performance systems.

\section{Transport security}\label{sec:zmq:security}
Since version 4.0 (released in October 2013), \zmq boasts strong encryption and
authentication, based on the excellent and highly renown
\gls{libsodium}\footnote{There is also the possibility to do it using
\gls{tweetnacl} which avoids the additional dependency.}. The protocol used
(CurveZMQ) is described in \cite{zmq:curvezmq}.

Transport encryption is completely transparent to the application, and is one
of the available security mechanisms. It is called CURVE. The
security handshake is designed to be highly resilient against
\glspl{ddos-attack} attacks using encrypted and authenticated cookies, so the
server does not actually allocate any memory before the handshake is completed,
and key generation is very cheap \cite[p.~2]{djb:ed25519} with
Curve25519\footnote{\url{https://en.wikipedia.org/wiki/Curve25519}} \gls{ECC}.

To enable, one socket is designated as the CURVE server, the other as the CURVE client.
Server authentication is always performed. This means that nodes running the
client sockets must be in possession of the server's public key.

Client authentication is optional. If desired --- meaning not just any client
is allowed to connect --- the clients' public keys (either shared by all
clients, or a unique key for each client) must be already available to the node
running the server sockets. The server socket actually talks to another,
designated socket to do authentication. The protocol used between them is
\gls{ZAP}. Completely abstracting the authentication in another socket allows
any kind of authentication service to be easiliy plugged in. It is also trivial
to write such a ZAP handler to register new client public keys somewhere to be
checked and confirmed by a human, which simplifies the process of adding new
client public keys.


% TODO: CurveCP, amplification attacks, message size, ...

\section{Data serialization}
Data serialization is outside the scope of \zmq. To fill the gap, one typically
uses another library such as MsgPack\footnote{\url{http://msgpack.org}},
Protocol
Buffers\footnote{\url{https://developers.google.com/protocol-buffers/}}, or
even a programming language's built-in object serialization
support\footnote{such as Ruby's marshalling support:
\url{http://ruby-doc.org/core/Marshal.html}}.

\section{Language availability}
Bindings or full-blown reimplementations of \zmq exist for a plethora of
programming languages, including C, C++, Java, Ruby, Perl, Erlang, Python, Lua,
Go, Tcl, PHP, and .Net. This allows for building distributed software systems
using modern approaches like service-oriented architectures, where different
parts may be implemented in different languages.

\section{CZMQ}\label{sec:zmq:czmq}
\gls{czmq} is a high-level abstraction layer for \zmq. It makes working with the \zmq
library more expressive and allows for better portability. It also provides
additional functionality such as a reactor, a simple actor implementation, as
well as utilities for certificate and authentication handling, and LAN node
discovery. This is the recommended way of using \zmq nowadays, as it allows for
much cleaner C code and also simplifies bindings for other languages.

\subsection{CZTop}
\emph{CZTop} is a young, \gls{FFI}-based Ruby binding for CZMQ. It is currently
the only binding in the Ruby ecosystem to support multiple Ruby interpreters
(\gls{MRI}, \gls{JRuby}, and \gls{Rubinius}), as well as recent versions of
CZMQ. That implies support for encryption.

It has been developed to offer a Ruby-like high-level interface built on common
Ruby idioms, instead of just being a bare binding to the low-level C library
functions.

The low-level handling is actually handled in a different Ruby library called
\emph{czmq-ffi-gen}, which is nothing but a bare Ruby binding to CZMQ, handling
nothing more than the creation and destruction of the native objects (where
pointer handling is involved in the form of \sh{FFI::Pointer} objects). This
library is purely generated from the API definitions within the CZMQ repository
(fairly simple XML files) using Zproject. Zproject can be used to generate
dozens of language bindings for any C library adhering to the \gls{CLASS}
standard.

Both \emph{CZTop} and \emph{czmq-ffi-gen} have been developed by Patrik Wenger
during the winter 2015/16.

\subsection{Differences to RabbitMQ}
% TODO: differences to RabbitMQ, \cite{ph:amqp}
The big difference between \zmq and RabbitMQ is that \zmq is brokerless and do not 
have built in reliabilty for message delivery. For further reading 
visit the \zmq site\footnote{\url{http://zeromq.org/docs:welcome-from-amqp}}.
