\documentclass[a4paper]{article}
\usepackage[utf8]{inputenc}
\usepackage[a4paper]{geometry}
\geometry{verbose, marginparwidth=15mm, marginparsep=3mm, tmargin=25mm}
\usepackage[english]{babel}
\usepackage{graphicx}
\usepackage[hyphens]{url}
\usepackage[flushmargin,hang]{footmisc}
\usepackage{hyperref}
\usepackage[
backend=biber,
hyperref=true,
url=false,
isbn=false,
backref=false,
% style=custom-numeric-comp,
%citereset=chapter,
maxcitenames=3,
maxbibnames=100,
block=none]{biblatex}
\bibliography{goals}
\hypersetup{
	unicode=true,
	pdftitle={Bachelor Thesis "Roadster"},
	pdfsubject={Extending a SCADA framework to support high availability},
	pdfauthor={Patrik Wenger},
	pdfkeywords={Ruby} {port} {high-availability} {bachelor thesis},
}
\usepackage{csquotes}
\usepackage{pdfpages}
\title{Extending a SCADA framework to support high availability --- Bachelor Thesis: Roadster}
\author{Andy Rohr, Patrik Wenger, Manuel Schuler}

\usepackage{listings}
\lstdefinestyle{customruby}{ % custom style for Ruby listings
  belowcaptionskip=1\baselineskip,
  breaklines=true,
  frame=single,
  xleftmargin=\parindent,
  language=C,
  showstringspaces=false,
  basicstyle=\ttfamily,
  keywordstyle=\bfseries\color{green!40!black},
  commentstyle=\itshape\color{purple!40!black},
  identifierstyle=\color{blue},
  stringstyle=\color{orange},
}
\newcommand{\rb}[1]{\lstinline[style=customruby]{#1}} % inline Ruby
\usepackage{fancyhdr}
\usepackage{lastpage}

\begin{document}
\lstset{language=ruby}
\pagestyle{fancyplain}
\fancyhf{}
\cfoot{\thepage\ of \pageref{LastPage} }
\setlength{\headheight}{50pt}
\maketitle
\fancyhead[L]{\includegraphics[trim=10 10 10 10, clip=true, width=0.3\textwidth]{img/hsr_logo.pdf}}

\section{Client and Supervisor}
\begin{description}
	\item [Client:] mindclue GmbH
	\item [Client Contact:] Andy Rohr, andy.rohr@mindclue.ch
	\item [Supervisor:] Prof. Dr. Farhad Mehta, HSR Rapperswil
\end{description}


\section{Students}
\begin{itemize}
	\item Patrik Wenger, pwenger@hsr.ch
	\item Manuel Schuler, mschuler@hsr.ch
\end{itemize}

\section{Setting}
The company mindclue GmbH, located in Ziegelbrücke, develops
SCADA\footnote{Supervisory Control And Data Acquisition, see
https://en.wikipedia.org/wiki/SCADA} applications for controlling systems used
in traffic systems, energy, and water supply. For that purpose, the company
developed the \emph{Roadster} framework, which provides the basis for project
specific applications.\\

\emph{Roadster} is implemented in Ruby and is architecturally based on the
Actor model \cite{wiki:actor-model}, which means that multiple parallel running,
single-threaded processes (actors) are coupled via messaging (shared-nothing
architecture). The messaging layer is based on ZeroMQ (ZMQ \cite{zeromq}) and
has an asynchronous/non-blocking nature. Several different messaging
patterns/messaging protocols are used. Additionally, the system includes a web
UI which is based on ember.js and connected to the messaging via WebSocket.
Fundamentally, the system follows the Reactive Manifesto
\cite{reactivemanifesto}.\\

\clearpage
\section{Goals}
The main aim of this thesis is to extend the \emph{Roadster} framework to
support high availability.\\

\emph{Roadster} currently lacks the following features:

\begin{enumerate}
	\item A \emph{Roadster} application is currently limited to one node
		(one instance). The goal is to be able to build systems which
		consist of multiple nodes. Example: A master node forms
		together with multiple subordinate nodes a system (basically a
		distributed system). The subordinate nodes are responsible for
		their respective subtask of a facility and communicate with
		their components (e.g. SPS). The subsystems are integrated into
		the master node to form an overall view of the facility, which
		is visualized in the web UI.

This requirement implies:

\begin{itemize}
	\item Extension of the messsaging protocols to allow the communication
		between nodes across levels in the hierarchy.

	\item Encryption of the communication.
\end{itemize}

\item A \emph{Roadster} application has to have the ability to be run as a
	highly available active/passive cluster. Two nodes (primary and backup)
	of the same level in the hierarchy form a hot-standby cluster, where
	the two nodes stay in constant connection with each other. In case the
	active node fails, the passive node immediately takes over and becomes
	the new active node.

This requirement implies:

\begin{itemize}
	\item Extension of the messaging protocols to allow the communication
		between nodes within the same level in the hierarchy.

	\item Implementation of resilient failover mechanisms.

	\item Encryption of the communication.
\end{itemize}


\item With (2), it is possible to implement a highly available OPC UA server.
	OPC UA \cite{opc-ua} includes a concept for redundant UA servers.
	\emph{Roadster} already implements a OPC UA server, although not highly
	available.

This requirement implies:

\begin{itemize}
	\item Extension of the OPC UA implementation to support OPC UA HA
		mechanisms.
\end{itemize}
\end{enumerate}

The client essentially wants \emph{Roadster} to be extended by the three
features described above, whereas the \textbf{third one is optional} and is
only to be approached in case there is time for it. The \textbf{same applies
to the encrypted communication requirement}.

\section{Tasks}
Here is an overview of the currently planned tasks that need to be performed:

\begin{enumerate}
	\item Getting familiar with the concepts and implementation of
		\emph{Roadster}, particularly the messaging layer. For that,
		Andy Rohr (mindclue GmbH) will provide an extensive
		introduction.

	\item Elaboration of a subnode concept. This includes the design,
		implementation, and testing of extensions of the existing
		messaging protocols for the communication between nodes of
		different levels in the hierarchy.

		One of the most important \emph{Roadster} messaging protocols
		is called \emph{Clone State Protocol} and is based on the
		\emph{Clone Pattern} described in the zguide
		\cite{zguide:clone-pattern}. It provides means to replicate the
		current state of the domain model into the different actors
		within an application, as those actors behave according to the
		following principle \cite{golang:sharemem}:

		\begin{quote}
		``Don't communicate by sharing state, share state by communicating.''
		\end{quote}

		For the communication between nodes, the protocol has to be
		extended accordingly. The messages are basically Ruby objects
		serialized using \rb{Marshal.dump} and transported over ZMQ
		sockets.

	\item Elaboration of a HA concept. This includes the design,
		implementation, and testing of extensions of the existing
		messaging protocols for the communication betwen nodes within
		the same level in the hieararchy, including resilient failover
		mechanisms.

		The \emph{Binary Star Pattern}
		\cite{zguide:binary-star}\cite{zguide:adding-binary-star} forms
		the basis of the HA concept. However, the concept will have to
		be adapted to fit \emph{Roadster}'s needs. Availability has to
		be ensured under the following scenarios:

		\begin{itemize}
			\item hardware or software failure of the primary node
			\item network failure
		\end{itemize}

		A more detailed definition will have to be worked out during
		the elaboration phase of the thesis.

	\item Implementation of encrypted communication. The current
		implementation is based on ZMQ 3 and the Ruby binding ffi-rzmq
		\cite{github:ffi-rzmq}. However, encryption has been introduced
		in ZMQ 4, which isn't supported by ffi-rzmq. On top of that,
		ffi-rzmq is not being maintained anymore. A possible solution
		is CZTop \cite{github:cztop}, which is based on CZMQ
		\cite{czmq} and authored by Patrik Wenger.

		In case CZTop is used, it would have to be extended to allow
		the watching of ZMQ sockets by EventMachine
		\cite{gem:eventmachine}, e.g.
		\rb{EM.watch(socket_file_descriptor)}. \emph{Roadster} uses
		EventMachine as a reactor implementation
		\cite{wiki:reactor-pattern}.

	\item Extensions of the \emph{Roadster} OPC UA server implementation to
		support HA mechanisms. This part of \emph{Roadster} is a Ruby
		extension, which is implemented based on the Unified Automation
		C++ SDK \cite{ua-server-sdk}. The extension is written in C++
		and uses rbplusplus \cite{github:rbplusplus}.
\end{enumerate}

\section{License}
To grant mindclue GmbH unrestricted usage of the student's contributions, the
student's code changes and additions shall be protected under the ISC License
\cite{wiki:isc-license}, which is functionally equivalent to the MIT license
and the Simplified BSD license, but uses simpler language.

\printbibliography
\end{document}
