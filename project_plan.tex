% vim: ft=tex
\chapter{Project plan}
\section{Software development process}
The \gls{RUP} is used to plan and manage this term project. It’s an iterative,
structured, yet flexible development process which suits this kind of project.
At HSR, it’s taught as part of the Software Engineering courses and is thus a
primary candidate.

Another candidate was Scrum, which we decided against as it’s only feasible
with teams of three to nine developers.

\section{Timetable}
\subsection{Estimated time}
The projekt span is from 19.09 until 23.12. We expect 28.5 hours work per week which is a total of
400 hours per group member. See \autoref{tab:timetable}.

\begin{table}[H]
  \centering
  \scriptsize
  \begin{tabular}{|p{80mm}|p{15mm}|}
    \hline 	\bf Project duration & 14 weeks \\ \hline
	\bf count of workers & 2 \\ \hline
	\bf time per worker & 28.5 hours per week \\ \hline
	\bf Total estimated time (without weighted damage) & 720 hours \\ \hline
	\bf Total estimated time (inclusive weighted damage) & 856 hours \\ \hline
	\bf Project start & 19.09.2016 \\ \hline
	\bf Project end & 23.12.2016 \\ \hline
  \end{tabular} \\
  \caption{Timetable}
  \label{tab:timetable}
\end{table}

\subsection{Time tracking}
All schedulings are done with Everhour and with GitHub. GitHub hosts all issues
and Everhour is used for time tracking (should / is).

\subsection{Infrastructure}
\autoref{tab:infrastructure} shows the infrastructure provided by HSR.
\begin{table}[H]
  \centering
  \scriptsize
  \begin{tabular}{|p{25mm}|p{30mm}|}
    \hline 	\bf Servername & sinv-56092.edu.hsr.ch \\ \hline
	\bf ip address & 152.96.56.92 \\ \hline
	\bf os & Ubuntu 14.04 LTS \\ \hline
	\bf VM EndofLife & 03.03.2017 \\ \hline
	\bf CPU & 1 vCPU max. 2.2 GHz \\ \hline
	\bf RAM & 1 GB \\ \hline
	\bf Disk drive & 15 GB \\ \hline
  \end{tabular} \\
  \caption{Infrastructure}
  \label{tab:infrastructure}
\end{table}

\subsection{Tools}
The source code of this document and all of our code contributions are hosted
on GitHub. The students will organize and perform their work directly on the
site as far as possible. This means creating a Project board for each of the
development phases, creating, assigning, and closing issues, as well as using
the Wiki feature to plan and document meetings with the professor and the
client. Time tracking is done externally on Everhour. See
\autoref{tab:dev-tools}.

% TODO: section: quality assurance
\begin{table}[H]
  \centering
  \scriptsize
  \begin{tabular}{|p{50mm}|p{15mm}|p{15mm}|}
    \hline 	\bf Use & Name & Version \\ \hline
	\bf IDE & Vim, RubyMine &  \\ \hline
	\bf version control system & Git & 2.* \\ \hline
	\bf project management  & GitHub, Everhour &  \\ \hline
  \end{tabular} \\
  \caption{Development tools}
  \label{tab:dev-tools}
\end{table}

\subsection{Quality measures}

\subsubsection{Documentation review}
The project partner proof reads important sections which are then discussed
during the weekly stand up meetings.


\subsubsection{Meeting minutes}
Meeting minutes include the participants, agenda items, decisions, as well as
pending items. Manuel Schuler takes over this task and logs everything.
Meeting minutes are saved on the GitHub wiki site 24 hours before the meeting
starts and after the meeting it will be updated with the new information.

\subsubsection{Git policy}
% TODO to be defined..
% FIXME: do we have one?
It's only allowed to push if all ruby spec tests pass.
Each project participant is allowed to mutate and check-in each file. 
In the case of critical changes the remaining project participants must be informed.


\subsection{Meetings}
Meetings are agreed in consultation with the professor or the client.
The meeting minutes are kept, the individual agenda items, decisions and todos.
The project team has a weekly standup meeting, placed in early / mid-week and
serves only as a short exchange of information, to review the last week, recap
the week goals, problems and plenary for questions or suggestions.


\section{Risks}
There were nine risks that have been identified by the end of the Inception phase.
The risks have a total damage of 369 hours. The total damage hours multiplied
by the probability of admission result in a total of 137 hours. The weighted damage
hours are included in the project planning.

\subsection{Handling risks}
Due to the nature of the risks, it is only natural they change over the course of a project.
To mitigate this, the risks are checked regularly (in weekly meetings) using \autoref{tab:init-risks}.

Changes to existing risks are possible. This usually means that either the likelihood or
the unweighted damage must be adapted immediately. Moreover, it is possible for a risk to
be completely ruled out, or that a new risk arises. All these points need to be discussed in
the team and tracked accordingly.

\section{Listed risks}
\begin{tabular}[t]{@{}>{\raggedright}p{0.45\textwidth}}
  \textbf{\textit{P = Probability}}
  \begin{enumerate}[topsep=0pt,itemsep=-2pt,leftmargin=13pt]
  \item Rare (10\%)
  \item Unlikely (30\%)
  \item Possible (50\%)
  \item Likely (70\%)
  \item Certain (90\%)
  \end{enumerate}
\end{tabular}
\begin{tabular}[t]{@{}>{\raggedright}p{0.52\textwidth}@{}}
  \textbf{\textit{D = Damage potential  / R = Risk}}
  \begin{enumerate}[topsep=0pt,itemsep=-2pt,leftmargin=13pt]
  \item Insignificant
  \item Negligible
  \item Moderate
  \item Serious
  \item Significant
  \end{enumerate}
\end{tabular}

\autoref{tab:init-risks} lists the initially identified risks. The delay is
specified in days. One day equals 16 man-hours.

\begin{center}
  \begin{longtable}{|p{6mm}|p{30mm}|p{6mm}|p{8mm}|p{30mm}|p{64mm}|}
    \hline \multicolumn{1}{l}{\textbf{ID}} &
    \multicolumn{1}{l}{\textbf{Description}} &
    \multicolumn{1}{l}{\textbf{P}} &
    \multicolumn{1}{l}{\textbf{DP}} &
    \multicolumn{1}{l}{\textbf{Prevention}} &
	\multicolumn{1}{l}{\textbf{Measures to be taken upon event}} \\ \hline
    \endfirsthead

    \multicolumn{6}{c}%
    {{\bfseries \tablename\ \thetable{} -- continues}} \\
    \hline \multicolumn{1}{c}{\textbf{ID}} &
    \multicolumn{1}{c}{\textbf{Description}} &
    \multicolumn{1}{c}{\textbf{P}} &
    \multicolumn{1}{c}{\textbf{DP}} &
    \multicolumn{1}{c}{\textbf{Prevention}} &
	\multicolumn{1}{c}{\textbf{Measures to be taken upon event }} \\ \hline
    \endhead

    \hline \multicolumn{5}{r}{{Continues on the next page}} \\ \hline
    \endfoot

    \hline
    \endlastfoot
    R01
		& Roadster requires different ZMQ contexts to function (not possible with CZTop because CZMQ hides contexts)
		& \cellcolor{green!50}1
		& \cellcolor{green!50}3
		& check with client	(done)
		& extract and run affected ZMQ \newline sockets in their own process \newline delay: 1-2 days \\ \hline
	R02
		& wrong protocols chosen / protocol design flaw
		& \cellcolor{orange!50}3
		& \cellcolor{orange!50}5
		& architecture \newline reviews prototypes
		& fix (reevaluate reengineer, redesign) \newline delay: 8-12 days	\\ \hline
	R03
		& ZMQ communication patterns (such as Binary Star) are difficult to implement as clean, reusable code
		& \cellcolor{yellow!50}2
		& \cellcolor{yellow!50}4
		& use software engineering knowhow to aim for clean, reusable prototypes
		& nice solution: \newline build more iteratively, step by step \newline delay: 2-4 days \newline \newline dirty solution:
		\newline customized solution built right into Roadster, not as a public gem \newline delay: 1-2 days \\ \hline
	R04
		& CZTop design flaws/limitations
		& \cellcolor{green!50}2
		& \cellcolor{green!50}2
		& check functionality in the elaboration phase
		& adapt CZtop \newline delay: 1-2 days \\ \hline
	R05
		& CZMQ changes API
		& \cellcolor{green!50}1
		& \cellcolor{green!50}3
		& (hope)
		& adapt CZTop, change CZTop adapter in Roadster, or just avoid upgrading CZMQ (use a commit before the breaking change) \newline delay: 1-2 days \\ \hline
	R06
		& wrong time estimations
		& \cellcolor{yellow!50}4
		& \cellcolor{yellow!50}3
		& use time well during planning, and define clear milestones
		& If possible, change the duration of the individual project phases. Otherwise, drop planned features
		(starting with the optional goal) \newline delay: 4-5 days \\ \hline
	R07
		& managing multiple Projects (at least one per repo) on Github too painful
		& \cellcolor{green!50}4
		& \cellcolor{green!50}1
		& setup project structure in the elaboration phase
		& partial solution: \newline CodeTree (cannot seem to be used for private repos like Roadster itself (maybe yes! see  mindclue/roadster\#5) \newline
		\newline complete solution: \newline Use a single Project which just has cards that link to issues from other repos.
		Linking to "foreign" issues is additional effort but should be straight forward using Github syntax (https://github.com/org/repo/issues/42)
		\newline delay: 1 day \\ \hline
	R08
		& Prolonged loss of a team member
		& \cellcolor{green!50}2
		& \cellcolor{green!50}3
		& Track absences in meeting minutes.
		& In a prolonged absence, move milestones and, if necessary, change the project scope. \newline delay: 3-10 days \\ \hline

	R09
		& Failure to achieve the defined task in time
		& \cellcolor{green!50}1
		& \cellcolor{green!50}4
		& Continuous monitoring whether we are on schedule and whether all requirements are met.
		& Meeting convened as we still can transpose a large part of the required task within the prescribed period. \newline delay: 1-5 days\\ \hline
   \caption{Initial risks} \label{tab:init-risks} \\
   \end{longtable}
\end{center}

\begin{table}[H]
  \centering
  \scriptsize
  \begin{tabular}{|m{27mm}|m{24mm}|m{20mm}|m{20mm}|m{20mm}|m{20mm}|@{}m{0pt}@{}}
    \hline 	\bf Propability / Damage & \bf 1-Insignificant. 	& \bf 2-Negligible 				& \bf 3-Moderate 			& \bf 4-Serious 			& \bf 5-Significant 	& \\ [10pt]
    \hline 	\bf 5-Certain 			& \cellcolor{yellow!50} 	& \cellcolor{yellow!50} 	& \cellcolor{orange!50} 			& \cellcolor{red!50} 		& \cellcolor{red!50} 			& \\ [10pt]
			\bf 4-Likely 			& \cellcolor{green!50} R07 	& \cellcolor{yellow!50} 	& \cellcolor{yellow!50} R06 	& \cellcolor{orange!50} 		& \cellcolor{red!50} 			& \\ [10pt]
			\bf 3-Possible 				& \cellcolor{green!50} 		& \cellcolor{green!50} 		& \cellcolor{yellow!50} 		& \cellcolor{yellow!50} 	& \cellcolor{orange!50} R02 		& \\ [10pt]
			\bf 2-Unlikely 			& \cellcolor{green!50} 	& \cellcolor{green!50} R04 	& \cellcolor{green!50} R08 		& \cellcolor{yellow!50} R03 & \cellcolor{yellow!50} 		& \\ [10pt]
			\bf 1-Rare 			& \cellcolor{green!50} 		& \cellcolor{green!50} 		& \cellcolor{green!50} R01, R05 & \cellcolor{green!50} R09 	& \cellcolor{green!50} 		& \\ [10pt]
    \hline
  \end{tabular} \\
  \caption{Initial risk matrix}
\end{table}

\begin{ganttchart}[
  hgrid,
  vgrid,
  x unit=9mm
]{1}{14}
\ganttset{bar incomplete/.append style={fill=gray!40},
  group/.append style={draw=black, fill=gray},}
\gantttitle{Calendar weeks}{14} \\
\gantttitlelist{38,...,51}{1} \\
\gantttitle{Semester week}{14} \\
\gantttitlelist{1,...,14}{1} \\
\ganttgroup{Inception}{1}{1} \\
\ganttgroup{Elaboration 1}{2}{4} \\
\ganttgroup{Elaboration 2}{5}{6} \\
\ganttgroup{Elaboration 3}{7}{7} \\
\ganttgroup{Construction 1}{8}{9} \\
\ganttgroup{Construction 2}{10}{11} \\
\ganttgroup{Construction 3}{12}{13} \\
\ganttgroup{Transition}{14}{14} \\
\ganttbar[bar/.append style={fill=green}]{R01}{1}{6} \\
\ganttbar[bar/.append style={fill=orange}]{R02}{1}{5}\ganttbar[bar/.append style={fill=yellow}]{}{5}{5}\ganttbar[bar/.append style={fill=green}]{}{5}{6} \\
\ganttbar[bar/.append style={fill=yellow}]{R03}{1}{5}\ganttbar[bar/.append style={fill=green}]{}{5}{6} \\
\ganttbar[bar/.append style={fill=green}]{R04}{1}{5} \\
\ganttbar[bar/.append style={fill=green}]{R05}{1}{4} \\
\ganttbar[bar/.append style={fill=yellow}]{R06}{1}{5}\ganttbar[bar/.append style={fill=green}]{}{5}{7} \\
\ganttbar[bar/.append style={fill=green}]{R07}{1}{2} \\
\ganttbar[bar/.append style={fill=green}]{R08}{1}{13} \\
\ganttbar[bar/.append style={fill=green}]{R09}{1}{13} \\
%\ganttmilestone{Milestone 1}{11}
\end{ganttchart}

\begin{table}[H]
  \centering
  \begin{tabular}{|p{20mm}|p{20mm}|p{102mm}|}
    \hline \bf Week & \bf Risk & \bf Description \\% [10pt]
    \hline 3 & R01 & .... \\% [10pt]
    \hline 3 & R06 & .... \\% [10pt]
    \hline
  \end{tabular} \\
  \caption{Risk timeline change protocol}
\end{table}


\section{Phases \& iterations}
\autoref{tab:phases} illustrates the project's phases and iterations.

\begin{center}
  \begin{longtable}{|p{25mm}|p{25mm} p{45mm}|p{10mm}|}
    \hline \multicolumn{1}{c}{\textbf{Iteration}} &
    \multicolumn{2}{p{70mm}}{\textbf{Description}} &
    \multicolumn{1}{c}{\textbf{Due}} \\ \hline
    \endfirsthead

    \multicolumn{4}{c}%
    {{\bfseries \tablename\ \thetable{} -- continues}} \\
    \hline \multicolumn{1}{c}{\textbf{Iteration}} &
    \multicolumn{2}{p{70mm}}{\textbf{Description}} &
    \multicolumn{1}{c}{\textbf{Due}} \\ \hline
    \endhead

    \hline \multicolumn{4}{r}{{Continues on the next page}} \\ \hline
    \endfoot

    \hline
    \endlastfoot
	Inception
		& \multicolumn{2}{p{70mm}|}{setup proj mgmt, init documentation, define scope, understand requirements,
		set priorities, assess \& analyze risks, estimate schedule, get familiar with Roadster}
		& SW01 \\ \hline
	MS Inception
		& \textbf{Date}
		& \multicolumn{2}{l|}{25th Sept 2016} \\
		& \textbf{Description}
		& \multicolumn{2}{l|}{Inceptionphase ended} \\
		& \textbf{Workproducts}
		& \multicolumn{2}{l|}{project plan} \\
		& & \multicolumn{2}{l|}{risk matrix} \\
		& & \multicolumn{2}{l|}{project mgmt infrastructure} \\ \hline
	Elaboration 1
		& \multicolumn{2}{p{70mm}|}{write use cases, fundamental thoughts on testing, roughly design protocols,
		federation, single \& multi level HA, persistence, key distribution, OPC-UA HA interface}
		& SW04 \\ \hline
	MS E1 Protocol
		& \textbf{Date}
		& \multicolumn{2}{l|}{16th Oct 2016} \\
		Designs & \textbf{Description}
		& \multicolumn{2}{l|}{Protocol designs are defined} \\
		& \textbf{Workproducts}
		& \multicolumn{2}{l|}{requirements \& use cases} \\
		& & \multicolumn{2}{l|}{protocol designs} \\ \hline
	Elaboration 2
		& \multicolumn{2}{p{70mm}|}{implement prototypes (federation, single \& multi level HA, persistence, secure socket,
		communication, OPC-UA HA interface}
		& SW06 \\ \hline
	MS E2 Proto-
		& \textbf{Date}
		& \multicolumn{2}{l|}{30th Oct 2016} \\
	types & \textbf{Description}
		& \multicolumn{2}{l|}{Prototypes are implemented and tested.} \\
		& \textbf{Workproducts}
		& \multicolumn{2}{l|}{Runnable prototypes} \\ \hline
	Elaboration 3
		& \multicolumn{2}{p{70mm}|}{revise risks, finish bulk of documentation, (reserve)}
		& SW07 \\ \hline
	Construction 1
		& \multicolumn{2}{p{70mm}|}{port CZTop, federation (refactor \& integrate prototype)}
		& SW09 \\ \hline
	MS C1 Federation
		& \textbf{Date}
		& \multicolumn{2}{l|}{20th Nov 2016} \\
		& \textbf{Description}
		& \multicolumn{2}{l|}{Runnable federation functionality on top of CZTop.} \\
		& \textbf{Workproducts}
		& \multicolumn{2}{l|}{Federation functonality, CZTop integration} \\ \hline
	Construction 2
		& \multicolumn{2}{p{70mm}|}{refactor, integrate and verify HA prototypes, persistence replication}
		& SW11 \\ \hline
	MS C2 HA
		& \textbf{Date}
		& \multicolumn{2}{l|}{4th Dec 2016} \\
		& \textbf{Description}
		& \multicolumn{2}{l|}{Working HA functionality and persistence replication} \\
		& \textbf{Workproducts}
		& \multicolumn{2}{l|}{HA functonality} \\
		& & \multicolumn{2}{l|}{persistence replication} \\ \hline
	Construction 3
		& \multicolumn{2}{p{70mm}|}{security (implement prototype, test), OPC UA HA (implement prototype, verify)}
		& SW13 \\ \hline
	MS C3 Security
		& \textbf{Date}
		& \multicolumn{2}{l|}{18th Dec 2016} \\
		& \textbf{Description}
		& \multicolumn{2}{l|}{secure communication between nodes} \\
		& \textbf{Workproducts}
		& \multicolumn{2}{l|}{Security} \\ \hline
	Transition 1
		& \multicolumn{2}{p{70mm}|}{polish documentation, write abstract, create poster, print documentation \& burn CDs}
		& SW14 \\ \hline
	MS T1 Delivery
		& \textbf{Date}
		& \multicolumn{2}{l|}{23rd Dec 2016 - 5:00 pm} \\
		& \textbf{Description}
		& \multicolumn{2}{l|}{Complete handover of thesis} \\
		& \textbf{Workproducts}
		& \multicolumn{2}{l|}{Thesis in paperform} \\ \hline

   \caption{Phase and iterations} \label{tab:phases}
   \end{longtable}
\end{center}
