% vim: ft=tex
\chapter{Discussion}
TODO something like a SWOT analysis here (strengths, weaknesses, opportunities, threats)\\
TODO general advice: be concise, brief, and specific\\

\section{Value Added}
TODO what's better than before\\

\section{Limitations}
TODO identify potential limitations and weaknesses of the product\\

BStar pair has to be complete (both nodes running) during initialization. Otherwise, only primary node can serve requests; the backup node can't.

Message traffic towards root node sums up because of persistence synchronization. This shouldn't be a problem because of \zmq's brilliant message batching, so the real limit is given by the inter-node network links.

\section{Business Benefits}
TODO potential applications (UeLS powered by Roadster?)\\

\section{Ideas for Improvement}
\begin{itemize}
	\item HA within a node: kill and respawn an actor when it's unresponsive
	\item switch to Moneta for a unified key-value store interface, then eventually away from TokyoCabinet to something more modern and maintained, like LMDB (it's super fast and crash-proof)
	\item TIPC: high performance cluster communication protocol, suitable because Roadster nodes are Linux and there are direct links to peers (required for TIPC)
	\item client authentication
	\item key management in a DB (instead of files), with GUI to accept new clients
	\item dynamic node topology (maybe via DSL-file in Etcd, or DIM-only, or Zookeeper)
	\item other method for data serialization (like MessagePack), would allow adding other programming languages to the cluster
	\item fast compression for messages, like LZ4 or Snappy
	\item SERVER/CLIENT sockets from ZMQ 4.2 for simplified message routing
\end{itemize}
