% vim: ts=2 sw=2 et
\lstloadlanguages{Ruby}
\lstset{language=Ruby}
\lstset{
  prebreak=\raisebox{0ex}[0ex][0ex]{\ensuremath{\hookleftarrow}},
  postbreak=\raisebox{0ex}[0ex][0ex]{\ensuremath{\rcurvearrowse\space}},
}

% custom style for Ruby listings
\lstdefinestyle{customruby}{
  belowcaptionskip=1\baselineskip,
  breaklines=true,
  frame=single,
  frameround=tttt,
  aboveskip=1\bigskipamount,
  belowskip=1\bigskipamount,
  xleftmargin=\parindent,
  language=Ruby,
  showstringspaces=false,
  basicstyle=\footnotesize\ttfamily,
  keywordstyle=\bfseries\color{green!40!black},
  commentstyle=\itshape\color{lightgray!40!black},
  identifierstyle=\color{NavyBlue},
  stringstyle=\color{red},
  captionpos=b,
}

% inline Ruby
% This style can be used within tabular. For some reason, the customruby style
% above won't work. But it has to be used like this (not using the \rb command):
%
% 	\lstinline[style=custominlineruby]{char}
%
\lstdefinestyle{custominlineruby}{
  basicstyle=\ttfamily,
  language=Ruby,
  keywordstyle=\bfseries\color{green!40!black},
  commentstyle=\itshape\color{lightgray!40!black},
  identifierstyle=\color{NavyBlue},
  stringstyle=\color{red},
  breaklines=true,
}

\newcommand{\rb}[1]{\lstinline[
  style=custominlineruby,
  basicstyle=\ttfamily,
  prebreak={},
  postbreak={}]{#1}
}

% custom style for shell command listings
\lstdefinestyle{customsh}{
  language=sh,
  showstringspaces=false,
  basicstyle=\footnotesize\ttfamily,
  breaklines=true,
  frame=single,
  frameround=tttt,
  aboveskip=1\bigskipamount,
  belowskip=1\bigskipamount,
  captionpos=b,
}

% inline shell
\newcommand{\sh}[1]{\lstinline[
  style=customsh,
  basicstyle=\ttfamily,
  prebreak={},
  postbreak={}]{#1}}

\definecolor{diffstart}{named}{Grey}
\definecolor{diffincl}{named}{Green}
\definecolor{diffrem}{named}{OrangeRed}

% diffs
\lstdefinelanguage{diff}{
  basicstyle=\ttfamily,
  frame=single,
  frameround=tttt,
  morecomment=[f][\color{diffstart}]{@@},
  morecomment=[f][\color{diffincl}]{+},
  morecomment=[f][\color{diffrem}]{-},
}
\lstdefinestyle{customdiff}{ % custom style for diff command listings
  language=diff,
  showstringspaces=false,
  basicstyle=\scriptsize\ttfamily,
  breaklines=true,
  captionpos=b,
}

\lstdefinelanguage{worddiff}{
  basicstyle=\ttfamily,
  morecomment=[s][\color{diffincl}]{\{+}{+\}},
  morecomment=[s][\color{diffrem}]{[-}{-]}
}
\lstdefinestyle{customworddiff}{ % custom style for git diff --word-diff listings
  language=worddiff,
  showstringspaces=false,
  basicstyle=\scriptsize\ttfamily,
  breaklines=true,
  captionpos=b,
}

% C/C++ listings
\lstloadlanguages{[11]C++}
\lstset{language=C++}
\lstset{
  prebreak=\raisebox{0ex}[0ex][0ex]{\ensuremath{\hookleftarrow}},
  postbreak=\raisebox{0ex}[0ex][0ex]{\ensuremath{\rcurvearrowse\space}},
}

\lstdefinestyle{customcpp}{
  belowcaptionskip=1\baselineskip,
  breaklines=true,
  frame=single,
  frameround=tttt,
  aboveskip=1\bigskipamount,
  belowskip=1\bigskipamount,
  xleftmargin=\parindent,
  language=C++,
  showstringspaces=false,
  basicstyle=\footnotesize\ttfamily,
  keywordstyle=\bfseries\color{green!40!black},
  commentstyle=\itshape\color{lightgray!40!black},
  identifierstyle=\color{NavyBlue},
  stringstyle=\color{red},
  captionpos=b,
}

% inline C++
% This style can be used within tabular. For some reason, the customcpp style
% above won't work. But it has to be used like this (not using the \cpp command):
%
% 	\lstinline[style=custominlinecpp]{char}
%
\lstdefinestyle{custominlinecpp}{
  basicstyle=\ttfamily,
  language=C++,
  keywordstyle=\bfseries\color{green!40!black},
  commentstyle=\itshape\color{lightgray!40!black},
  identifierstyle=\color{NavyBlue},
  stringstyle=\color{red},
  breaklines=true,
}

\newcommand{\cpp}[1]{\lstinline[
  style=custominlinecpp,
  basicstyle=\ttfamily,
  prebreak={},
  postbreak={}]{#1}
}

\lstdefinestyle{customgherkin}{
  basicstyle=\small,
  keywords={When, Then, Given, And, But},
  morekeywords={Feature, Scenario, Background},
  comment=[l]{\#},
  morestring=[b]',
  morestring=[b]"
}
