% vim: ft=tex
\chapter{Scope}
The technical goals of this bachelor thesis include extending mindclue GmbH's
Roadster framework by adding features such as clustering, high availability and
transport security. This chapter outlines the general scope of this project.

\section{Motivation}
% TODO Why do we care about this thesis? Why are we interested?\\

\subsection*{Backgrounds}
To better understand our motivation, it might help to understand our personal backgrounds first.

\begin{description}
	\item [Patrik Wenger]\hfill\\
		Having done his apprenticeship in computer science at Swisscom
		Schweiz AG, he continued to work as a full-time employee for
		five years afterwards. In programming he's most fluent in
		\gls{ruby} and \gls{c}. During the winter of 2015/2016, he
		created \gls{cztop} during leisure time because there was no
		good Ruby binding for \gls{zmq}/\gls{czmq} available and a side
		project of his demanded it.

		Fascinated with event-driven programming and software design
		patterns such as the \gls{actor-model}\footnote{known from
		Erlang, brought to Ruby by the Celluloid library, as well as the young
		programming language Pony which is completely based on actors},
		distributed computing and high availability have always been
		part of his core interests, especially in conjunction with the
		brilliant \zmq library.

		Having a passion for information security and state-of-the-art
		cryptography\footnote{such as \gls{nacl} or \gls{libsodium} as
		used by \gls{zmq}}, especially in this post-Snowden era, this
		bachelor thesis is like a dream come true.

	\item [Manuel Schuler]\hfill\\
		%TODO: Manuel: Javascript, .NET, horizont extension, learning new things, no hesitation for this project\\
		Always keen on learning new things, he did not hesitate to join
		this bachelor thesis at the first opportunity.
\end{description}

\noindent
In essence, we're both thrilled to gain more experience in the following
fields and technologies:

\begin{itemize}
	\item Distributed Computing
	\item High Availability
	\item Information Security
	\item \gls{actor-model}
	\item \gls{zmq}
	\item \gls{ruby}
\end{itemize}

\subsection*{Opportunities}
Coming from different backgrounds and having different levels of experience in
each of the above technologies, we can't wait to learn more about them and put
them to actual use. The fact that the product of this bachelor thesis is most
likely going to be used in the real world only adds to the excitement.

This bachelor thesis involves working with Ruby, the Actor Model, \zmq,
distributed computing with high availability, and state-of-the-art
cryptography. Furthermore, in case of successful completion of this thesis, the results will be used in real-world settings like the Ceneri
Base Tunnel. It is a huge opportunity for a solution completely based on free
and open-source software interacting with other industrial systems over open standards. The students, as well as the client, strongly believe
in customized solutions built on reusable, free open-source software.

In addition to that, we look at this bachelor thesis as an opportunity to
become more fluent in English, both written and spoken, as well as to improve
our skills in crafting scientific documents using {\LaTeX}.

Depending on how we perform together as a team, further collaboration might
result in the future, either between the students themselves, or between the
students and the client. Even if our paths will part, this project will
serve as a valuable reference for future job hunting.

Last but not least, we feel like Prof. Dr. Mehta is a respected and competent
teacher whose opinions we highly value. Due to his polite parlance, discussing
project matters, both of the management and the technical kind, has always been
an enrichment.

\subsection{Open-Source engagement}
Getting the chance to use \gls{cztop} and watch it perform definitely adds to
the motivation as well. Its software design has yet to be proven in more
serious settings.

Another personal goal is to create a reusable open-source library as a
byproduct. The intention is that the library makes certain \zmq-based
communication protocols readily available for other developers facing the same
problems.

\section{Initial Situation}

\subsection{mindclue GmbH}
The company mindclue GmbH, located in Ziegelbr\"ucke GL, provides its partner
REMTEC AG with complete \gls{SCADA} applications. These are then used to control and
monitor operation and safety equipment found in national freeways, water
supply systems, as well as in energy facilities and many other specialized
fields. To build these customized applications, their in-house creation
Roadster, a next-generation SCADA framework, is used.

\subsection{Roadster}
Roadster is a SCADA framework written in Ruby. It was, and still
is, developed to produce next-generation SCADA applications to replace legacy
solutions based on its predecessor found in numerous tunnel
facilities in Switzerland.

A Roadster installation combines the following responsibilities:

\begin{itemize}
	\item interaction with subordinate field devices (monitoring \& controlling)
	\item persisting data (e.g. certain sensor data, and events)
	\item sophisticated alarm (\emph{case}) management
	\item providing an interface (\gls{OPC} \gls{UA}) to superordinate systems
	\item providing a modern web UI for interaction with operational and executive personnel
\end{itemize}

Among others, subordinate field devices include \glspl{PLC} and emergency information systems.
These are interacted with over numerous propietary and/or
standardized protocols. Superordinate systems interact with Roadster over
protocols including \gls{SOAP} and \gls{OPC} \gls{UA}. They purpose is to
collect and aggregate information from larger regions. At the top of the
hiearchy are the ASTRA and MINSTRA which combine the information of all
subsystems and provide a nationwide overview.

\subsubsection{Typical hardware}
Roadster typically runs on entry-level rack server hardware powered by an
Intel\textregistered{} Xeon\textregistered{} processor, or industrial box PCs for smaller systems
commonly used for \gls{IoT} which are powered by more energy efficient processors
such as Intel\textregistered{} Core\textregistered{} and Intel\textregistered{}
Atom\texttrademark{}. The machines are usually equipped with 4 -- 6 GiB of main
memory and Gigabit Ethernet. For reliable systems without any moving parts, one
industrial grade \gls{SSD} or two (in a software \gls{RAID} level 1) setup are used.

\subsection{\zmq}
To understand Roadster's architecture and the rest of this document, it's
helpful to understand the basics of \zmq first. This is a brief introduction to
\zmq for the unfamiliar reader. What follows is a quote from the \gls{zguide}
which does a fairly good job at describing \zmq in a 100 words:

\begin{quote}
``ZeroMQ (also known as \zmq, 0MQ, or zmq) looks like an embeddable networking
library but acts like a concurrency framework. It gives you sockets that carry
atomic messages across various transports like in-process, inter-process, TCP,
and multicast. You can connect sockets N-to-N with patterns like fan-out,
pub-sub, task distribution, and request-reply. It's fast enough to be the
fabric for clustered products. Its asynchronous I/O model gives you scalable
multicore applications, built as asynchronous message-processing tasks. It has
a score of language APIs and runs on most operating systems.  ZeroMQ is from
iMatix and is LGPLv3 open source.''
\end{quote}

Roadster uses \zmq to carry messages between its processes.

For a more detailed introduction, see \autoref{ch:zmq}.

\subsection{Software architecture}
As mentioned earlier, Roadster is event-driven and built on the Actor model, meaning it exhibits a
shared-nothing architecture. Each Roadster node runs a number of Ruby processes
which communicate via \zmq sockets. The key here is communication:

\begin{quote}
``Don't communicate by sharing state; share state by communicating.''
\end{quote}

Running multiple, loosely coupled processes (actors) allows leveraging the full
potential of modern multi-core processors, while avoiding a whole class of
traditional concurrency problems.

Every Roadster node runs a group of actors:

\begin{description}
	\item [CORE:]
		It is responsible to start the other actors. It also plays a
		key role in keeping state in all actors synchronized, being the
		source of truth.

	\item [COMM:]
		A bunch of COMM actors communicate with the outside world of a
		node. It typically either acts as a client of various kinds of
		subordinate field devices, or as a server to superordinate
		systems. To communicate with subordinate systems, a COMM actor
		uses an adapter specifically written for the communication
		protocol in place.

	\item [STORAGE:]
		This actor is used when information needs to be persisted, such
		as time series or event journals. It's the interface to a
		key-value store.

	\item [LOGGER:]
		This actor collects logging data and sends it to whatever
		target is configured, be it STDOUT, a file, or a syslog server.
\end{description}

\autoref{fig:roadster:arch} illustrates Roadster's architecture.

\begin{figure}[]
	\includegraphics[trim=4cm 2cm 3.5cm 2.8cm, clip=true, width=\textwidth]{img/roadster_arch.pdf}
	\source{Andy Rohr}
	\caption{Roadster's software architecture}
	\label{fig:roadster:arch}
\end{figure}

\subsubsection{Communication Layers}
The communication architecture in Roadster consists of three layers, as
illustrated in \autoref{fig:roadster:layers}. The following list briefly
explains the layers from top (most abstracted) to bottom:

\begin{description}
	\item [Engine layer:]\hfill\\
		Here is the business logic of Roadster, e.g. the \gls{DIM},
		user authentication, adapters for different devices, the web
		\gls{UI}, etc.

	\item [Messaging layer:]\hfill\\
		The \gls{RMP} reside here and implement essential protocols used
		for logging, state synchronization, commands, application controlling,
		and storage. They're explained below in \autoref{sec:rmp}.

	\item [Reactor layer:]\hfill\\
		This layer forms the base, which is where the \zmq sockets and
		WebSockets used. In case of COMM actors, there can also be raw
		TCP sockets, e.g. to interact with certain \glspl{PLC}.
\end{description}

\begin{figure}[]
	\includegraphics[trim=1.95cm 2.5cm 1.65cm 2.8cm, clip=true, width=\textwidth]{img/roadster_layering.pdf}
	\source{Andy Rohr}
	\caption{Roadster's communication layers}
	\label{fig:roadster:layers}
\end{figure}

\subsubsection{RMP}\label{sec:rmp}
The \gls{RMP} are a collection of protocols implemented and used by Roadster
internally. They reside in the messaging communication layer, and include:

\begin{description}
	\item [\gls{CSP}:]\hfill\\
		Used to synchronize state between the actors.
	\item [\gls{ACP}:]\hfill\\
		Used to control the application state, e.g. things like shutdown.
	\item [\gls{PDP}:]\hfill\\
		Used when data needs to be persisted.
	\item [\gls{SMP}:]\hfill\\
		Used to suppress the generation of certain \glspl{case}, e.g.
		when a sensor is defect and repeatedly causes cases.
	\item [\gls{PCP}:]\hfill\\
		Used for asynchronous command execution via COMM peers with feedback.
	\item [\gls{LOG}:]\hfill\\
		Used for system logging.
\end{description}

Every actor in Roadster uses a subset of these protocols to perform its job.

Passing messages from actor to actor, which are nothing but serialized Ruby
objects, happens in one of two modes:
\begin{description}
\item [Fire \& Forget:]
No guarantee of correct processing, e.g. DIM updates from COMM to CORE. This
doesn't mean there are no other mechanisms in place to ensure reliability.

\item [Dialog:] An immediate answer is expected, e.g. when creating a user
session. Sending a message like this looks like it's a synchronous call, even
though it's handled asynchronously under the hood\footnote{This is done by
wrapping the affected code in a Ruby \rb{Fiber}, which is similar to a thread
but allows for cooperative scheduling as opposed to preemtive.} Any protocol
can make use of this primitive.
\end{description}

\subsubsection{DIM}
The \gls{DIM} is a data structure that lives inside every actor of a Roadster
node. Every actor builds it when starting up by reading the configuration
files. Updates to certain parts\footnote{Namely instances of the meta-model
classes \rb{Case}, \rb{DataItem}, and \rb{Session}} within it are replicated
across all actors. An updated item is marked dirty\footnote{It is marked dirty
by setting its \rb{@lifecycle_state = "updated"}} so it is subsequently synchronized
via the \gls{CSP}.

\subsubsection{Existing CSP in a nutshell}
\emph{This is a brief introduction/refresher for the Clone State Pattern
implemented by Roadster, which is used for the DIM synchronization. Although Roadster actually sends serialized instances
of CSP message classes to fulfill this protocol, for better readability the
\gls{zguide}'s canonical nomenclature of \gls{clone-pattern} messages will be used.}

The existing \gls{CSP} is closely related to the \gls{clone-pattern} from the \gls{zguide}. Its
goal is to keep a state (a list of key-value pairs) in sync across a set of
participants. To greatly reduce the complexity, it's not decentralized: There's
a server part which serves as the single source of truth.

The server uses a ROUTER, a PULL, and a PUB socket; each client a DEALER, a
PUSH, and a SUB socket. The protocol consists of three distinct messages flows:

\begin{description}
	\item [Snapshots:]
		Requesting and receiving the complete, current snapshot of the
		state (all key-value pairs). This happens via a
		ROUTER/DEALER pair of sockets. The request message consists solely of
		the humorously named ICANHAZ command. The response is the
		complete set of KVSET messages so a late-joining (or previously
		disconnected) client can rebuild the current snapshot.

	\item [Upstream updates:]
		Updates always originate from clients and are sent to the
		server via a PUSH/PULL pair of sockets. These are KVSET messages.

	\item [Downstream updates:]
		After being applied to the server's copy of the state,
		updates get a sequence number and are published back to all
		clients. This happens via the PUB socket and
		uses KVPUB messages.
\end{description}

By making all updates go through the server, a total order is enforced,
which is crucial to keep the state consistent across all clients.

To avoid risking a gap between requesting the current snapshot and subscribing
to updates, a client actually subscribes to the updates first, then gets the
snapshot, and then starts reading the updates from the socket (which has been
queueing updates in the meantime, if any). Updates that are older or the same
age as the received snapshot are skipped, and only successive updates are
applied (tested by comparing the sequence numbers).

Because message loss via the third message flow (PUB-SUB) is unlikely but
theoretically possible, the client checks for gaps in the sequence number of
each KVPUB message. If a gap is detected, the current state is discarded and a
complete resynchronization happens. This is brutal, but is very simple and thus
robust; there's no complexity that would leave room for nasty corner cases.

Keys can be treated hierarchically (e.g. \sh{topic.subtopic.key}) and thus, a
client can optionally subscribe to only a particular subtree. This is useful
when the number of client grows and not all of the state needs to be on every
client. In that case, the topic of interest is sent by the client along with
the ICANHAZ message.


\section{Goals}
% TODO preprocessed mandatory goals\\
% TODO retrospective, diff with initial goals\\

To summarize the mandatory goals from the Task Description in \autoref{ch:task-desc}:

\begin{enumerate}
	\item Getting familiar with Roadster
	\item Extending the communication protocols to support clustering
	\item Extending the communication protocols to add high availability
\end{enumerate}

The optional goals are:

\begin{enumerate}
	\item Encryption of the communication
	\item Providing of the highly available \gls{OPC} \gls{UA} server interface
\end{enumerate}

Secure inter-node communication within a Roadster cluster is important to
mitigate common security concerns with SCADA systems which are becoming more
and more open due to standardization. To quote wikipedia:

\begin{quote}
``In particular, security researchers are concerned about:
	\begin{itemize}
		\item the lack of concern about security and authentication in the design, deployment and operation of some existing SCADA networks
		\item the belief that SCADA systems have the benefit of security through obscurity through the use of specialized protocols and proprietary interfaces
		\item the belief that SCADA networks are secure because they are physically secured
		\item the belief that SCADA networks are secure because they are disconnected from the Internet.''
	\end{itemize}
\end{quote}
