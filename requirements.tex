% vim: ft=tex
\chapter{Requirements}
The requirements gathered during the first meeting with the client are
explained in this chapter. These are more concrete than the ones listed in the
Task Description in \autoref{ch:task-desc}.

First of all, these are the priorities from the client's point of view in
descending order:

\begin{enumerate}
\item Clustering
\item Single-level \gls{HA}
\item Multi-level \gls{HA}
\item Persistence synchronization
\item Security (optional)
\item OPC UA \gls{HA} (optional)
\end{enumerate}

The following sections explain the requirements in greater detail.

\section{Functional}
Below are the functional requirements, as opposed to the non-functional
requirements.


\subsection{Cluster}
This requirement is to allow running Roadster on multiple nodes in a
hierarchical topology.

Extending the \gls{CSP} to keep the \gls{DIM} in sync across all nodes is a central
part of the clustering functionality. This means synchronizing the
items within the DIM that have \rb{@lifecycle_status = "updated"}. According to
the Data Model class diagram, these should only be instances of one of three
classes (marked yellow in the diagram):
\begin{itemize}
	\item \rb{DataItem}
	\item \rb{Session}
	\item \rb{Case}
\end{itemize}

In addition to that, there needs to be some kind of message routing so a user
of one node's web UI can send a command to another node where it will be
executed. An example for this is a forced value in the DIM to ignore the
actually measured value reported by a device in case the device is known to be
wrong.

It's important that every node subtree can live on autonomously even if the
link to its supernode or the supernode itself fails.

The above requirements imply that changes to the DIM can only be done by the
respective node. In other words, a node can only change its own values. It
cannot change values of supernodes, nor is it allowed to change values of
subnodes directly. This is to ensure that each node is its own source of truth
for all other nodes in the setup.


\subsubsection{Typical Usecases}
The typical use cases include:

\begin{description}
	\item [ Single level, single node ]
		This is the legacy setup and is what Roadster is already able to do.

	\item [ Single level \gls{HA} ]
		This is when there are exactly two nodes, both of them
		connected to the same PLC. There are two nodes for redundancy.

	\item [ Multi level, \gls{HA} at root only ]
		There can be multiple levels in the hierarchy, such as two or
		three (anything else is considered exotic), and there's a HA
		cluster at the root.
\end{description}

The exotic cases which can be ignored, if need be, include:

\begin{itemize}
	\item Multi level, HA at bottom
	\item Multi level, HA in the middle
\end{itemize}


\subsection{Single Level HA}
This is where there's a node pair directly connected to a PLC. Both nodes have
read\slash write access to the PLC, but only one of the nodes (the active one)
must do so. The nodes must automatically find consensus on who's active. The
passive one must automatically take over in case the active one is confirmed to
be dead.\\

The kinds of failures that need to be handled include:
\begin{itemize}
	\item Hardware/software failure on the primary node
	\item Failure of the link between the two nodes
\end{itemize}



\subsection{Multi Level HA}
This is where a node pair is the parent of one or more subnodes.

The kinds of failures that need to be handled include:
\begin{itemize}
	\item Hardware/software failure on the primary node
	\item Failure of the link between the two nodes
\end{itemize}

\subsection{Persistence Synchronization}
This is about the synchronization of the TokyoCabinet databases. Data flow is
from south to north (towards the root node), so the root node collects and
maintains a replication of the persisted data of all subnodes, recursively.

It's important that every node and its subnodes form an autonomous subtree. So
in case the link to its supernode fails, it has to continue working. As soon as
the link is repaired, the delta of the newly added data to the database can be
bubbled up. Sending just the delta should be possible since keys in the
database contain timestamps.

This is not the same as \gls{CHP} (\gls{DIM} synchronization), as the DIM is shared across
all nodes and is a relatively small data structure. The TokyoCabinet databases
can possibly contain large amounts of data (in the hundreds of megabytes) and
are shared only towards the root node (thus "bubbling up").

100\% consistency is not an absolute requirement for persistence synchronization.

\subsection{Security}
TODO common concern about SCADA systems lacking security\\

\begin{itemize}
	\item transport needs to be secure (encrypted and authenticated)
	\item TODO verify requirements with Andy (we didn't really discuss this
		during the meeting)
	\item this requirement comes as the last mandatory goal not because
		it's insignificant, but because it's easy to enable transport
		level security on ZMQ sockets, and it would just interfere with
		the previous development
\end{itemize}

\subsection{OPC UA HA}
\begin{itemize}
	\item provide standardized interface upwards from HA pair
\end{itemize}

% ---------------------------------------------------------------------------
\section{Use Cases}
TODO maybe there are any?

% ---------------------------------------------------------------------------
\section{Non-Functional Requirements}
\subsection{Simplicity}
The two reoccuring patterns that surfaced during the requirements gathering
meeting were:

\begin{enumerate}
\item \gls{KISS} principle. Simplicity is favored, as experience shows that
	simpler systems are more stable, so complexity should be avoided if not
	absolutely necessary.

\item No premature optimization since it's the root of all evil.\footnote{Quote
	by Donald Knuth: ``Premature optimization is the root of all evil.''}
\end{enumerate}


\subsection{Testing}
\begin{itemize}
	\item we write unit tests for our own contributions
	\item we test the integrated result in a close-to-reality setup
\end{itemize}


\subsection{Coding Guidelines}
The coding guidelines desired by the client are basically the ones written down
in the popular Ruby style guide \cite{rb:style-guide}, with the following
differences or special remarks:

\begin{itemize}
	\item method calls: only use parenthesis when needed, even with arguments (as opposed to \footnote{\url{https://github.com/bbatsov/ruby-style-guide\#method-invocation-parens}})
	\item 2 blank lines before method definition (slightly extending \footnote{\url{https://github.com/bbatsov/ruby-style-guide\#empty-lines-between-methods}})
	\item YARD API doc, 1 blank comment line before param documentation, one blank comment line before code (ignoring \footnote{\url{https://github.com/bbatsov/ruby-style-guide\#rdoc-conventions}})
	\item Ruby 1.9 symbol keys are wanted (e.g. \rb{foo: "bar", baz: 42} instead of \rb{:foo => "bar", :baz => 42}, just like \footnote{\url{https://github.com/bbatsov/ruby-style-guide\#hash-literals}})
	\item align multiple assignments so there's a column of equal signs
\end{itemize}
