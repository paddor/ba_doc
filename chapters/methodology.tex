% vim: ft=tex
\chapter{Methodology}
TODO what have we done to arrive at the goal (should be reproducible)\\
TODO this is probably what we know as "Concept"\\



\section{Port to new ZMQ library}\label{sec:meth:port}
TODO justify why port is needed right at the beginning (exclude faults from unmaintained ffi-rzmq gem, encryption is needed anyway, all the other tasks involve communication over ZMQ)\\
TODO explain binding options out there, why CZTop (including difference between ZMQ and CZMQ)\\
TODO explain preliminary task of adding support for the ZMQ options FD and EVENTS in CZTop\\
TODO explain concept of exchanging ffi-rzmq with CZTop\\


\section{Cluster}\label{sec:meth:cluster}
TODO explain planned multi node setup\\
TODO election/design of appropriate protocol\\
TODO explain Clustered Hashmap Protocol (I guess)\\


\begin{itemize}
	\item PCP: use DIM to know node tree and determine next hop for (dialog or fire+forget) messages
	\item decide on sync variant
	\begin{itemize}
		\item variant 1
		\begin{itemize}
			\item always sync on self-subtree only
			\item con: no copy of remaining tree
		\end{itemize}

		\item variant 2:
		\begin{itemize}
			\item always sync on complete tree
			\item get snapshot and merge own subtree
		\end{itemize}

		\item variant 3:
		\begin{itemize}
			\item make it configurable: either sync on subtree or complete tree
		\end{itemize}
	\end{itemize}

	\item node topology in DSL, static file shared on all nodes, read by each actor on startup
	\item specific config file on each node (conf.rb) knows its own place in topology\\
		\sh{conf.system_id = "nodes.root" # no HA}\\
		OR\\
		\sh{conf.system_id = "nodes.root_ha.foo" # with root HA, subnode A directly below root level}

	\item maybe a HA pair is one DIM object, has one name, but two IP addresses (primary and backup, in order)
\end{itemize}


\section{High Availability}\label{sec:meth:ha}
TODO we have two different kinds of HA\\
TODO explain how the failures we're required to be able to handle can be handled\\
TODO expalin similarities between the two kinds of HA\\

\subsection*{Single Level}
\begin{itemize}
	\item this is different from what's described in the zguide because the concept of client requests is missing here (PLCs don't request anything)
	\item life sign from one node to the other through some continually updated PLC value
	\item mark active HA peer in DIM, OR PUSH-PULL \& different route back
	\item side note: PUSH-PULL is probably not feasible, because message are sent to inactive pull anyway, until queue full
\end{itemize}

\subsection*{Multi Level}
TODO explain why is this one different from SL-HA\\
TODO Finding consensus should be easier here, as it's closely related to the CHP described in the zguide.\\

\section{Persistence Synchronization}\label{sec:meth:psync}
\begin{itemize}
	\item super node requests for delta of TC periodically
\end{itemize}

\section{Security}\label{sec:meth:security}
TODO briefly describe ZMQ's security features, what's left for us to decide (key destribution)\\
TODO how it can be verified (-> using wireshark)\\




\section{OPC UA Interface: High Availability}\label{sec:meth:opc-ua}
TODO This is the optional goal.\\
TODO explain new opportunity for OPC UA HA server\\
TODO describe whatever needs to be described\\

\begin{itemize}
	\item study standard
	\item use Andy's gem
	\item according to Andy, this should be a simple thing
\end{itemize}
