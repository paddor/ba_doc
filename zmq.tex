% vim: ft=tex
\chapter{\zmq}\label{ch:zmq}
\zmq is a \gls{MOM} implemented as an open source library, that is, it doesn't
require a dedicated broker. Instead, it offers sockets with an abstract
interface similar to \acrshort{BSD} sockets. Different types of sockets are used for
different messaging patterns such as request-reply, publish-subscribe, and
push-pull.

A single socket can bind/connect to multiple endpoints, which allows \zmq to
use round-robbin on the sender side, and fair-queueing on the receiver side,
where applicable. It doesn't matter whether the communication happens
in-process (between threads), inter-process (e.g. over \glspl{unix-domain-socket}), or
inter-node (e.g. over \acrshort{TCP}/\acrshort{PGM}/\acrshort{TIPC}), since the transport is completely
abstracted away. The same goes for connection handling; an arbitrary amount of
connections is handled over a single socket and reconnecting after short
network failures is done transparently.

\zmq is lightweight and allows for extremely low latencies, which means it can
also be used as the fabric of concurrent applications, e.g. for the \gls{actor-model}.
In case of the TCP transport, it incorporates advanced techniques such
as smart message batching to achieve significantly higher throughputs than with
raw TCP or other \gls{MOM} solutions \cite[Figure 2, Middleware evaluation and
prototyping, p.~4]{cern:new-cmw}.

To build a solution with \zmq, its sockets are used as building blocks to
design custom message flows. Certain patterns are used to achieve reliability
with respect to the failure types that need to be addressed in particular.  The
\gls{zguide} explains best practices, including commonly needed, resilient
messaging patterns.

The above characteristics make \zmq a valuable asset when it comes to building
robust, distributed high-performance systems.

\section{Transport security}\label{sec:zmq:security}
Since version 4.0 (released in October 2013), \zmq boasts strong encryption and
authentication, based on the excellent and highly renown
\gls{libsodium}\footnote{There's also the possibility to do it using
\gls{tweetnacl} which avoids the additional dependency.}. The protocol used
(CurveZMQ) is described in \cite{zmq:curvezmq}.

Transport encryption is completely transparent to the application . The
security handshake is designed to be highly resilient against
\glspl{ddos-attack} attacks using encrypted and authenticated cookies, so the
server doesn't actually allocate any memory before the handshake is completed,
and key generation is very cheap \cite[p.~2]{djb:ed25519} with
Curve25519\footnote{\url{https://en.wikipedia.org/wiki/Curve25519}} \gls{ECC}.

To enable, one socket is designated as the server, the other as the client.
Server authentication is always performed. This means that nodes running the
client sockets must be in possession of the server's public key.

Client authentication is optional. If desired --- meaning not just any client
is allowed to connect --- the clients' public keys (either shared by all
clients, or a unique key for each client) must be already available to the node
running the server sockets. The server socket actually talks to another,
designated socket to do authentication. The protocol used between them is
\gls{ZAP}. Completely abstracting the authentication in another socket allows
any kind of authentication service to be easiliy plugged in. It's also trivial
to write such a ZAP handler to register new client public keys somewhere to be
checked and confirmed by a human, which simplifies the process of adding new
client public keys.

\section{Data serialization}
Data serialization is outside the scope of \zmq. To fill the gap, one typically
uses another library such as MsgPack\footnote{\url{http://msgpack.org}},
Protocol
Buffers\footnote{\url{https://developers.google.com/protocol-buffers/}}, or
even a programming language's built-in object serialization
support\footnote{such as Ruby's marshalling support:
\url{http://ruby-doc.org/core/Marshal.html}}.

\section{Language availability}
Bindings or full-blown reimplementations of \zmq exist for a plethora of
programming languages, including C, C++, Java, Ruby, Perl, Erlang, Python, Lua,
Go, Tcl, PHP, and .Net. This allows for building distributed software systems
using modern approaches like service-oriented architectures, where different
parts may be implemented in different languages.

\section{CZMQ}
\gls{czmq} is a high-level abstraction layer for \zmq. It makes working with the \zmq
library more expressive and allows for better portability. It also provides
additional functionality such as a reactor, a simple actor implementation, as
well as utilities for certificate and authentication handling, and LAN node
discovery. This is the recommended way of using \zmq nowadays, as it allows for
much cleaner C code and also simplifies bindings for other languages.

% TODO: CZTop
