% vim: ft=tex
\chapter{Context}
\section{Initial Situation}
Roadster is mindclue GmbH's in-house framework to build modern monitoring and
controlling applications in different fields such as traffic systems, energy,
and water supply. It is written in Ruby, a modern and expressive scripting
language, and is built on a shared-nothing architecture to avoid a whole class
of concurrency and scalability issues found in traditional application
architectures.

Although considered to be the next generation of its kind, it still lacks
important features such as the ability to be run as a cluster on multiple
nodes, high availablity, and secure network communications.

\section{Goals}
Adding the aforementioned, missing features to form the next version of the
framework would mean a distinct advantage for mindclue GmbH and thus increase
its competitiveness in its sector.

Planning the exact architectural changes and additions, as well as performing
the implementation is the students' goal for this bachelor thesis. Using
engineering methodology practiced at HSR, solutions for particular problems
will be worked out and the best fitting one will be chosen.

Although not exactly part of the requirements, spreading knowledge about Roadster's
architecure and code basis is also in the interest of the client, as Andy Rohr
is currently the framework's only developer and thus a single point of failure
in an increasingly important piece of software.

\section{Software Development Process}
The \gls{RUP} is used to plan and manage this term project. It’s an iterative,
structured, yet flexible development process which suits this kind of project.
At HSR, it’s taught as part of the Software Engineering courses and is thus a
primary candidate.

Another candidate was Scrum, which we decided against as it’s only feasible
with teams of three to nine developers.

\section{Project Management Infrastructure}
The source code of this document and all of our code contributions are hosted
on GitHub. The students will organize and perform their work directly on the
site as far as possible. This means creating a Project board for each of the
development phases, creating, assigning, and closing issues, as well as using
the Wiki feature to plan and document meetings with the professor and the
client.

Time tracking, as required by the process for bachelor theses
\cite{hsr:thesis-rules}, are done externally on Everhour.

\section{Personal Goals}
Next to excelling in this bachelor thesis, the aim is also to create a
reusable open-source library as a byproduct. The intention is that the library
makes certain \zmq-based communication protocols readily available for other
developers facing the same problems.

% TODO: section: quality assurance

\chapter{Project Phases}
TODO describe this phase in retrospection\\


\section{Plan}
% TODO: one big table with the whole project plan (from wiki)
% TODO: risk table & matrix

\section{Inception}
TODO describe this phase in retrospection\\

\section{Elaboration}
TODO include Gantt chart for this phase\\
TODO describe this phase in retrospection (risk eliminiation)\\

\section{Construction}
TODO include Gantt chart for this phase\\
TODO describe this phase in retrospection (risk eliminiation)\\

\section{Transition}
TODO include Gantt chart for this phase\\
TODO describe this phase in retrospection (risk eliminiation)\\

\chapter{Results}
TODO describe results\\
