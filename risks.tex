\section{Risks}
There were X risks that have been identified by the end of the Inception phase.
The risks have a total damage of xxx hours. The total damaged hours multiplied
with the probability of admission get a total of xx hours. The weighted damage
hours are included in the project planning.

\section{Handling Risks}
Due to the nature of the risks, it is only natural they change during the course of a project.
To mitigate this, the risks are checked regularly (in weekly meetings) using the table below.

Changes to existing risks are possible. This usually means that either the likelihood or
the unweighted damage must be adapted immediately. Moreover, it's possible for a risk to
be completely ruled out, or that a new risk arises. All these points need to be discussed in
the team and tracked accordingly.

\section{Listed Risks}
\begin{tabular}[t]{@{}>{\raggedright}p{0.45\textwidth}}
  \textbf{\textit{P = Probability}}
  \begin{enumerate}[topsep=0pt,itemsep=-2pt,leftmargin=13pt]
  \item Unlikely
  \item Very rare
  \item Rare
  \item Possible
  \item Common
  \end{enumerate}
\end{tabular}
\begin{tabular}[t]{@{}>{\raggedright}p{0.52\textwidth}@{}}
  \textbf{\textit{D = Damage potential  / R = Risk}}
  \begin{enumerate}[topsep=0pt,itemsep=-2pt,leftmargin=13pt]
  \item Insignificant
  \item Low
  \item Significant
  \item Critical
  \item Project Threatening
  \end{enumerate}
\end{tabular}

The delay is specified in days. One day equals 16 man-hours.

\begin{center}
  \begin{longtable}{|p{6mm}|p{30mm}|p{6mm}|p{8mm}|p{30mm}|p{64mm}|}
    \caption[Initial Risks]{Initial Risks} \label{tbl:risks} \\

    \hline \multicolumn{1}{l}{\textbf{ID}} &
    \multicolumn{1}{l}{\textbf{Description}} &
    \multicolumn{1}{l}{\textbf{P}} &
    \multicolumn{1}{l}{\textbf{DP}} &
    \multicolumn{1}{l}{\textbf{Prevention}} &
	\multicolumn{1}{l}{\textbf{Measures to be taken upon event}} \\ \hline
    \endfirsthead

    \multicolumn{6}{c}%
    {{\bfseries \tablename\ \thetable{} -- continues}} \\
    \hline \multicolumn{1}{c}{\textbf{ID}} &
    \multicolumn{1}{c}{\textbf{Description}} &
    \multicolumn{1}{c}{\textbf{P}} &
    \multicolumn{1}{c}{\textbf{DP}} &
    \multicolumn{1}{c}{\textbf{Prevention}} &
	\multicolumn{1}{c}{\textbf{Measures to be taken upon event }} \\ \hline
    \endhead

    \hline \multicolumn{5}{r}{{Continues on the next page}} \\ \hline
    \endfoot

    \hline
    \endlastfoot
    R01
		& Roadster requires different ZMQ contexts to function (not possible with CZTop because CZMQ hides contexts)
		& \cellcolor{yellow!50}1
		& \cellcolor{yellow!50}3
		& check with client	(done)
		& extract and run affected ZMQ \newline sockets in their own process \newline delay: 1-2 days \\ \hline
	R02 
		& wrong protocols chosen / protocol design flaw
		& \cellcolor{red!50}3
		& \cellcolor{red!50}5
		& architecture \newline reviews prototypes
		& fix (reevaluate reengineer, redesign) \newline delay: 8-12 days	\\ \hline
	R03 
		& ZMQ communication patterns (such as Binary Star) are difficult to implement as clean, reusable code
		& \cellcolor{yellow!50}2 
		& \cellcolor{yellow!50}4 
		& use software engineering knowhow to aim for clean, reusable prototypes
		& nice solution: \newline build more iteratively, step by step \newline delay: 2-4 days \newline \newline dirty solution: 
		\newline customized solution built right into Roadster, not as a public gem \newline delay: 1-2 days \\ \hline
	R04	
		& CZTop design flaws/limitations
		& \cellcolor{yellow!50}2
		& \cellcolor{yellow!50}2
		& check functionality in the elaboration phase
		& adapt CZtop \newline delay: 1-2 days \\ \hline		
	R05 
		& CZMQ changes API
		& \cellcolor{yellow!50}1
		& \cellcolor{yellow!50}3
		& (hope)
		& adapt CZTop, change CZTop adapter in Roadster, or just don't upgrade CZMQ (use a commit before the breaking change) \newline delay: 1-2 days \\ \hline
	R06 
		& wrong time estimations
		& \cellcolor{orange!50}4
		& \cellcolor{orange!50}3
		& use time well during planning, and define clear milestones
		& If possible, change the duration of the individual project phases. Otherwise, drop planned features
		(starting with the optional goal) \newline delay: 4-5 days \\ \hline		
	R07 
		& managing multiple Projects (at least one per repo) on Github too painful
		& \cellcolor{yellow!50}4
		& \cellcolor{yellow!50}1
		& setup project structure in the elaboration phase
		& partial solution: \newline CodeTree (can't seem to be used for private repos like Roadster itself (maybe yes! see  mindclue/roadster\#5) \newline
		\newline complete solution: \newline Use a single Project which just has cards that link to issues from other repos. 
		Linking to "foreign" issues is additional effort but should be straight forward using Github syntax (https://github.com/org/repo/issues/42)
		\newline delay: 1 day \\ \hline
	R08 
		& Prolonged loss of a team member
		& \cellcolor{yellow!50}2
		& \cellcolor{yellow!50}3
		& Track absences in meeting minutes.
		& In a prolonged absence, move milestones and, if necessary, change the project scope. \newline delay: 3-10 days \\ \hline
		
	R09 
		& Failure to achieve the defined task in time
		& \cellcolor{yellow!50}1
		& \cellcolor{yellow!50}4
		& Continuous monitoring whether we are on schedule and whether all requirements are met. 
		& Meeting convened as we still can transpose a large part of the required task within the prescribed period. \newline delay: 1-5 days\\ \hline		
   \end{longtable}
\end{center}

\begin{table}[H]
  \centering
  \scriptsize
  \caption{Initial Risk Matrix}
  \begin{tabular}{|m{27mm}|m{24mm}|m{20mm}|m{20mm}|m{20mm}|m{20mm}|@{}m{0pt}@{}}
    \hline 	\bf Propability / Damage & \bf 1-Insignificant. 	& \bf 2-Low 				& \bf 3-Significant 			& \bf 4-Critical 			& \bf 5-Project Threatening 	& \\ [10pt]
    \hline 	\bf 5-Common 			& \cellcolor{yellow!50} 	& \cellcolor{orange!50} 	& \cellcolor{red!50} 			& \cellcolor{red!50} 		& \cellcolor{red!50} 			& \\ [10pt]
			\bf 4-Possible 			& \cellcolor{green!50} R07 	& \cellcolor{yellow!50} 	& \cellcolor{orange!50} R06 	& \cellcolor{red!50} 		& \cellcolor{red!50} 			& \\ [10pt]
			\bf 3-Rare 				& \cellcolor{green!50} 		& \cellcolor{green!50} 		& \cellcolor{yellow!50} 		& \cellcolor{orange!50} 	& \cellcolor{red!50} R02 		& \\ [10pt]
			\bf 2-Very rare 			& \cellcolor{green!50} 	& \cellcolor{green!50} R04 	& \cellcolor{green!50} R08 		& \cellcolor{yellow!50} R03 & \cellcolor{orange!50} 		& \\ [10pt]
			\bf 1-Unlikely 			& \cellcolor{green!50} 		& \cellcolor{green!50} 		& \cellcolor{green!50} R01, R05 & \cellcolor{green!50} R09 	& \cellcolor{yellow!50} 		& \\ [10pt]
    \hline
  \end{tabular} \\
\end{table}

\begin{ganttchart}[
  hgrid, 
  vgrid, 
  x unit=9mm
]{1}{14}
\ganttset{bar incomplete/.append style={fill=gray!40},
  group/.append style={draw=black, fill=gray},}
\gantttitle{Calendar weeks}{14} \\
\gantttitlelist{38,...,51}{1} \\
\gantttitle{Semester week}{14} \\
\gantttitlelist{1,...,14}{1} \\
\ganttgroup{Inception}{1}{1} \\
\ganttgroup{Elaboration 1}{2}{3} \\
\ganttgroup{Elaboration 2}{4}{5} \\
\ganttgroup{Elaboration 3}{6}{6} \\
\ganttgroup{Construction 1}{7}{8} \\
\ganttgroup{Construction 2}{9}{10} \\
\ganttgroup{Construction 3}{11}{12} \\
\ganttgroup{Reserve 1}{13}{13} \\
\ganttgroup{Transition}{14}{14} \\
\ganttbar[bar/.append style={fill=green}]{R01}{1}{3}
\ganttbar[bar/.append style={fill=red}]{R02}{1}{3}\ganttbar[bar/.append style={fill=yellow}]{}{3}{4}\ganttbar[bar/.append style={fill=green}]{}{4}{5} \\
\ganttbar[bar/.append style={fill=yellow}]{R03}{1}{4}\ganttbar[bar/.append style={fill=green}]{}{4}{5} \\
\ganttbar[bar/.append style={fill=green}]{R04}{1}{4} \\
\ganttbar[bar/.append style={fill=green}]{R05}{1}{3} \\
\ganttbar[bar/.append style={fill=orange}]{R06}{1}{5}\ganttbar[bar/.append style={fill=green}]{}{5}{7} \\
\ganttbar[bar/.append style={fill=green}]{R07}{1}{4} \\
\ganttbar[bar/.append style={fill=green}]{R08}{1}{13} \\
\ganttbar[bar/.append style={fill=green}]{R09}{1}{13} \\
%\ganttmilestone{Milestone 1}{11}
\end{ganttchart}

\begin{table}[H]
  \centering
  \caption{Risk-Timeline change protocol}
  \begin{tabular}{|p{20mm}|p{20mm}|p{102mm}|}
    \hline \bf Week & \bf Risk & \bf Description \\% [10pt]
    \hline 3 & R01 & .... \\% [10pt]
    \hline 3 & R06 & .... \\% [10pt]
    \hline
  \end{tabular} \\
\end{table}
