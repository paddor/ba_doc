% vim: ft=tex
\chapter{Results}\label{ch:res}
This chapter documents the final results.

\section{System tests}
All functional, required features have been implemented, as shown in \autoref{tab:systemtestresults}.
\begin{table}[H]
  \centering
  \begin{tabular}{|m{5mm}|m{5mm}|m{5mm}|m{50mm}|}
    \hline
    \bf C1 & \bf C2 & \bf C3 & \bf Feature \\
    \hline
	  \bf \color{green!65!black}\cmark & \color{green!65!black}\cmark & \color{green!65!black}\cmark & Federation \\
    %\hline
    \bf \color{green!65!black}\cmark & \color{green!65!black}\cmark & \color{green!65!black}\cmark & DIM extension \\
    %\hline
    \bf \color{green!65!black}\cmark & \color{green!65!black}\cmark & \color{green!65!black}\cmark & Autonomy \\
    %\hline
    \bf \color{green!65!black}\cmark & \color{green!65!black}\cmark & \color{green!65!black}\cmark & Message routing \\
    %\hline
    \bf \color{red!50}\xmark & \color{green!65!black}\cmark & \color{green!65!black}\cmark & High Availability \\
    %\hline
    \bf \color{red!50}\xmark & \color{green!65!black}\cmark & \color{green!65!black}\cmark & Persistence synchronization \\
    %\hline
    \bf \color{red!50}\xmark & \color{red!50}\xmark & \color{green!65!black}\cmark & Encryption (optional) \\
    %\hline
    \bf \color{red!50}\xmark & \color{red!50}\xmark & \color{green!65!black}\cmark & OPC-UA (optional) \\
    \hline
  \end{tabular} \\
  \caption{Systen tests results}
  \label{tab:systemtestresults}
\end{table}

\section{Unit tests}
All new contributions are 100\% covered by unit tests.

\section{Integration tests}
Existing integration tests have been refactored.

\section{Non-functional requirements}
All non-functional requirements, including the Ruby style desired by mindclue
GmbH, have been achieved.


\section{Programming statistics}

\autoref{tab:programmingstatistics} shows the programming statistics.
\begin{table}[H]
  \centering
  \begin{tabular}{|m{50mm}|m{30mm}|}
   \hline
	\bf Ruby LoC before & 7161 \\
	\hline
	\bf Ruby LoC after & 9123 \\
	\hline
	\bf Ruby LoC & 1962 \\
	\hline
	\bf Ruby Classes before BA & 187 \\
	\hline
	\bf Ruby Classes after BA & 223 \\
	\hline
	\bf New Ruby Comments & 1468 \\
	\hline
	\bf New Ruby files & 23 \\
	\hline
	\bf Python LoC & 684 \\
	\hline
	\bf New Python files & 14 \\
	\hline
	\bf Number of commits in 
		\newline roadster repository & 579 \\
	\hline
	\bf Number of commits in 
		\newline ba-roadster-app repository & 26 \\
	\hline
	\bf Number of line changes 
		\newline roadster repository & 18749{\color{green!70} ++} / 8063{\color{red!70} -{}-} \\
	\hline
	\bf Number of line changes 
		\newline ba-roadster-app repository & 61{\color{green!70} ++} / 364{\color{red!70} -{}-} \\
	\hline
	\bf Number of new branches & 14 \\
	\hline
	\bf Number of new unit-tests & 803 \\
	\hline
	\bf Number of features & 7 \\
	\hline
	\bf Number of scenarios & 15 \\
	\hline
	\bf Number of system-tests & 104 \\
    \hline
  \end{tabular} \\
  \caption{Programming statistics}
  \label{tab:programmingstatistics}
\end{table}
%TODO cztop, federation, federation_csp, message_routing, ping_pong, webui_csp_fix, case_handling_fix, model_ownership, profiling, optimizations, msgpack_codec, node_as_root, high_availability, encryption
\subsection{Metrics}
% TODO  * number of new classes
% TODO  * flog/reek/rubocop/flay/metric_fu... metrics
% TODO short description ...
% ABC metric Assignments, Branches, Conditions. An otherway for countig something like LoC. Lower values means the code is short and easily to read.
The follow listing shows the calculated ABC metric values. Higher means more complex code. It is an modern approach to LoC or Function Point.
\begin{listing}[H]
	\begin{minted}[bgcolor=bg]{text}
mschuler@mschuler-vb:/tmp/roadster/lib$ flay  -s .
Total score (lower is better) = 2162

  225.00: ./roadster/messaging/handlers/webui.rb
  172.00: ./roadster/messaging/handlers/core.rb
   95.00: ./roadster/messaging/handlers/bstar.rb
   90.00: ./roadster/engines/modules/report/eventjournal.html.erb
   88.00: ./roadster/messaging/handlers/storage.rb
   85.00: ./roadster/messaging/handlers/downstream.rb
   77.00: ./roadster/messaging/handlers/upstream.rb
   77.00: ./roadster/messaging/handlers/comm.rb
   72.00: ./roadster/adapters/protocols/iec104/info_objects/point_information.rb
   72.00: ./roadster/engines/core.rb
   70.00: ./roadster/messaging/protocols/csp/api.rb
   68.00: ./roadster/messaging/messages/base.rb
   64.00: ./roadster/engines/modules/database/tc_eventjournal_db.rb
   57.00: ./roadster/adapters/protocols/modbus/pdu/write_multiple_registers.rb
   57.00: ./roadster/adapters/protocols/modbus/pdu/read_holding_registers.rb
   57.00: ./roadster/actors/downstream.rb
   56.00: ./roadster/engines/modules/domain/host/host.rb
   54.00: ./roadster/messaging/protocols/pcp/api.rb
   54.00: ./roadster/engines/modules/domain/model/case.rb
   52.00: ./roadster/engines/comm.rb
   51.00: ./roadster/messaging/handlers/logging.rb
   38.00: ./roadster/messaging/protocols/usp/api.rb
   38.00: ./roadster/messaging/protocols/smp/api.rb
   38.00: ./roadster/actors/upstream.rb
   36.00: ./roadster/engines/modules/database/tc_base_db.rb
   35.00: ./roadster/actors/core.rb
   35.00: ./roadster/actors/storage.rb
   32.00: ./roadster/messaging/protocols/pdp/messages.rb
   32.00: ./roadster/messaging/protocols/smp/messages.rb
   32.00: ./roadster/messaging/protocols/pdp/api_timeseries.rb
   22.00: ./roadster/adapters/protocols/iec104/info_objects/info_object_70.rb
   22.00: ./roadster/adapters/protocols/iec104/info_objects/info_object_100.rb
   20.00: ./roadster/engines/webui.rb
   19.00: ./roadster/messaging/protocols/pdp/api_parameters.rb
   19.00: ./roadster/messaging/protocols/usp/messages.rb
   19.00: ./roadster/messaging/protocols/pcp/messages.rb
   16.00: ./roadster/messaging/protocols/cmp/messages.rb
   16.00: ./roadster/actors/comm.rb
	\end{minted}
	\caption{flay result}
	\label{lst:metrics:flay:result}
\end{listing}

\subsection{Collaboration statistics}
\autoref{tab:collaborationstatistics} shows the collaboration statistics.
\begin{table}[H]
  \centering
  \begin{tabular}{|m{60mm}|m{10mm}|}
	\hline
	\bf Slack messages with the projectpartner & 3700 \\
	\hline
	\bf Slack messages with the client & 1300 \\
	\hline
	\bf Median time per issue  & 8.68 hours \\
	\hline
	\bf Median time per task & 3.55 hours \\
    \hline
  \end{tabular} \\
  \caption{Collaboration statistics}
  \label{tab:collaborationstatistics}
\end{table}

% TODO  * test examples
% TODO test results (red/green)
